

% Gdzieniegdzie w pytaniach są oznaczenia W12-xx i Ox. Cholera wie co to, może jakieś identyfikatory w bazie danych pytań. Nie przepisywałem ich. %

\newpage
\section{Windows API}

\begin{enumerate}
	
	\item \question{Do funkcji Windows APi należą:}
	{Tak}{CreateWindowsEx}
	{Nie}{strcmp}
	{Tak}{ShowWindow}
	{Nie}{atoi}
	
	\item \question{Kiedy musi być zarejestrowana klasa okna w Windows API}
	{Nie}{klasa okna może być zarejestrowana zarówno przed jak i po utworzeniu okna}
	{Tak}{przed utworzeniem okna}
	{Nie}{po utworzeniu okna}
	{Nie}{klasa okna nie jest rejestrowana w Window API}
	
	\item \question{HWND:}
	{Nie}{Jest strukturą przechowującą wskaźniki do poszczególnych okien aplikacji}
	{Nie}{Jest wskaźnikiem na funkcję obsługującą komunikaty napływające do okna aplikacji}
	{Tak}{Jest uchwytem okna aplikacji}
	{Nie}{Jest funkcją pozwalającą na zdefiniowanie głównego okna aplikacji}
	
	\item \question{Aby wyświetlić krótki komunikat dla użytkownika przy użyciu okna modalnego można użyć funkcji}
	{Nie}{ShowDialog(...)}
	{Nie}{MsgBox(...)}
	{Tak}{MessageBox(...)}
	{Nie}{ShowModDialog(...)}
	
	\item \question{Kod programów pisanych z bezpośrednim wykorzystaniem Win32API musi zawierać:}
	{Nie}{Instrukcję $\sharp$include}
	{Nie}{Wywołanie funkcji CreateWindowEx(...)}
	{Tak}{Funkcję WinMain(...)}
	{Nie}{Funkcję WINAPI(...)}
	
	\item \question{Windows API pozwala na:}
	{Tak}{komunikację sieciową}
	{Tak}{ostęp do systemu plików}
	{Tak}{tworzenie interfejsu graficznego}
	{Tak}{dostęp do rejestrów systemu}

	\item \question{MDi w API jest skrótem od:}
	{Nie}{Media Download Interface}
	{Nie}{Mass Data Interface}
	{Tak}{Multiple Data Interface}
	{Nie}{Multicolor Data Interface}

	\item \question{UpdateWindow:}
	{Tak}{Jest funkcją wysyłającą komunikat do okna aplikacji informującym go o potrzebie przerysowania}
	{Nie}{Jest domyślną funkcją obsługującą przerysowanie okna lub jego fragmentu}
	{Nie}{Jest komunikatem wysyłanym do okna bezpośrednio po jego wyświetleniu}
	{Nie}{Jest komunukatem wysyłanym do okna aplikacji informującym go o potrzebe przerysowania}
	
	\item \question{Czy dany przycisk został naciśnięty możemy sprawdzić poprzez:}
	{Tak}{Porównanie uchwytu do przycisku wewnątrz procedury obsługi komunikatów przy zdarzeniu \texttt{WM\_COMMAND}}
	{Nie}{Porównanie adresu kontrolki przycisku}
	{Tak}{Porównanie ID przypisanego do przycisku wewnątrz procedury obsługi komunikatów przy zdarzeniu \texttt{WM\_COMMAND}}
	{Nie}{Wykonanie procesury obsługi przerwania danego przycisku}
	
	\item \question{Wyświetlenie okna Message Box:}
	{Nie}{Powoduje utworzenie dla niego nowego procesu w systemie}
	{Tak}{Jest wywołaniem blokującym (blokuje wykonanie dalszej części kodu aż do zamknięcia Message Box'a)}
	{Nie}{Polega na obsłudze odpowiedniego komunikatu w pętli obsługi komunikatów.}
	{Tak}{Możemy uzyskać poprzez wywołanie kodu: MessageBox(NULL, L"Welcome to Win32 Application Development$\backslash$n", NULL, NULL);}
	
	\item \question{DefWindowProc}
	{Tak}{Jest domyślną funkcją obsługującą komunikaty napływające do okna aplikacji}
	{Nie}{Jest wskaźnikiem na funkcję obsługującą komunikaty napływające do okna aplikacji}
	{Nie}{Jest funkcją pozwalającą na zdefiniowanie głównego okna aplikacji}
	{Nie}{Jest strukturą pozwalająca na m.in. zdefiniowanie głównego okna aplikacji}

	\item \question{Jakie rodzaje komunikatów mogą docierać do okna?}
	{Tak}{zmiana rozmiaru okna}
	{Tak}{pojedyncze bądź podwójne kliknięcie myszą w obszarze okna}
	{Tak}{zmiana położenia okna}
	{Tak}{naciśnięcie klawisza}
	
	\item \question{WNDCLASS$\slash$WNDCLASSEX}
	{Nie}{Obsługuje kolejkę komunikatów napływających do okna aplikacji}
	{Nie}{Jest strukturą przechowującą wskaźniki do poszczególnych okien aplikacji}
	{Tak}{Jest strukturą pozwalającą zdefiniować np. koloty okna aplikacji}
	{Nie}{Jest odpowiednikiem funkcji main() w programach pisanych w WinAPI}
	
\end{enumerate}