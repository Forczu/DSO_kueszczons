%\item \questionVIII{%
%	question={} %
%}{%
%	isTrue1=, %
%	answer1={}, %
%	isTrue2=, %
%	answer2={}, %
%	isTrue3=, %
%	answer3={}, %
%	isTrue4=, %
%	answer4={}, %
%	isTrue5=, %
%	answer5={}, %
%	isTrue6=, %
%	answer6={}, %
%	isTrue7=, %
%	answer7={}, %
%	isTrue8=, %
%	answer8={}, %
%	isTrue9=, %
%	answer9= {}, %
%	isTrue10=, %
%	answer10={},%
%	isTrue11=, %
%	answer11={}, %
%	isTrue12=, %
%	answer12={}.
%}

% !TeX spellcheck = pl_PL
\newpage
\section{Linux LAMP}
	\begin{enumerate}
		\item \questionVIII{%
				question=Zaznacz wszystkie poprawne stwierdzenia dotyczące rozwiązania LAMP: %
			}{%
				isTrue1=Nie, %
				answer1=Konfiguracja baz danych może odbywać się wyłącznie poprzez narzędzie phpMyAdmin., %
				isTrue2=Nie, %
				answer2=MySQL pozwala na wykonywanie kodu zapisanego w języku PHP na stronie wwww., %
				isTrue3=Tak, %
				answer3=Funkcją MySQL jest zarządzanie bazą danych., %
				isTrue4=Tak, %
				answer4=Podstawową funkcją serwera Apache jest przesyłanie do klienta treści plików znajdujących się na dysku przy wykorzystaniu protokołu HTTP., %
				isTrue5=Nie, %
				answer5=Kod PHP wewnątrz pliku z rozszerzeniem .html może znajdować się pomiędzy znacznikiem $ < $php$ > $ oraz znacznikiem $ < $/php$ > $., %
				isTrue6=Tak, %
				answer6=Kod PHP wewnątrz pliku z rozszerzeniem .php może znajdować się pomiędzy znacznikiem $ < $? oraz znacznikiem ?$ > $., %
				isTrue7=Tak, %
				answer7=Pliki konfiguracyjne serwera Apache znajdują się w katalogu /etc/apache2/, %
				isTrue8=Nie, %
				answer8=phpMyAdmin jest narzędziem do konfiguracji w trybie tekstowym., %
				isTrue9=Nie, %
				answer9= {Elementy LAMP to Apache, MySQL i Prolog}, %
				isTrue10=Tak, %
				answer10={Można powiedzieć, że dynamiczna strona internetowa stworzona w PHP na Linuksie, korzystająca z serwera Apache, z bazą danych MySQL jest opartą o LAMP.},%
				isTrue11=Nie, %
				answer11=Jako język programowania stron w LAMP można wykorzystać wyłącznie PHP., %
				isTrue12=Tak, %
				answer12={Elementy LAMP zostały stworzone jako osobne rozwiązania, ale razem stanowią popularną platformę systemową.},%
				isTrue13=Nie, %
				answer13={Kod w HTML wymaga kompilacji zanim zostanie umieszczony na serwerze.}, %
				isTrue14=Nie, %
				answer14={Narzędzie phpMyAdmin służy do konfiguracji serwera Apache.}, %
				isTrue15=Tak, %
				answer15={MySql może być użyty jako serwer bazy danych.}, %
				isTrue16=Tak, %
				answer16={PHP może być użyty do tworzenia stron dynamicznych.},%
				isTrue17=Nie, %
				answer17={Tylko administrator może korzystać z narzędzia phpMyAdmin.}, %
				isTrue18=Nie, %
				answer18={PostgreSQL może być użyty jako język skryptowy do tworzenia stron dynamicznych.}, %
				isTrue19=Nie, %
				answer19={Kod w PHP wymaga kompilacji zanim zostanie umieszczony na serwerze.}, %
			}
			
		\item \questionVIII{%
			question=Wskaż zdania prawdziwe dotyczące języka PHP%
		}{%
			isTrue1=Nie, %
			answer1={PHP wymaga by zmiennym nadawać typy.}, %
			isTrue2=Tak, %
			answer2={Nazwy zmiennych zaczynają się znakiem dolara.}, %
			isTrue3=Nie, %
			answer3={Jeśli kod PHP jest połączony ze znacznikami HTML, to musi się znajdować w pliku o rozszerzeniu phtml.}, %
			isTrue4=Nie, %
			answer4={Skrypt MUSI znajdować się w znacznikach $ < $?php ?$ > $ (żadnych innych)}, %
			isTrue5=Nie, %
			answer5={Funkcja mysql\_query() zwraca wynik w formie tablicy stringów.}, %
			isTrue6=Tak, %
			answer6={Skrypty PHP w typowych rozwiązaniach wykonywane są po stronie serwera.}, %
			isTrue7=Nie, %
			answer7={W pliku .php może wystąpić tylko jeden blok ograniczony znacznikami $ < $? i ?$ > $.}, %
			isTrue8=Tak, %
			answer8={PHP jest językiem interpretowanym.}, %
		}
		
		\newpage
		\item \questionVIII{%
			question=Język PHP: %
		}{%
			isTrue1=Nie, %
			answer1={Jest językiem kompilowanym}, %
			isTrue2=Tak, %
			answer2={Posiada biblioteki umożliwiające dostęp do bazy danych np. MySQL.}, %
			isTrue3=Tak, %
			answer3={Może być przeplatany z kodem HTML.}, %
			isTrue4=Nie, %
			answer4={Jest statycznie typowany.}, %
			isTrue5=Tak, %
			answer5={może być przeplatany z językiem HTML.}, %
			isTrue6=Nie, %
			answer6={wymaga deklarowania zmiennych.}, %
			isTrue7=Tak, %
			answer7={nie wymaga deklarowania zmiennych.}, %
			isTrue8=Tak, %
			answer8={wymaga, aby każda zmienna była poprzedzona znakiem \$.}, %
		}
		
		\item \question{Każda zmienna w PHP poprzedzona jest znakiem:}%
			{Nie}{\%}%
			{Nie}{\#}%
			{Nie}{$ < $?}%
			{Tak}{\$}
			
		\item \questionVIII{%
			question={W jaki sposób w języku PHP można odczytać dane (lub ich część) przesłane przez formularz na stronie internetowej (pobrać dane z formularza)?}%
		}{%
			isTrue1=Tak, %
			answer1={Używając tablicy \$\_POST}, %
			isTrue2=Nie, %
			answer2={Używając tablicy \$\_SEND\_DATA}, %
			isTrue3=Tak, %
			answer3={Używając tablicy \$\_GET}, %
			isTrue4=Tak, %
			answer4={Używając tablicy \$\_REQUEST}, %
			isTrue5=Nie, %
			answer5={Używając tablicy \$\_DATA}, %
			isTrue6=Nie, %
			answer6={Używając tablicy \$\_RESPONSE}, %
		}
			
		\item \question{Skrypty PHP:}%
			{Nie}{Są wykonywane po stronie przeglądarki internetowej klienta.}%
			{Tak}{Mogą zostać osadzone w plikach HTML.}%
			{Tak}{ZAWSZE rozpoczynają się od: $ < $?php .}%
			{Tak}{Mogą być zdefiniowane w osobnych plikach, bez osadzania w kodzie HTML.}
			
		\item \questionVIII{%
			question=Od jakich elementów systemu pochodzi określenie LAMP?
		}{%
			isTrue1=Nie, %
			answer1={Linux, Apache, McEdit, Perl}, %
			isTrue2=Tak, %
			answer2={Linux, Apache, MySQL, Perl}, %
			isTrue3=Nie, %
			answer3={Linux, Access, McEdit, PHP}, %
			isTrue4=Tak, %
			answer4={Linux, Apache, MySQL, Python}, %
			isTrue5=Nie, %
			answer5={Linux, Apache, MySQL, PHP}, %
		}
		
		\item \question{W skład LAMP wchodzi:}%
			{Nie}{PostgreSQL}%
			{Tak}{Linux}%
			{Tak}{Perl}%
			{Nie}{Windows}
		
		\newpage
		\item \question{Co może oznaczać „P” w skrócie LAMP?}%
			{Nie}{PostgreSQL}%
			{Tak}{Perl}%
			{Tak}{Python}%
			{Tak}{PHP}
				
		\item \question{Do poprawnego działania LAMP pod Linuxem potrzebny jest:}%
			{Tak}{PHP}%
			{Tak}{Apache}%
			{Tak}{MySQL}%
			{Tak}{Pakiety wiążące ze sobą pozostałe składniki.}%
			
		\item \question{Jaki serwer www wchodzi w skład LAMP?}%
			{Nie}{MySQL.}%
			{Nie}{IIS}%
			{Tak}{Apache}%
			{Nie}{Zależy od konfiguracji}
			
		\item \question{Serwer Apache:}%
			{Tak}{Jest serwerem www.}%
			{Tak}{Można zainstalować osobno.}%
			{Nie}{Można zainstalować tylko razem z serwerem bazy danych MySQL oraz bibliotekami języka PHP.}%
			{Tak}{Współpracuje z interpreterem języka PHP po doinstalowaniu odpowiednich pakietów.}
			
		\item \question{Moduł userdir serwera Apache umożliwia:}%
			{Nie}{Edycję ustawień dotyczących folderów znajdujących się w pliku konfiguracyjnym serwera Apache.}%
			{Tak}{Zakładanie stron poprzez dodawanie folderu public\_html w katalogu domowym użytkownika.}%
			{Tak}{Proste dodawanie stron www użytkownikom systemu.}%
			{Nie}{Dostęp do założonych stron użytkownika poprzez adres http://localhost/?NAZWA\_UZYTKOWNIKA}
		
		\item \question{Przy prawidłowo działającym w domyślnej konfiguracji module userdir zawartość strony http://localhost/~joe to:}%
			{Tak}{zawartość folderu public\_html w katalogu domowym użytkownika joe.}%
			{Nie}{zawartość folderu localhost na pulpicie użytkownika joe.}%
			{Nie}{zawartość folderu Joe na dysku C.}%
			{Nie}{zawartość folderu www w katalogu domowym użytkownika joe.}
			
		\newpage
		\item \question{Po włączeniu w Apache modułu userdir, pliki umieszczone przez użytkownika "jan" w folderze "/home/jan/public\_html" będą (...) :}%
			{Tak}{http://localhost/$\sim$jan}%
			{Nie}{http://localhost/home/jan/public\_html}%
			{Nie}{http://localhost/jan}%
			{Tak}{http://127.0.0.1/$\sim$jan}
			
		\item \questionVIII{%
			question={Które z podanych niżej operacji są prawidłowe, aby włączyć dowolny moduł w serwerze Apache?}%
		}{%
			isTrue1=Nie, %
			answer1={Skorzystanie z polecenia /etc/init.d/apache2 restart}, %
			isTrue2=Tak, %
			answer2={Utworzenie łącza symbolicznego w katalogu mods-enabled do pliku z katalogu mods-available.}, %
			isTrue3=Tak, %
			answer3={Skorzystanie z polecenia 'a2enmod'}, %
			isTrue4=Nie, %
			answer4={Skorzystanie z polecenia 'anenmod $ < $ nazwa\_modulu $ > $'}, %
			isTrue5=Nie, %
			answer5={Skorzystanie z polecenia /etc/init.d/apache2 force-reload} %
		}
			
		\item \question{Jakim poleceniem można zrestartować serwer Apache w celu odświeżenia konfiguracji?}
			{Tak}{/etc/init.d/apache2 restart}%
			{Tak}{/etc/init.d/apache2 stop \&\& /etc/init.d/apache2/start}%
			{Nie}{/etc/init.d/apache2 refresh}%
			{Nie}{apache2-restart}%
			
		\item \question{Konfiguracja serwera Apache w systemie Ubuntu Server:}
			{Nie}{wpływa na działający serwer zarz po zapisaniu pliku.}%
			{Tak}{w przypadku modułów serwera opiera się o dowiązania plików.}%
			{Tak}{znajduje się w folderze /etc/apache2.}%
			{Tak}{jest wstępnie przygotowana po zainstalowaniu serwera.}%
			
		\item \question{Który z modułów odpowiada za włączenie obsługi języka PHP w serwerze Apache?}%
			{Nie}{status}%
			{Nie}{proxy}%
			{Tak}{php5}%
			{Nie}{userdir}
			
		\item \question{Który z modułów pozwala dodawać strony www w Apache poprzez utworzenie katalogu public\_html w katalogu domowym?}%
			{Nie}{status}%
			{Nie}{proxy}%
			{Nie}{php5}%
			{Tak}{userdir}
			
		\item \question{Które moduły należy uruchomić, aby była możliwość dodawania stron www przez zwykłego użytkownika?}%
			{Nie}{usertrack}%
			{Nie}{proxy}%
			{Nie}{cache}%
			{Tak}{userdir}
		
		\newpage
		\item \question{Do wybrania bazy danych w MySQL w języku PHP służy funkcja:}%
			{Nie}{mysql\_db\_name}%
			{Nie}{mysql\_connect\_db}%
			{Nie}{mysql\_select\_db}%
			{Tak}{mysqli\_connect}
		
		\item \question{Jaki będzie wynik polecenia w języku skryptowym PHP:\\
			mysql\_connect("server:db", password, user) ?}%
			{Nie}{Połączenie się z bazą "db:server" na lokalnym komputerze.}%
			{Nie}{Połączenie się z bazą "server" na serwerze "db".}%
			{Nie}{Połączenie się z bazą "db" na serwerze "server".}%
			{Tak}{Zwrócenie błędu.}
		
	\end{enumerate}





