%\item \question{}%
%{Tak}{}%
%{Nie}{}%
%{}{}%
%{}{}

% !TeX spellcheck = pl_PL
\newpage
\section{Linux LAMP}
\begin{enumerate}
	\item \questionVIII{%
			question=Zaznacz wszystkie poprawne stwierdzenia dotyczące rozwiązania LAMP: %
		}{%
			isTrue1=Nie, %
			answer1=Konfiguracja baz danych może odbywać się wyłącznie poprzez narzędzie phpMyAdmin., %
			isTrue2=Nie, %
			answer2=MySQL pozwala na wykonywanie kodu zapisanego w języku PHP na stronie wwww., %
			isTrue3=Tak, %
			answer3=Funkcją MySQL jest zarządzanie bazą danych., %
			isTrue4=Tak, %
			answer4=Podstawową funkcją serwera Apache jest przesyłanie do klienta treści plików znajdujących się na dysku przy wykorzystaniu protokołu HTTP., %
			isTrue5=Nie, %
			answer5=Kod PHP wewnątrz pliku z rozszerzeniem .html może znajdować się pomiędzy znacznikiem $ < $php$ > $ oraz znacznikiem $ < $/php$ > $., %
			isTrue6=Tak, %
			answer6=Kod PHP wewnątrz pliku z rozszerzeniem .php może znajdować się pomiędzy znacznikiem $ < $? oraz znacznikiem ?$ > $., %
			isTrue7=Tak, %
			answer7=Pliki konfiguracyjne serwera Apache znajdują się w katalogu /etc/apache2/, %
			isTrue8=Nie, %
			answer8=phpMyAdmin jest narzędziem do konfiguracji w trybie tekstowym. %
		}
\end{enumerate}





