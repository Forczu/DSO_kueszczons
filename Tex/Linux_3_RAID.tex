%\item \question{}%
%{Tak}{}%
%{Nie}{}%
%{}{}%
%{}{}

% !TeX spellcheck = pl_PL
% *****************************************************
% Możliwe, że coś się wymieszało z Windowsem,
% do sprawdzenia bliżej laborek
% *****************************************************
\newpage
\section{Linux RAID}
\begin{enumerate}
	\item \question{Macierz typu raid 5 złożona z 3 dysków o jednakowej pojemności i parametrach:}%
	{Nie}{ma pojemność 2 dysków i nie jest odporna na awarię ani jednego dysku}%
	{Tak}{oferuje spowolniony odczyt w przypadku awarii 1 dysku}%
	{Nie}{ma pojemność 1 dysku i jest odporna na awarię maksymalnie 2 dysków}%
	{Tak}{ma pojemność 2 dysków i jest odporna na awarię maksymalnie 1 dysku}
	\item \question{W systemie Ubuntu, zakładając, że pliki blokowe /dev/sdb1 i /dev/sdb2 reprezentują partycje o rozmiarze 50MB, bezpośrednio po utworzeniu woluminu /dev/md0 poleceniem:\\
		mdadm $ -- $create $ -- $verbose /dev/md0 $ -- $level=linear $ -- $raid-devices=2\\/dev/sdb1/dev/sdb2:}%
	{Tak}{wolumin /dev/md0 będzie miał wielkość 100MB}%
	{Nie}{wolumin /dev/md0 będzie miał wielkość 50MB}%
	{Nie}{wolumin /dev/md0 będzie można zamontować poleceniem mount /dev/md0 /mnt}%
	{Tak}{uszkodzenie dokładnie jednego spośród urządzeń /dev/sdb1 oraz /dev/sdb2 może spowodować utratę danych w woluminie /dev/md0}
	\item \question{Zaznacz prawdziwe stwierdzenia:}%
	{Tak}{Sprzętowy RAID oferuje większą wydajność poprzez zmniejszenie obciążenia CPU, gdyż przeliczaniem sum kontrolnych zajmuje się wówczas dedykowany kontroler.}%
	{Nie}{RAID sprzętowy jest niekompatybilny z dużą liczbą systemów operacyjnych, ze względu na zachowanie odróżniające taki RAID od pojedynczego dysku twardego.}%
	{Tak}{RAID software'owy oferuje możliwość łączenia różnych interfejsów takich jak ATA, SCSI, SATA, USB w obrębie jednej macierzy.}%
	{Nie}{Dla takich samych dysków RAID 6 oferuje większą szybkość zapisu niż RAID 0.}
	\item \question{RAID5 może składać się z następującej ilości dysków:}%
	{Nie}{2}%
	{Tak}{3}%
	{Tak}{4}%
	{Tak}{5}
	\item \question{RAID inaczej zwanym lustrzanym (mirroringiem) to:}%
	{Tak}{RAID1}%
	{Nie}{RAID2}%
	{Nie}{RAID3}%
	{Nie}{RAID5}
	\item \question{Jakie polecenie pozwoli na rozpoczęcie procedury tworzenia partycji:}%
	{Tak}{fdisk /dev/hda}%
	{Nie}{mkdir /dev/sda}%
	{Tak}{fdisk /dev/sdb}%
	{Nie}{mdadd /dev/sdb}
	\item \question{Jaka ilość dysków jest wystarczająca, aby zastosować RAID 5:}%
	{Nie}{1}%
	{Nie}{2}%
	{Tak}{3}%
	{Tak}{4}
	
	\newpage
	\item \question{Mając do dyspozycji 3 identyczne dyski twarde można stworzyć macierz RAID w konfiguracji:}%
	{Tak}{RAID 0}%
	{Tak}{RAID 5}%
	{Nie}{RAID 6}%
	{Nie}{RAID 10}
	\item \question{Trzy dyski zostały połączone w macierz RAID 0.}%
	{Nie}{Łączna przestrzeń dyskowa jest równa sumie przestrzeni, każdego z dysków}%
	{Tak}{Łączna przestrzeń dyskowa jest równa potrojonej przestrzeni dyskowej najmniejszego dysku}%
	{Tak}{Szybkość jest równa potrojonej szybkości najwolniejszego z dysków}%
	{Nie}{Szybkość jest równa szybkości najwolniejszego z dysków}
	\item \question{Zaznacz cele zastosowania macierzy RAID:}%
	{Tak}{Zwiększenie odporności na awarie}%
	{Tak}{Zwiększenie wydajności transmisji danych}%
	{Tak}{Powiększenie przestrzeni dyskowej, dostępnej jako jedna całość}%
	{Nie}{Dwukrotne zwiększenie całkowitej przestrzeni dyskowej}
	\item \question{Administrator podłączył do komputera dwa dyski twarde o pojemności 200GB każdy i połączył je w macierz RAID 1. Do komputera nie zostały podłączone żadne inne dyski. Które z poniższych twierdzeń są prawidłowe?}%
	{Tak}{Całkowita pojemność partycji dostępnych w systemie nie przekracza 200GB.}%
	{Nie}{Rozwiązanie takie zapewnia o wiele większą prędkość odczytu i zapisu danych niż macierz RAID 0.}%
	{Tak}{Rozwiązanie takie zapewnia o wiele większe bezpieczeństwo danych niż macierz RAID 0.}%
	{Nie}{W przypadku awarii jednego dysku użytkownik straci wszystkie swoje dane}
	\item \question{Zaznacz zdania prawdziwe dotyczące sprzętowej macierzy RAID:}%
	{Tak}{Macierz jest zupełnie przezroczysta, przez co z punktu widzenia Systemu Operacyjnego zachowuje się ona jak każdy inny dysk twardy}%
	{Nie}{mniejsza wydajność poprzez zwiększenie obciążenia CPU}%
	{Tak}{Minimalna liczba dysków potrzebna do stworzenia macierzy to 2}%
	{Nie}{Sprzętowa macierz RAID zawsze umożliwia przywrócenie danych w razie awarii jednego z dysków}
	\item \question{Zaznacz zdania prawdziwe dotyczące programowej macierzy RAID:}%
	{Nie}{Macierz jest zupełnie przezroczysta, przez co z punktu widzenia Systemu Operacyjnego zachowuje się ona jak każdy inny dysk twardy}%
	{Tak}{mniejsza wydajność poprzez zwiększenie obciążenia CPU}%
	{Tak}{Minimalna liczba dysków potrzebna do stworzenia macierzy to 2}%
	{Nie}{Programowa macierz RAID zawsze umożliwia przywrócenie danych w razie awarii jednego z dysków}
	
	\newpage
	\item \question{System Linux pozwala na:}%
	{Tak}{Tworzenie programowych macierzy RAID.}%
	{Tak}{Tworzenie wolumenów liniowych.}%
	{Nie}{Tworzenie partycji za pomocą polecenia "create"}%
	{Tak}{Tworzenie macierzy RAID 5.}
	\item \question{Woluminy liniowe w katalogu dev oznaczone są jako:}%
	{Tak}{md0,md1,...}%
	{Nie}{ma0,ma1,...,mb0,mb1,...}%
	{Nie}{raid0,raid1,...}%
	{Nie}{rda0,rda1,...,rdb0,rdb1,...}
	\item \question{Za pomocą polecenia mdadm można:}%
	{Tak}{utworzyć wolumin liniowy}%
	{Nie}{Sformatować partycję}%
	{Tak}{Sprawdzić konfigurację macierzy}%
	{Tak}{Zasymulować awarię woluminu}
	\item \question{Która z aplikacji umożliwia stworzenie partycji na twardym dysku?}%
	{Nie}{/etc/fstab}%
	{Tak}{/sbin/fdisk}%
	{Tak}{/sbin/cfdisk}%
	{Nie}{/etc/mtab}
	\item \question{Wskaż poprawne zdania dotyczące RAID.}%
	{Nie}{Polecenie „mdadm -C -v /dev/md0 -{}-level=0 -n 2 /dev/sda1 /dev/sdb1” służy do stworzenia wolumenu liniowego na partycjach sda1 i sdb1.}%
	{Tak}{Polecenie „mdadm -C -v /dev/md0 -{}-level=1 -n 2 /dev/sda1 /dev/sdb1” służy do stworzenia mirroru.}%
	{Tak}{Polecenie „mkfs -t ext3 /dev/md0” służy do sformatowania urządzenia.}%
	{Nie}{Wolumenu liniowego /dev/md0 nie można dodać do pliku /etc/fstab, aby była montowana przy starcie systemu operacyjnego.}
	\item \question{Które z wymienionych rodzajów macierzy RAID zapewniają mirroring:}%
	{Nie}{RAID 0}%
	{Tak}{RAID 1}%
	{Tak}{RAID 5}%
	{Tak}{RAID 10}
	\item \question{Które z wymienionych poleceń umożliwia zarządzanie macierzami RAID w systemie GNU/Linux:}%
	{Nie}{hdparm}%
	{Tak}{mdadm}%
	{Nie}{fdisk}%
	{Nie}{parted}
	
	\newpage
	\item \question{Celem wyłączenia automatycznego montowania urządzenia cdrom w systemie Linux należy:}%
	{Tak}{Odpowiednio zmodyfikować plik '/etc/fstab'.}%
	{Nie}{Wykonać polecenie 'nmount -n cdrom'.}%
	{Nie}{Wykonać polecenie 'nmount cdrom'.}%
	{Nie}{Odpowiednio zmodyfikować plik '/etc/amount'.}
	\item \question{Polecenie 'fdisk' w systemie Linux można wykorzystać do:}%
	{Tak}{tworzenia partycji.}%
	{Tak}{wypisania informacji o dysku.}%
	{Nie}{montowania dysku.}%
	{Nie}{tworzenia kopii zapasowej danych.}
	\item \question{Wskaż poprawne odpowiedzi dotyczące RAID5:}%
	{Tak}{Umożliwia odzyskanie danych w razie awarii jednego z dysków}%
	{Nie}{Składa się z minimum 2 dysków}%
	{Nie}{Odzyskiwanie danych w razie awarii odbywa się przy wykorzystaniu danych i kodów korekcyjnych zapisanych na jednym, specjalnie do tego przeznaczonym dysku}%
	{Tak}{W przypadku awarii dysku dostęp do danych jest spowolniony}
	\item \question{Wskaż poprawne odpowiedzi dotyczące mirroring-u:}%
	{Tak}{Polega na zapisywaniu tych samych danych na dwóch lub więcej dyskach jednocześnie}%
	{Nie}{W przypadku awarii co najmniej połowy z dysków nie ma możliwości odzyskania wszystkich danych}%
	{Tak}{Dostępna przestrzeń ma rozmiar najmniejszego nośnika}%
	{Tak}{Czas równoległego zapisu jest równy czasowi zapisu na najwolniejszym dysku}
	\item \question{Wskaż poprawne zdania dotyczące RAID5 w systemie Linux:}%
	{Nie}{Do utworzenia RAID5 potrzebne są co najmniej dwie partycje.}%
	{Nie}{Do utworzenia RAID5 można użyć maksymalnie trzech partycji.}%
	{Nie}{Do odtworzenia danych z uszkodzonej partycji zawsze wykorzystywana jest jedna, specjalnie do tego przygotowanej partycja.}%
	{Tak}{RAID5 jest całkowicie odporny na uszkodzenie jednej partycji (dane można w pełni odtworzyć).}
	\item \question{Wskaż poprawne zdania dotyczące RAID1 (mirror) w systemie Linux.}%
	{Tak}{Całkowita pojemność partycji połączonych w RAID1 jest taka jak pojemność najmniejszej z tych partycji.}%
	{Tak}{Do utworzenia RAID1 można wykorzystać trzy partycje.}%
	{Nie}{Zastosowanie RAID1 pozwala na zwiększenie szybkości zapisu i odczytu danych.}%
	{Tak}{RAID1 jest całkowicie odporny na uszkodzenie jednej partycji (dane można w pełni odtworzyć).}
	\item \question{Które z poniższych funkcji macierzy RAID zwiększają bezpieczeństwo danych?}%
	{Tak}{mirroring (lustrzane odbicie)}%
	{Nie}{stripping (paskowanie)}%
	{Nie}{macierze liniowe}%
	{Tak}{kontrola parzystości}
	
	\newpage
	\item \question{Trzy dyski, każdy o pojemności 1TB, połączyliśmy w macierz RAID5. Jaką pojemnośd ma uzyskany wolumien?}%
	{Nie}{0.5 TB}%
	{Nie}{1 TB}%
	{Tak}{2 TB}%
	{Nie}{3 TB}
	\item \question{Zaznacz poprawną odpowiedz dotyczącą RAID:}%
	{Tak}{RAID pozwala łączyć ze sobą dyski celem stworzenia pamięci masowej o dużej pojemności I niezawodności}%
	{Tak}{macierz RAID można stworzyć za pomocą sprzętowych kontrolerów oraz systemowych narzędzi}%
	{Nie}{do utworzenia RAID5 wystarczą dwa dyski}%
	{Nie}{nie da stworzyć się macierzy dyskowej z dwóch dysków}
	\item \question{Skrót RAID oznacza:}%
	{Tak}{Redundant Array of Independent Disks}%
	{Nie}{Redundant Array of Independent Drives}%
	{Nie}{Remote Array of Independent Disks}%
	{Nie}{Reserved Array of Independent Disks}
	\item \question{Macierz RAID 5 charakteryzuje się}%
	{Nie}{Zastosowaniem minimum 2 dysków}%
	{Tak}{Zastosowaniem minimum 3 dysków}%
	{Nie}{Odpornością na awarię dwóch dysków}%
	{Tak}{Zmniejszoną szybkością zapisu}
	\item \question{Macierz RAID 0 używana jest do:}%
	{Tak}{Poprawy wydajności zapisu}%
	{Nie}{Zabezpieczeniem danych przed awarią dysku kosztem dostępnego miejsca}%
	{Nie}{Zabezpieczeniem danych przed awarią dysku kosztem czasu dostępu}%
	{Nie}{Skrócenia czasu odbudowy macierzy}
	\item \question{Co jest zawartością pliku /proc/mdstat ?}%
	{Tak}{Konfiguracje RAID}%
	{Tak}{Aktualny stan macierzy}%
	{Nie}{Standardowe procery obsługi RAID}%
	{Nie}{Listę uruchomionych procesów}
	\item \question{Aby połączyć dwa wolumeny w wolumen liniowy użyjemy instrukcji:}%
	{Tak}{mdadm –create –verbose /dev/md0/ $ -- $level=linear –raid-dervices=2 /dev/sdb1 /dev/sdb2}%
	{Nie}{Mdfs –create –verbose /dev/md0/ $ -- $level=linear –raid-dervices=2 /dev/sdb1 /dev/sdb2}%
	{Nie}{mdadm –create –verbose /dev/md0/ $ -- $level=raid1 –raid-dervices=2 /dev/sdb1 /dev/sdb2}%
	{Nie}{mdadm –new –verbose /dev/md0/ $ -- $level=linear –raid-dervices=2 /dev/sdb1 /dev/sdb2}
	\item \question{Zaznacz poprawne twierdzenia na temat RAID 0 :}%
	{Nie}{Zapewnia ochronę przed utratą danych}%
	{Tak}{Zapewnia zwiększoną wydajność zapisu}%
	{Tak}{Zapewnia zwiększoną wydajność odczytu}%
	{Nie}{Do jej stworzenia potrzebne są minimalnie 3 dyski}
	
	\newpage
	\item \question{Na komputerze została stworzona macierz RAID 1 złożona z 3 partycji sda1, sdb1 i sdc1, wszystkie dyski pracuja poprawnie i nie są uszkodzone, co się stanie w momencie wywołania komendy:\\'mdadm /dev/md0 -- remove /dev/sda1'}%
	{Nie}{Partycja sda1 zostanie usunięta z macierzy md0}%
	{Tak}{Nic}%
	{Tak}{Partycja sda1 zostanie usunięta z macierzy jeśli przedtem wywołano komendę 'mdadm $ -- $fail /dev/md0 /dev/sda1'}%
	{Nie}{Macierz md0 zostanie usunięta}
	\item \question{Wskaż typy macierzy dyskowych, które do ochrony danych wykorzystują sumy kontrolne}%
	{Nie}{RAID 0}%
	{Nie}{RAID 1}%
	{Tak}{RAID 3}%
	{Tak}{RAID 5}
	\item \question{Cztery dyski twarde o rozmiarach 200GB 200GB 150GB 150GB połączono w macierz typu striped volume:}%
	{Nie}{Macierz taka jest bardziej odporna na awarie niż pojedynczy dysk}%
	{Tak}{Sumaryczna szybkość takiej macierzy jest 4-krotnością szybkości najwolniejszego z dysków}%
	{Nie}{Macierz jest widziana w systemie jako pojedynczy dysk logiczny o rozmiarze 700GB}%
	{Tak}{Prawdopodobieństwo utraty danych jest większe niż dla analogicznej macierzy RAID 1}
	\item \question{Zaznacz prawdziwe zdania dotyczące RAID5.}%
	{Nie}{RAID5 polega na tworzeniu kopi danych na rożnych dyskach (mirroring)}%
	{Nie}{Macierz składa się z 5 lub więcej dysków}%
	{Nie}{Macierz składająca się z n dysków jest odporna na awarię n – 2 dysków}%
	{Tak}{Wszystkie powyższe odpowiedzi są nie poprawne}
	\item \question{W maszynie zainstalowana jest macierz RAID. Jeden z dysków podlega awarii. Zaznacz zdania prawdziwe.}%
	{Tak}{Dla macierzy RAID 5 po wymianie uszkodzonego dysku dane zostaną odbudowane.}%
	{Nie}{Macierz RAID 1 przestanie funkcjonować.}%
	{Tak}{Jeśli zainstalowane były 3 dyski, macierz RAID 1 pozwoli na dalsza pracę bez utraty danych.}%
	{Tak}{Macierz RAID 5 nie wymaga wymiany dysku na nowy przed wznowieniem pracy.}
	\item \question{Wpisanie polecenia fdisk /dev/hda oraz p spowoduje:}%
	{Nie}{sformatowanie dysku hda}%
	{Tak}{wypisanie listy partycji istniejących na dysku hda}%
	{Nie}{utworzenie na dysku hda partycji zajmującej całą dostępną przestrzeń}%
	{Nie}{uruchomienie systemu operacyjnego z dysku hda}
	
	\newpage
	\item \question{Wskaż cechy RAID 5:}%
	{Nie}{bity parzystości są zapisywane na specjalnie do tego przeznaczonym dysku}%
	{Nie}{szybkość dostępu do danych nie ulega zmianie w wypadku awarii jednego z dysków}%
	{Tak}{gwarantuje stuprocentowe bezpieczeństwo danych przy awarii jednego dysku}%
	{Tak}{jego zaletą jest szybki odczyt, jego wada to powolny zapis}
	\item \question{Co odróżnia macierze RAID programowe od sprzętowych?}%
	{Tak}{Obsługą macierzy programowych zajmuje się odpowiednie oprogramowanie, np. mdadm.}%
	{Nie}{Macierze programowe mają większą wydajność w porównaniu do sprzętowych.}%
	{Nie}{Problem awarii fizycznego nośnika w żaden sposób nie dotyczy macierzy programowych.}%
	{Nie}{W macierzach programowych problem awarii fizycznego dotyczy jedynie poziomu RAID 0.}
	\item \question{Wykonywanie jakich czynności związanych z macierzami RAID umożliwia polecenie „mdadm” w systemach z rodziny Linux?}%
	{Tak}{Podłączanie nowych urządzeń do macierzy.}%
	{Tak}{Generowanie zawartości plików konfiguracyjnych macierzy.}%
	{Tak}{Sprawdzanie statusu macierzy.}%
	{Tak}{Programowe symulowanie awarii w macierzy.}
	\item \question{Polecenie mount umożliwia: Wskaż wszystkie poprawne odpowiedzi}%
	{Tak}{Zamontowanie wszystkich partycji wymienionych w fstab}%
	{Nie}{Odmontowanie wszystkich partycji wymienionych w fstab}%
	{Nie}{Odmontowanie partycji}%
	{Tak}{Zamontowanie partycji}
	
	
\end{enumerate}
