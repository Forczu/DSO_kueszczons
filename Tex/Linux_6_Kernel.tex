% !TeX spellcheck = pl_PL
\newpage
\section{Linux Kernel}

\begin{itemize}
	
	
	\item \questionVIII{%
		question={Zaznacz wszystkie poprawne odpowiedzi:}%
	}{%
		isTrue1=Nie, %
		answer1={Jądro Linuxa jest mikrojądrem (microkernel)}, %
		isTrue2=Nie, %
		answer2={Jądro Linuxa jest jądrem typu hybrydowego (hybrid)}, %
		isTrue3=Tak, %
		answer3={Jądro Linuxa jest jądrem typu monolitycznego (monolythic)}, %
		isTrue4=Nie, %
		answer4={Jądro Linuxa jest napisane w C++}, %
		isTrue5=Nie, %
		answer5={Jądro Linuxa wykorzystuje bibliotekę libc (dzięki temu można wykorzystywać np. funkcję printf()}, %
		isTrue6=Tak, %
		answer6={Jądro Linuxa jest napisane w C},
		isTrue7=Tak,
		answer7={Jądro Linuxa zarządza pamięcią operacyjną (przydziały/zwolnienia)}.
	}
	
	\item \questionVIII{%
		question={Zaznacz wszystkie poprawne odpowiedzi:}%
	}{%
		isTrue1=Nie, %
		answer1={Do sterowania parametrami pracy jądra można wykorzystać pliki znajdujące się w katalogu \textbf{/var}}, %
		isTrue2=Tak, %
		answer2={Do sterowania pracą jądra Linuxa można wykorzystać polecenie \textbf{sysctl}}, %
		isTrue3=Tak, %
		answer3={Do jądra systemu operacyjnego Linux można, w czasie jego pracy, dołączać różnorodną funkcjonalność (np. sterowniki urządzenia)}, %
		isTrue4=Nie, %
		answer4={Do załadowania modułu w jądrze można wykorzystać polecenia rmmod oraz modprobe -r}, %
		isTrue5=Tak, %
		answer5={Do sterowania parametrami pracy jądra można wykorzystać pliki znajdujące się w katalogu \textbf{/proc}}, %
		isTrue6=Nie, %
		answer6={Do sterowania pracą jądra Linuxa można wykorzystać polecenie \textbf{sysinfo}}, %
		isTrue7=Nie, %
		answer7={Do usunięcia modułu z jądra można wykorzystać polecenie insmod}, %
		isTrue8=Tak, %
		answer8={Do sprawdzenia jakie moduły załadowane są do jądra można wykorzystać polecenie lsmod}, %
		isTrue9=Tak, %
		answer9={Do załadowania modułu w jądrze można wykorzystać polecenie modprobe oraz insmod}, %
		isTrue10=Tak, %
		answer10={Katalog \textbf{/proc} zawiera pliki, pozwalające na zmianę sposobu przydzielania pamięci programom przez jądro Linux},%
		isTrue11=Nie, %
		answer11={Katalog \textbf{/var} zawiera pliki, pozwalające na zmianę sposobu przydzielania pamięci programom przez jądro systemu Linux}, %
		isTrue12=Tak, %
		answer12={Do usunięcia modułu z jądra można wykorzystać polecenia modprobe oraz mmod},
		isTrue13=Tak, %
		answer13={Katalogi /proc, /sys oraz polecenie sysctl pozwalają na kontrolę pracy systemu},
		isTrue14=Tak,
		answer14={Z jądra systemu operacyjnego Linux, w trakcie jego pracy, można usuwać różnorodną funkcjonalność (na przykład sterowniki urządzenia)},
		isTrue15=Tak, %
		answer15={Do kontroli pracy systemu można wykorzystać polecenia sysctl oraz zawartość katalogu /proc}, %
		isTrue16=Tak, %
		answer16={Do sprawdzenia jakie moduły załadowane są do jądra można wykorzystać polecenie lsmod}, %
		isTrue17=Tak, %
		answer17={Do załadowania modułu w jądrze można wykorzystać polecenia modprobe oraz insmod}, %
		isTrue18=Nie, %
		answer18={Do kontroli pracy systemu można wykorzystać polecenia sysctl oraz zawartość katalogu /var}
	}
	
	
	\item \question{Zaznacz wszystkie funkcje realizowane przez jądro monolityczne (na przykład jądro Linuxa)}
	{Tak}{Szeregowanie procesów}
	{Tak}{Zarządzanie pamięcią (zwalnianie/przydzielanie)}
	{Tak}{Szeregowanie I/O}
	{Tak}{Obsługa systemu plików}
	
	\newpage
	\item \question{Jakie operacje można wykonać za pomocą polecenia sysctl?}
	{Tak}{Ustawić wartości dla parametrów jądra}
	{Nie}{Ustawić wartości dla parametrów systemu plików}
	{Tak}{Wypisać wszystkie parametry jądra w trakcie działania systemu}
	{Nie}{Wypisać wszystkie parametry systemu plików}
	

	
	\item \questionVIII{%
		question={Polecenie sysctl:} %
	}{%
		isTrue1=Nie, %
		answer1={Służy do zmiany hasła użytkownika systemu}, %
		isTrue2=Nie, %
		answer2={Umożliwia zmianę nazwy użytkownika}, %
		isTrue3=Nie, %
		answer3={Wyświetla listę użytkowników w systemie}, %
		isTrue4=Tak, %
		answer4={Pozwala na zmianę parametrów jądra systemu w trakcie działania systemu operacyjnego}, %
		isTrue5=Tak, %
		answer5={To komenda pozwalająca na konfiguracje parametrów jądra systemu Linux.}, %
		isTrue6=Tak, %
		answer6={Wykonuje konfigurację jaką można także wykonać w wirtualnym systemie plików /proc/sys.}, %
		isTrue7=Nie, %
		answer7={Pozwala na rekompilację jądra z uwzględnieniem nowych plików konfiguracyjnych.}, %
		isTrue8=Nie, %
		answer8={Wyświetla wszystkie procesy w systemie.}.
	}
	
	\item \question{Wskaż prawdziwe zdania:}
	{Nie}{przy overcommit\_memory ustawionym na 2 system zawsze przydzieli aplikacjom dokładnie 100\% pamięci RAM}
	{Tak}{przy overcommit\_memory ustawionym na 1 możliwe jest uzyskanie za pomocą malloc() ilości pamięci wirtualnej większej niż objętość pamięci fizycznej + swap}
	{Tak}{przy overcommit\_memory ustawionym na 2 ilość pamięci przydzielonej aplikacjom zależy od overcommit\_ratio}
	{Nie}{kernel nigdy nie przydziela więcej pamięci niż jest dostępne fizycznie}
	
	\item \question{Sterowanie jądrem systemu Linux. Zaznacz poprawne odpowiedzi:}
	{Nie}{Nawet najdrobniejsza zmiana w pracy jądra systemu wymaga jego ponownej kompilacji.}
	{Tak}{Możliwa jest zmiana niektórych parametrów jądra w "locie" korzystając z komendy sysctl.}
	{Nie}{Po każdej zmianie parametru w jądrze systemu Linux należy ponownie uruchomić komputer.}
	{Tak}{Wartości sysctl wczytywane są podczas startu systemu z pliku /etc/sysct.conf.}
	
	\item \question{Sterowniki w systemach Linuxowych: Wskaż poprawne odpowiedzi.}
	{Tak}{Można wkompilować w jądro, ale można ładować dynamicznie bez potrzeby wkompilowywania.}
	{Tak}{Mogą być ładowane dynamicznie w trakcie działania systemu.}
	{Nie}{Są tylko wkompilowane w jądro i uruchamiane automatycznie. Nie ma innej możliwości instalacji i uruchomienia.}
	{Nie}{Po instalacji nowego sterownika zawsze wymagane jest ponowne uruchomienie komputera.}
	
	\item \question{W jaki sposóbm ożna wyłączyć partycję SWAP?}
	{Nie}{Nie można wyłączyć partycji SWAP}
	{Nie}{sudo setswap off}
	{Tak}{sudo swapoff -a}
	{Nie}{sudo swap stop}
	
	\item \question{Jakie jest zadanie jądra w systemie Linux?}
	{Tak}{Ładuje i odładowuje sterowniki urządzeń.}
	{Nie}{Tylko i wyłącznie zarządza pamięcią.}
	{Tak}{Pośredniczy pomiędzy aplikacją użytkownika a sprzętem.}
	{Tak}{Zarządza pamięcią.}
	
	\item \question{Jądro w systemie Linux odpowiedzialne jest za:}
	{Tak}{Sterowniki urządzeń}
	{Nie}{Wygląd interfejsu graficznego}
	{Tak}{Zarządzanie procesami}
	{Tak}{Obsługę pamięci}
	
	\item \question{Moduły jądra systemu Linux: wskaż wszystkie poprawne odpowiedzi.}
	{Tak}{Można pisać w języku C}
	{Nie}{Mogą być załadowane przez każdego użytkownika}
	{Tak}{Nie posiadają możliwości wyprowadzania danych na standardowe wyjście stdout za pomocą printf}
	{Tak}{Można je kompilować na tym samym systemie na którym zamierzamy je uruchomić.}
	
	\item \question{Co znajduje się w katalogu /proc/?}
	{Tak}{Informacje o procesach w systemie}
	{Nie}{Informacje o użytkownikach}
	{Tak}{Informacje o sieci}
	{Tak}{Ogólne informacje o systemie}
	
	\item \question{Program modprobe:}
	{Nie}{wymaga restartu aby zmiany zostały wprowadzone}
	{Tak}{umożliwia usuwanie modułów z kernela}
	{Tak}{umożliwia ładowanie modułów kernela}
	{Tak}{automatycznie dodaje moduły zależne}
	
	\item \question{Parametry jądra systemu Linux można odczytać za pomocą:}
	{Nie}{pliku /proc/stat}
	{Tak}{katalogu /proc/sys}
	{Nie}{komendy ps}
	{Tak}{komendy sysctl}
	
	\item \question{Które z poniższych komend sprawdza logi jądra systemu Linux}
	{Tak}{dmesg}
	{Nie}{klog}
	{Nie}{kmllg}
	{Nie}{kernelog}
	
	\newpage
	\item \question{Jądro systemu operacyjnego Linux:}%
	{Tak}{pośredniczy pomiędzy aplikacjami użytkownika, a sprzętem}%
	{Tak}{pośredniczy pomiędzy aplikacjami użytkownika, a pamięcią}%
	{Nie}{służy wyłącznie do uruchomienia systemu i skonfigurowania urządzeń, potem kończy swoją pracę}%
	{Nie}{NIE pozwala na ładowanie dodatkowych modułów}
	
	\item \question{Które ze zdań dotyczących sysctl jest poprawne?}%
	{Tak}{Katalog /proc/sys dostarcza interfejs do parametrów sysctl}%
	{Tak}{/proc/sys/vm/overcommit\_memory jest odpowiednikiem parametru vm.overcommit\_memory w sysctl.conf}%
	{Nie}{jeżeli katalog /proc/sys jest tylko do odczytu to da się mimo to zmieniać parametry przez komendę sysctl}%
	{Nie}{Wartości sysctl są wczytywane przy starcie systemu z /proc/sys/vm/sysctl.conf}
	
	\item \question{Zaznacz prawdziwe zdania dotyczące partycji wymiany (SWAP) w systemie Linux:}%
	{Tak}{Domyślnie jest na niej zapisywany zrzut pamięci RAM przy hibernacji}%
	{Tak}{Można go aktywować i dezaktywować podczas działania systemu}%
	{Nie}{Jest zamontowana w katalogu /swap}%
	{Nie}{Jest konieczna do działania systemu Linux}
	
	
	
\end{itemize}