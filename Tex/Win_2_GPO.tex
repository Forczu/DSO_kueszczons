\newpage
\section{Obiekty Zasad Grup (GPO)}
	\begin{enumerate}
		\item \question{Na jakich poziomach w Active Directory mogą być przypisywane obiekty GPO?}%
		{Tak}{Lokalnie}% 
		{Nie}{Na poziomie lokacji}%
		{Tak}{Na poziomie domeny}%
		{Tak}{Na poziomie jednostki organizacyjnej}%
		
		\item \question{Aby wyświetlić wynikowy zestaw zasad dla użytkownia Sysop należy użyć polecenia:}%
		{Nie}{gpresult /gpo Sysop}%
		{Nie}{gpresult /?}%
		{Tak}{gpresult /user Sysop}%
		{Nie}{gpresult /u Sysop}
		
		\item \question{Wskaż prawdziwe zdania dotyczące GPO}%
		{Nie}{Akronin GPO rozwija się jako Group Policy Operation}%
		{Tak}{Za pomocą GPO Standard Desktop można zabronić dostępu do Panelu Sterowania}%
		{Nie}{Dane jednego GPO mogą być przypisane tylko jednej jednostce organizacyjnej}%
		{Nie}{Nie da się wyłączyć stosowania zasad GPO danej jednostki organizacyjnej bez usuwania GPO lu łącza obiektu}
		
		\item \question{Gdzie w rejestrze systemowym można znaleźć wpisy wynikające z GPO?}%
		{Tak}{HKEY LOCAL MACHINE (HKLM)}%
		{Nie}{HKEY CLASSES ROOT (HKCR)}%
		{Tak}{HKEY CURRENT USER (HKCU)}%
		{Nie}{HKEY USERS (HKU)}
		
		\item \question{W jaki sposób można modyfikować domyślne przetwarzanie obiektów zasad grupy?}%
		{Tak}{Blokując dziedziczenie zasad grupy}%
		{Nie}{Definiując warunkowe wprowadzanie ustawień.}%
		{Tak}{Wyłączając przetwarzanie konkretnego łącza GPO}%
		{Tak}{Wyłączając nadpisywanie ustawień wprowadzanych przez konkretne łącze GPO.}
		
		\item \question{Group Policy Management Console umożliwia:}%
		{Nie}{Wszystkie funkcje konsoli Power Shell, oraz dodatkowo funkcje zarządzania obiektami GPO}%
		{Tak}{Stworzenie kopii zapasowej obiektów GPO}%
		{Tak}{Łatwiejsze zarządzanie obiektami GPO, dzięki graficznemu interfejsowi użytkownika}%
		{Nie}{Tworzenie logów każdej operacji użytkownika w wybranej przez administratora grupie}
		
		\item \question{System Windows w ramach zarządzania GPO umożliwia:}%
		{Tak}{Filtrowanie ustawień GPO - wyłączenie stosowania określonych zasad GPO}%
		{Tak}{Wymuszanie stosowania zasad GPO}%
		{Tak}{Przeglądanie wdrażania elementów GPO dla danej jednostki organizacyjnej}%
		{Tak}{Blokowanie dziedziczenia ustawień obiektów GPO}
		
		\item \question{Które narzędzia służą do tworzenia i zarządzania GPO?}%
		{Tak}{Konsola Group Policy Management}%
		{Nie}{narzędzie gpadd}%
		{Tak}{Group Policy Object Editor z Active Directory Users and Computers}%
		{Nie}{narzędzie gpomod}
		
		\item \question{GPO jest to: Wskaż wszystkie poprawne odpowiedzi.}%
		{Tak}{Zbiór ustawień, który określa jak będzie się zachowywał i wyglądał system, dla zdefiniowanych grup użytkowników.}%
		{Nie}{Narzędzie administracyjne, służące do zarządzania zasadami grup.}%
		{Nie}{Obiekt, mogący istnieć tylko lokalnie zawierający zasady działania systemu dla grup użytkowników.}%
		{Nie}{Zbiór obiektów zawierający ustawienia dotyczące zasad działania systemu, po jednym obiekcie na każdego użytkownika.}
		
		\item \question{W jaki sposób można wyznaczyć efektywne ustawienia dla obiektów GPO?}%
		{Nie}{Za pomocą polecenia gpoeffective}%
		{Nie}{Za pomocaą Group Policy Preferences}%
		{Tak}{Za pomocą Group Policy Results Wizards}%
		{Tak}{Za pomocą polecenia gpresult}
		
		\item \question{Program "GPRESULT" służy do:}%
		{Nie}{Zmiany zasad GPO}%
		{Nie}{Zmiany efektywnych ustawień GPO}%
		{Nie}{Usuwania zasad GPO}%
		{Tak}{Przeglądania efektywnych ustawień GPO}
		
		\item \question{Które z podanych funkcji może pełnić narzędzie Group Policy Results w kontekście zasad grup?}%
		{Tak}{Generowanie raportów o wpływie zasad grupy na konkretnego użytkownika lub komputer.}%
		{Nie}{Rejestrowanie prób ominięcia lokalnych zabezpieczeń}%
		{Tak}{Wyświetlanie informacji o efektywnych ustawieniach dla obiektów jednostki organizacyjnej.}%
		{Nie}{Eksportowanie ustawień, aby umożliwić ich ponowne wdrożenie w dowolnym momencie.}
		
		\item \question{Które narzędzia służą do tworzenia i zarządzania GPO?}%
		{Nie}{Narzędzia gpmod}%
		{Tak}{Konsola Group Policy Management}%
		{Tak}{Group Policy Object Editor z Active Directory Users and Computers}%
		{Nie}{Narzędzie gpadd}
		
		\item \question{Diagnozujesz problem z ustawieniami Zasad Grupy w dużej domenie. chcesz się dowiedzieć, jakie ustawienia są stosowane dla danego komputera.}%
		{Nie}{Przeglądasz ustawienia zasad grupy w konsoli Active Directory Users and Computers}%
		{Tak}{Korzystasz z narzędzia gpresult}%
		{Tak}{Korzystasz z  "Group Policy Result" w konsoli Group Policy Management}%
		{Nie}{Nie jest możliwe uzyskanie takich informacji}
		
		\item \question{Jakimi narzędziami możesz konfigurować GPO?}%
		{Tak}{Group Policy Management Console}%
		{Tak}{Active Directory Sites and Services}%
		{Nie}{Remote Desktop Assistance}%
		{Tak}{Edytor obiektów zasad grup}
		
		\item \question{Jaki będzie wynik polecenia gpresult /user user1 w Windows Serwer 2008?}%
		{Nie}{Wyświetlenie rezultatu wykonania polecenia gpupdate dla użytkownika user1}%
		{Nie}{Nie ma takiego polecenia}%
		{Nie}{Wyświetlenie ustawień zmiennych globalnych użytkownika user1}%
		{Tak}{Wyświetlenie zasad grup dla użytkownika user1}
		
		\item \question{Wskaż prawdziwe zdania dotyczące Group Policy Objects}%
		{Nie}{Dany obiekt GPO można przypisać tylko do jednej jednostki organizacyjnej}%
		{Tak}{Zasady zawarte w GPO przypisanym do domeny są domyślnie dziedziczone przez jednostki organizacyjne tej domeny}%
		{Tak}{Do jednej jednostki organizacyjnej można przypisać wiele różnych obiektów GPO}%
		{Tak}{Zasady obiektów GPO przypisanych jednostkom podrzędnym domyślnie nadpisują zasady odziedziczone z GPO jednostek nadrzędnych}
		
		\item \question{Co się stanie, jeśli pewne ustawienia obiektu zasad grup ustawimy na poziomie domeny na pewną wartość, a na poziomie komputera ustawimy na przeciwną?}%
		{Tak}{Zastosowane zostanie ustawienie na poziomie \textbf{komputera}}%
		{Nie}{Wystąpu konflikt i zostanie zgłoszony błąd}%
		{Nie}{Poczas uruchamiania tego komputera losowo wybierana będzie wartość, która będzie się do niego stosować}%
		{Nie}{Zastosowane zostanie ustawienie ustalone na poziomie \textbf{domeny}}
		
	\end{enumerate}