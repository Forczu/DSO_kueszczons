%\item \question{}%
%{Tak}{}%
%{Nie}{}%
%{}{}%
%{}{}

% !TeX spellcheck = pl_PL
\newpage
\section{Windows Instalacja zdalna}
	\begin{enumerate}
		\item \question{Windows Deployment Services (WDS):}%
		{Nie}{Pozwala na przygotowanie obrazów dysków do zautomatyzowania lokalnej instalacji systemu Windows.}%
		{Tak}{Pozwala na instalację systemu Window.}%
		{Nie}{Możliwe jest instalowanie przez sieć wyłącznie systemów serwerowych np. Windows Server 2008.}%
		{Nie}{Możliwa jest zdalna instalacja (przez sieć) systemu Linux wykorzystując system Windows Server.}
		\item \question{Windows Deployment Services wykorzystuje obrazy z rozszerzeniem:}%
		{Nie}{BIN}%
		{Nie}{MDF}%
		{Tak}{WIM}%
		{Nie}{ISO}
		\item \question{Format obrazów instalacyjnych wykorzystywany przez Windows Deployment Services to:}%
		{Nie}{VHD}%
		{Nie}{ISO}%
		{Nie}{IMG}%
		{Tak}{WIM}
		\item \question{Windows Deployment Services to:}%
		{Nie}{Tworzenie instalatorów dla programów na platformę .NET}%
		{Nie}{Instalację systemu Windows poprzez nośnik USB.}%
		{Tak}{Usługa pozwalająca na instalację systemu Windows przez sieć.}%
		{Nie}{Instalację i konfigurację aplikacji internetowej na serwerze IS.}
		\item \question{Windows Deployment Services (WDS) to technologia serwerowa, która pozwala na:}%
		{Nie}{Zdalne logowanie do systemu.}%
		{Tak}{Sieciową instalację systemu operacyjnego.}%
		{Tak}{Instalację systemu operacyjnego bez płyty instalacyjnej typu CD lub DVD.}%
		{Nie}{Lokalne monitorowanie systemu operacyjnego chroniąc przed złośliwym oprogramowaniem.}
		\item \question{Aby możliwa była zdalna istalacja, to maszyna kliencka może uruchamiać się z:}%
		{Nie}{dysku twardego}%
		{Tak}{karty sieciowej}%
		{Nie}{napędu CD / DVD}%
		{Nie}{nie ma to znaczenia}
		
		\newpage
		\item \question{Jakie elementy są wymagane do poprawnej pracy WDS?}%
		{Nie}{Windows Server w wersji 2008 lub wyższej.}%
		{Tak}{Usługa Windows Deployment Services zainstalowana na serwerze udostępniającym obrazy do instalacji.}%
		{Nie}{Sprzęt sieciowy obsługujący protokół WDS (ro\textsl{uter, switch, karta siecio}wa)}%
		{Tak}{Kontroler domeny, serwer DNS, serwer DHCP}
		\item \question{Które z poniższych zdań na temat wymagań instalacji zdalnej jest prawdziwe?}%
		{Tak}{Serwer WDS musi być członkiem domeny Active Directory.}%
		{Tak}{W sieci musi znajdować się serwer DNS.}%
		{Tak}{W sieci musi znajdować się serwer DHCP.}%
		{Nie}{Serwery DHCP i DNS muszą być niezależne od serwera WDS.}
		\item \question{Wykorzystując zdalną instalację systemu Windows:}%
		{Tak}{Jeden serwer umożliwia instalację wielu wersji systemu (użytkownik może sam wybrać).}%
		{Nie}{Jeden serwer pozwala na instalację tylko jednej wersji systemu (np. Ultimate)}%
		{Tak}{Pliki z obrazem systemu muszą być dostępne na serwerze.}%
		{Nie}{Do komputera na którym instalowany jest system trzeba włożyć płytę z obrazem systemu (ale konfiguracja instalowanego systemu jest pobierana przez sieć)}
		\item \question{Jakie warunki muszą być spełnione by można było pomyślnie zainstalować usługę WDS?}%
		{Nie}{Sieć musi być połączona z Internetem.}%
		{Tak}{Komputer musi być członkiem domeny Active Directory.}%
		{Tak}{W sieci musi znajdować się serwer DNS.}%
		{Tak}{W sieci musi znajdować się serwer DHCP.}
		\item \question{Aby możliwe było wykorzystanie Windows Deployment Services konieczny jest:}%
		{Tak}{Serwer DHCP wskazujący lokalizację pliku uruchomieniowego.}%
		{Nie}{Serwer FTP z którego będą pobierane pliki instalacyjne.}%
		{Nie}{Obraz instalacyjny z systemem Windows 7 w edycji co najmniej Professional.}%
		{Tak}{Obraz środowiska Windows PE.}
		\item \question{Mechanizm WDS umożliwia:}%
		{Nie}{Zdalną instalację systemów z obrazów płyt .iso}%
		{Tak}{Zdalną instalację systemów Windows.}%
		{Nie}{Zdalne zarządzanie zainstalowanymi systemami Windows.}%
		{Tak}{Zdalną instalację systemów z obrazów płyt .wim}
		
		\newpage
		\item \question{Wskaż poprawne zdania dotyczące WDS:}%
		{Tak}{Proces instalacji systemu na komputerze klienckim rozpoczyna się od przesłania po sieci obrazu bardzo uproszczonego systemu operacyjnego służącego do uruchomienia głównego instalatora.}%
		{Nie}{Serwer w momencie instalowania usługi WDS automatycznie instaluje obrazy płyt używane do instalacji systemu po sieci.}%
		{Tak}{Aby zainstalować na komputerze klienckim system Windows, używając mechanizmu WDS, należy ustawić w BIOSie bootowanie rozpoczynające się od karty sieciowej.}%
		{Nie}{Używając WDS możemy instalować po sieci każdy system z rodziny Microsoft Windows i Linux.}
		\item \question{Wskaż poprawne zdania dotyczące WDS:}%
		{Nie}{Serwer w momencie instalowania usługi WDS automatycznie instaluje obrazy płyt używane to instalacji systemu po sieci}%
		{Tak}{Proces instalacji systemu na komputerze klienckim rozpoczyna się od przesłania po sieci obrazu bardzo uproszczonego systemu operacyjnego służącego do uruchomienia głównego instalatora}%
		{Nie}{Używając WDS możemy instalować po sieci każdy system z rodziny Microsoft Windows i Linux}%
		{Tak}{Aby zainstalować na komputerze klienckim system windows używając mechanizmu WDS należy ustawić w biosie boot'owanie rozpoczynające się od karty sieciowej}
		%\item \question{Serwer DHCP w systemie windows 2008 server:}%
		%{Nie}{jest zainstalowany w systemie po instalacji}%
		%{Tak}{jest dostępny w systemie jako rola}%
		%{Tak}{umożliwia tworzenie zakresu adresów, z których mają być przydzielane adresy klientom}%
		%{Nie}{jest w całości zarządzany tylko przy pomocy konsolowego narzędzia}
		%\item \question{Serwer DNS umożliwia:}%
		%{Nie}{dynamiczne przydzielanie adresów IP komputerom w sieci lokalnej}%
		%{Nie}{tłumaczenie adresów domenowych na adresy MAC}%
		%{Nie}{tłumaczenie adresów IP na adresy MAC}%
		%{Tak}{tłumaczenie adresów domenowych na adresy IP}
		%\item \question{Serwer DHCP umożliwia:}%
		%{Tak}{Automatyczną aktualizację adresu serwera DNS}%
		%{Nie}{Zamianę tekstowego adresu URL na adres IP}%
		%{Tak}{Dynamiczne przyznawanie adresu IP hostom}%
		%{Nie}{Dynamiczne nadawanie adresu MAC hostom}
		%\item \question{Za pomocą polecenia nslookup w systemie Windows możemy uzyskad informacje o:}%
		%{Tak}{Adresie IP serwera}%
		%{Tak}{Aliasach serwera}%
		%{Nie}{Lokalizacji geograficznej serwera}%
		%{Nie}{Czasu odpowiedzi serwera}

				
		%\item \question{}%
		%{Tak}{}%
		%{Nie}{}%
		%{}{}%
		%{}{}
		
	\end{enumerate}