% !TeX spellcheck = pl_PL
\documentclass[a4paper,twoside]{article}
\usepackage{polski}
\usepackage[utf8]{inputenc}
\usepackage{graphicx}
\usepackage{amsmath}

\usepackage[unicode, bookmarks=true]{hyperref} %do zakładek
\usepackage{tabto} % do tabulacji
\NumTabs{6} % globalne ustawienie wielkosci tabulacji
\usepackage{array}
\usepackage{multirow}
\usepackage{array}
\usepackage{dcolumn}
\usepackage{bigstrut}
\usepackage{color}
\usepackage[usenames,dvipsnames]{xcolor}
\usepackage{svg}
\usepackage{xfrac}
\usepackage{floatrow}
% Table float box with bottom caption, box width adjusted to content
\newfloatcommand{capbtabbox}{table}[][\FBwidth]
\usepackage{blindtext}
\usepackage{enumerate}
\usepackage{wrapfig}


\setlength{\textheight}{24cm}
\setlength{\textwidth}{15.92cm}
\setlength{\footskip}{10mm}
\setlength{\oddsidemargin}{0mm}
\setlength{\evensidemargin}{0mm}
\setlength{\topmargin}{0mm}
\setlength{\headsep}{5mm}

\newcolumntype{M}[1]{>{\centering\arraybackslash}m{#1}}
\newcolumntype{N}{@{}m{0pt}@{}}

% === Reset inkrementacji sekcji przy nowym parcie === %
\usepackage{titlesec}

\makeatletter
\@addtoreset{section}{part}
\makeatother
\titleformat{\part}[display]
{\normalfont\LARGE\bfseries\centering}{}{0pt}{}

% === DEFINICJA ZIELONEGO ==================== %
\definecolor{Gurin}{rgb}{0, 0.35, 0}

% === MAKRODEFINICJA POPRAWNEJ I ZŁEJ ODPOWIEDZI ==================== %
\newcommand{\Tak}[1] {
	\color{Gurin}{#1}
}
\newcommand{\Nie}[1] {
	\color{Red}{#1}
}

% === MAKRODEFINICJA PYTANIA I ODPOWIEDZI =========================== %
% ** Pierwszy argument to treść pytania
% ** Kolejne argumenty to:
% * parzyste - Tak lub Nie
% * nieparzyste - Treści odpowiedzi
\newcommand{\question}[9] {
	\textbf{#1}
	\begin{enumerate}[a.]
		\ifnum\pdfstrcmp{#2}{Tak}=0
			\Tak{\item #3}
		\else
			\Nie{\item #3}
		\fi
		\color{black}
		\ifnum\pdfstrcmp{#4}{Tak}=0
			\Tak{\item #5}
		\else
			\Nie{\item #5}
		\fi
		\color{black}
		\ifnum\pdfstrcmp{#6}{Tak}=0
			\Tak{\item #7}
		\else
			\Nie{\item #7}
		\fi
		\color{black}
		\ifnum\pdfstrcmp{#8}{Tak}=0
			\Tak{\item #9}
		\else
			\Nie{\item #9}
		\fi
	\end{enumerate}
}
\begin{document}
\bibliographystyle{plain}

% ************************************************************
% *** WZÓR PYTANIA DO SKOPIOWANIA

%\item \question{}%
%{Tak}{}%
%{Nie}{}%
%{}{}%
%{}{}

% ************************************************************

\begin{titlepage}
\title{\huge Dedykowane Systemy Operacyjne - zbiór pytań}
\author{\large zebrał SonMati}
\maketitle
\end{titlepage}

% =============== PYTANIA ==================================== %
% ======== WINDOWS =========================================== %
\part{Windows}
	
	% --- AD - Active Directory ------------ %
	\section{Usługi katalogowe (Active Directory)}
	\begin{enumerate}
		\item \question{Wykonanie kwerendy do wyszukiwania wyłączonych kont użytkowników jest możliwe za pomocą:}%
		{Nie}{polecenia ds-query}%
		{Tak}{polecenia dsquery}%
		{Tak}{konsoli Active Directory Users and Computers}%
		{Nie}{polecenia ds-get}
		\item \question{Role FSMO można:}%
		{Tak}{Przejmować}%
		{Nie}{Filtrować}%
		{Nie}{Nadpisywać}%
		{Tak}{Transferować}
		\item \question{W jednostkach organizacyjnych (Organization Unit) można utworzyć:}%
		{Tak}{Użytkownika}%
		{Tak}{Grupę}%
		{Tak}{Komputer}%
		{Tak}{Drukarkę}
		\item \question{Za pomocą jakiego polecenia można dodać obiekt określonego typu (korzystając z wiersza poleceń) w Active Directory}%
		{Nie}{dscreate user}%
		{Tak}{dsadd user}%
		{Tak}{dsadd computer}%
		{Nie}{dscreate computer}
		\item \question{Wybierz prawidłowe odpowiedzi dotyczące struktury Active Directory:}%
		{Tak}{Jeśli domeny wchodzące w skład lasu mają nieciągłe nazwy DNS, tworzą kilka odrębnych drzew w obrębie lasu}%
		{Tak}{Drzewo posiada zawsze przynajmniej jedną domenę - domenę najwyższego poziomu (ang. root) - korzeń drzewa}%
		{Nie}{Drzewo domen - domeny potomne mogą, ale nie muszą zawierać nazwy bezpośredniej domeny nadrzędnej}%
		{Nie}{Las jest zestawem przynajmniej dwóch lub więcej drzew, które formują zwartą, ciągłą przestrzeń nazw.}
		\newpage
		\item \question{Do czego służy polecenie dsget?}%
		{Nie}{Wyświetla różne właściwości grupy, włącznie z członkami grupy w katalogu}%
		{Nie}{Wyświetla właściwości komputera w katalogu.}%
		{Tak}{Umożliwia dodawanie użytkowników, grup, komputerów, kontaktów i jednostek organizacyjnych do usługi Active Directory.}%
		{Nie}{Umożliwia tworzenie, modyfikowanie i usuwanie obiektów katalogu.}
		\item \question{Jakie obiekty można dodawać za pomocą polecenia dsadd?}%
		{Tak}{Grupy}%
		{Tak}{Użytkowników}%
		{Tak}{Jednostki organizacyjne}%
		{Nie}{Pliki}
		\item \question{Wskaż zdania prawdziwe dotyczące usługi Active Directory:}%
		{Tak}{Struktura Active Directory ma strukturę drzewiastą.}%
		{Tak}{Liśćmi drzewa mogą być użytkownicy, grupy i }%
		{Nie}{Nie ma możliwości nadania określonemu użytkownikowi praw do zarządzania użytkownikami w jednostce organizacyjnej bez nadawania (...)}%
		{Tak}{Możliwe jest kopiowanie użytkowników.}
		\item \question{Program dsget.exe:}%
		{Tak}{wymaga praw administratora do działania}%
		{Nie}{może być uruchomione w graficznym interfejsie użytkownika za pomocą odpowiedniej opcji linii poleceń.}%
		{Tak}{ma tekstowy interfejs użytkownika.}%
		{Nie}{umożliwia tworzenie jednostek organizacyjnych.}
		\item \question{Jakie możliwości daje użycie polecenia dsadd (jako polecenie dla Active Directory)? }%
		{Nie}{Nie może dodawać obiektu typu Computer do katalogu}%
		{Tak}{Nie może dodać obiektu typu Doman Service do katalogu.}%
		{Tak}{Może dodawać obiekt typu Computer do katalogu.}%
		{Nie}{Może dodać obiekt typu Domain Service do katalogu.}
		\item \question{Różnica pomiędzy zaufaniem do domen forest i external polega na:}%
		{Nie}{Zaufanie typu forest pozwala na korzystanie z zasobów tylko w obrębie danego drzewa, zaś external we wszystkich drzewach.}%
		{Tak}{Zaufanie typu external pozwala większej liczby domen na korzystanie ze swoich zasobów niż typ forest.}%
		{Tak}{Zaufanie typu forest pozwala na korzystanie z zasobów tylko w obrębie drzew połączonych tego typu zaufaniem, zaś external nie musi (...)}%
		{Nie}{Zaufanie typu external jest dwukierunkowe, typu forest tylko jednokierunkowe.}
		\newpage
		\item \question{Polecenie dsquery:}%
		{Nie}{Zgodnie z określonymi kryteriami wykonuje kwerendę dotyczącą drzewa usługi DNS.}%
		{Nie}{Jest równoważne poleceniu dsget.}%
		{Nie}{Wykonuje kwerendę na dowolnej bazie danych.}%
		{Tak}{Zgodnie z określonymi kryteriami wykonuje kwerendę dotyczącą usługi Active Directory.}
		\item \question{Dodać grupę można za pomocą:}%
		{Tak}{Konsoli "Active directory users and computers".}%
		{Nie}{Polecenia Addgroup.}%
		{Tak}{Polecenia dsadd group.}%
		{Nie}{Polecenia adadd group.}
		\item \question{Jakie występują typy zaufania w Active Directory:}%
		{Tak}{Lasu (forest)}%
		{Nie}{Wewnętrzne (internal)}%
		{Tak}{Zewnętrzne (external)}%
		{Nie}{Płatka śniegu (snowflake)}
		\item \question{Struktura Active Directory:}%
		{Tak}{Podstawową jednostką jest tzw. liść, który położony jest w kontenerze w Active Directory nazywanym jednostką centralną.}%
		{Tak}{Liście i kontenery zorganizowane są w domeny.}%
		{Nie}{Domeny zorganizowane w drzewo reprezentowane są w rożnych przestrzeniach adresowych DNS.}%
		{Nie}{Domena może istnieć samodzielnie, nie musi istnieć w jakimś drzewie i jakimś lesie}
		\item \question{Polecenie dsadd może posłużyć do:}%
		{Nie}{Modyfikowania obiektów wewnątrz AD.}%
		{Tak}{Dodawania grup do AD.}%
		{Tak}{Dodawania użytkowników do AD.}%
		{Nie}{Wyszukiwania informacji o obiektach AD.}
		\item \question{Za pomoca konsoli Active Directory Users and Computers wykonano polecenie "dsadd Ala -pwd makota". Wskaż poprawne odpowiedzi.}%
		{Nie}{Jeżeli użytkownik "Ala" nie istnieje w systemie wykonanie polecenia nie powiedzie się.}%
		{Tak}{Powodzenie operacji jest zależne od poziomu uprawnień wykonującego go użytkownika.}%
		{Tak}{W przypadku powodzenia operacji zostanie utworzony nowy użytkownik o nazwie "Ala" i haśle "makota".}%
		{Nie}{W przypadku powodzenia operacji hasło istniejącego użytkownika "Ala" zostanie zmienione na "makota".}
		\item \question{Za pomoca polecenia dsadd można:}%
		{Tak}{Utworzyć grupę zabezpieczeń.}%
		{Tak}{Utworzyć jednostkę organizacyjną.}%
		{Tak}{Utworzyć konto użytkownika.}%
		{Tak}{Utworzyć grupę dystrybucyjną.}
		\item \question{Czym jest Active Directory?}%
		{Nie}{Narzędzie certyfikujące w Windows Server 2003.}%
		{Tak}{Usługą w systemie Windows, która udostępnia środki pozwalające zarządzać tożsamościami i relacjami.}%
		{Tak}{Jest to zbiór funkcji, która upraszczają zarządzanie użytkownikami i komputerem.}%
		{Nie}{Nakładka na system Linux pozwalająca zarządzać komputerem.}
		\item \question{Poznane w czasach laboratorium narzędzia pozwalające na zarządzanie Active Directory to:}%
		{Nie}{netsh}%
		{Tak}{dsget}%
		{Tak}{dsadd}%
		{Tak}{Przystawka Active Directory Users and Computers}
		\item \question{Ile kontrolerów może działać dla domeny w usłudze Active Directory?}%
		{Tak}{więcej niż 2}%
		{Tak}{2}%
		{Tak}{1}%
		{Nie}{domena może nie posiadać komputera}
		\item \question{Które z typów relacji zaufania są dwukierunkowe?}%
		{Tak}{Lasu (forest)}%
		{Tak}{Zewnętrzne (external)}%
		{Tak}{Obszaru (realm)}%
		{Tak}{Skrótu (shortcut)}
		\item \question{Aby uruchomić na serwerze usługę Active Directory należy:}%
		{Nie}{Utworzyć użytkownika do zarządzania usługą.}%
		{Tak}{Zainstalować serwer DNS.}%
		{Tak}{Dodać rolę Active Directory Domain Services.}%
		{Nie}{Zainstalować serwer DHCP.}
		\item \question{Co to jest jednostka organizacyjna (Organization Unit, OU)?}%
		{Nie}{Pojedynczy komputer wchodzący w skład domeny Active Directory.}%
		{Nie}{Serwer pracujący pod kontrolą systemu Windows 2008 Server.}%
		{Tak}{Kontener wykorzystywany do grupowania obiektów wewnątrz domeny w logiczne grupy, na których wykonywane są zadania administracyjne (...)}%
		{Nie}{Serwer służący do zarządzania protokołem ldap}
		\item \question{Jakie obiekty mogą być modyfikowane za pomocą usługi Active Directory Sites and Services?}%
		{Tak}{Łącza lokacji i lokacje.}%
		{Nie}{Tylko serwery i lokacje.}%
		{Nie}{Serwery, lokacje, replikacje.}%
		{Tak}{Serwery, podsieci, ustawienia usługi katalogowej (NTDS)}
		\item \question{Wskaż zdania prawdziwe:}%
		{Nie}{Usługa Active Directory dostępna jest w systemach Windows od wersji Vista.}%
		{Tak}{Za pomocą polecenia dsadd user możliwe jest dodanie użytkownika do Active Directory.}%
		{Tak}{Za pomocą polecenia dsadd group możliwe jest dodanie grupy do Active Directory.}%
		{Tak}{Szablon konta użytkownika to konto zawierające wzorcowy zestaw najczęściej wykorzystywanych właściwości, odpowiednich dla danej grupy użytkowników.}
		\item \question{Które polecenie umożliwia usuwanie obiektów określonego typu?}%
		{Nie}{dsmod}%
		{Nie}{csvde}%
		{Nie}{dsadd}%
		{Tak}{dsrm}
		\item \question{W Windows Server 2008 - Active Directory administrator edytując właściwości obiektu klasy Użytkownik, w zakładce Konto ma możliwość:}%
		{Nie}{Wyznaczenia daty i godziny, w której konto Użytkownika zostanie usunięte z systemu.}%
		{Tak}{Zablokowania Użytkownikowi możliwości samodzielnej zmiany hasła.}%
		{Tak}{Ustalenia godzin w których Użytkownik będzie mógł lub nie będzie mógł zalogować się do systemu.}%
		{Tak}{Określenia do jakich komputerów znajdujących się w domenie Użytkownik ma prawo do zalogowania się.}
		\item \question{Za pomocą polecenia dsadd w systemie Windows można:}%
		{Nie}{Zmodyfikować konto użytkownika}%
		{Tak}{Utworzyć jednostkę organizacyjną}%
		{Tak}{Utworzyć konto użytkownika}%
		{Nie}{Przenieść jednostkę organizacyjną z jednej domeny do drugiej}
		\item \question{Wskaż najważniejsze funkcje Active directory:}%
		{Tak}{Obsługa LDAP}%
		{Tak}{Obsługa DNS}%
		{Tak}{Obsługa zabezpieczeń}%
		{Tak}{Zgodność z TCP/IP}
		\item \question{W jednostce organizacyjnej można umieszczać:}%
		{Tak}{Komputery}%
		{Tak}{Użytkowników}%
		{Tak}{Inne jednostki organizacyjne}%
		{Nie}{Obiekty z innych domen niż domena jednostki organizacyjnej}
		\item \question{Wskaż poprawne stwierdzenia na temat uprawnień w Active Directory:}%
		{Nie}{Uprawnienia nie mogą być dziedziczone}%
		{Tak}{Istnieje możliwość nadania uprawnień administratora na danym komputerze bez posiadania uprawnień administratora w domenie}%
		{Nie}{Aby skopiować użytkownika, jego konto musi być uprzednio wyłączone}%
		{Tak}{Użytkownik może należeć do wielu grup}
		\item \question{Jakie protokoły są wykorzystywane przez Active Directory?}%
		{Nie}{SSH}%
		{Tak}{LDAP}%
		{Tak}{Kerberos}%
		{Tak}{DNS}
		\item \question{Do czego służy polecenie dsadd?}%
		{Nie}{Dodaje możliwość zdalnego dostępu do elementu serwisu Active Directory}%
		{Nie}{Dodaje atrybut do elementu serwisu Active Directory}%
		{Nie}{Dodaje połączenie do zarządzania elementem serwisu Active Directory}%
		{Tak}{Dodanie określony typ elementu do serwisu Active Directory}
		\item \question{Wskaż prawdziwe zdania dotyczące jednostek organizacyjnych (Organizational Units):}%
		{Tak}{Mogą tworzyć strukturę hierarchiczną}%
		{Nie}{Grupują wyłącznie użytkowników}%
		{Tak}{Mogą być wykorzystane do delegowania uprawnień administracyjnych}%
		{Nie}{Nie mogą być w sobie zagnieżdżane}
		\item \question{Za pomocą konsoli Active Directory Users and Computers wykonano polecenie: „dsmod group Alfa -addmbr Beta”. Wskaż wszystkie poprawne odpowiedzi.:}%
		{Nie}{W przypadku powodzenia operacji do grupy Beta zostanie dodany obiekt Alfa.}%
		{Tak}{W przypadku powodzenia operacji do grupy Alfa zostanie dodany obiekt Beta.}%
		{Tak}{Powodzenie operacji jest zależne od poziomu uprawnień wykonującego go użytkownika.}%
		{Tak}{Obiekt o nazwie Beta może reprezentować użytkownika.}
	\end{enumerate}
	
	% --- GPO - Obieky Zasad Grup ---------- %
	\newpage
\section{Obiekty Zasad Grup (GPO)}
	\begin{enumerate}
		\item \question{Na jakich poziomach w Active Directory mogą być przypisywane obiekty GPO?}%
		{Tak}{Lokalnie}% 
		{Nie}{Na poziomie lokacji}%
		{Tak}{Na poziomie domeny}%
		{Tak}{Na poziomie jednostki organizacyjnej}%
		
		\item \question{Aby wyświetlić wynikowy zestaw zasad dla użytkownia Sysop należy użyć polecenia:}%
		{Nie}{gpresult /gpo Sysop}%
		{Nie}{gpresult /?}%
		{Tak}{gpresult /user Sysop}%
		{Nie}{gpresult /u Sysop}
		
		\item \question{Wskaż prawdziwe zdania dotyczące GPO}%
		{Nie}{Akronin GPO rozwija się jako Group Policy Operation}%
		{Tak}{Za pomocą GPO Standard Desktop można zabronić dostępu do Panelu Sterowania}%
		{Nie}{Dane jednego GPO mogą być przypisane tylko jednej jednostce organizacyjnej}%
		{Nie}{Nie da się wyłączyć stosowania zasad GPO danej jednostki organizacyjnej bez usuwania GPO lu łącza obiektu}
		
		\item \question{Gdzie w rejestrze systemowym można znaleźć wpisy wynikające z GPO?}%
		{Tak}{HKEY LOCAL MACHINE (HKLM)}%
		{Nie}{HKEY CLASSES ROOT (HKCR)}%
		{Tak}{HKEY CURRENT USER (HKCU)}%
		{Nie}{HKEY USERS (HKU)}
		
		\item \question{W jaki sposób można modyfikować domyślne przetwarzanie obiektów zasad grupy?}%
		{Tak}{Blokując dziedziczenie zasad grupy}%
		{Nie}{Definiując warunkowe wprowadzanie ustawień.}%
		{Tak}{Wyłączając przetwarzanie konkretnego łącza GPO}%
		{Tak}{Wyłączając nadpisywanie ustawień wprowadzanych przez konkretne łącze GPO.}
		
		\item \question{Group Policy Management Console umożliwia:}%
		{Nie}{Wszystkie funkcje konsoli Power Shell, oraz dodatkowo funkcje zarządzania obiektami GPO}%
		{Tak}{Stworzenie kopii zapasowej obiektów GPO}%
		{Tak}{Łatwiejsze zarządzanie obiektami GPO, dzięki graficznemu interfejsowi użytkownika}%
		{Nie}{Tworzenie logów każdej operacji użytkownika w wybranej przez administratora grupie}
		
		\item \question{System Windows w ramach zarządzania GPO umożliwia:}%
		{Tak}{Filtrowanie ustawień GPO - wyłączenie stosowania określonych zasad GPO}%
		{Tak}{Wymuszanie stosowania zasad GPO}%
		{Tak}{Przeglądanie wdrażania elementów GPO dla danej jednostki organizacyjnej}%
		{Tak}{Blokowanie dziedziczenia ustawień obiektów GPO}
		
		\item \question{Które narzędzia służą do tworzenia i zarządzania GPO?}%
		{Tak}{Konsola Group Policy Management}%
		{Nie}{narzędzie gpadd}%
		{Tak}{Group Policy Object Editor z Active Directory Users and Computers}%
		{Nie}{narzędzie gpomod}
		
		
	\end{enumerate}
		
	% --- Instalacja Zdalna ---------------- %
	%\item \question{}%
%{Tak}{}%
%{Nie}{}%
%{}{}%
%{}{}

% !TeX spellcheck = pl_PL
\newpage
\section{Windows Instalacja zdalna}
	\begin{enumerate}
		\item \question{Windows Deployment Services (WDS):}%
		{Nie}{Pozwala na przygotowanie obrazów dysków do zautomatyzowania lokalnej instalacji systemu Windows.}%
		{Tak}{Pozwala na instalację systemu Windows przez sieć.}%
		{Nie}{Możliwe jest instalowanie przez sieć wyłącznie systemów serwerowych np. Windows Server 2008.}%
		{Nie}{Możliwa jest zdalna instalacja (przez sieć) systemu Linux wykorzystując system Windows Server.}
		\item \question{Windows Deployment Services wykorzystuje obrazy z rozszerzeniem:}%
		{Nie}{BIN}%
		{Nie}{MDF}%
		{Tak}{WIM}%
		{Nie}{ISO}
		\item \question{Format obrazów instalacyjnych wykorzystywany przez Windows Deployment Services to:}%
		{Nie}{VHD}%
		{Nie}{ISO}%
		{Nie}{IMG}%
		{Tak}{WIM}
		\item \question{Windows Deployment Services to:}%
		{Nie}{Tworzenie instalatorów dla programów na platformę .NET}%
		{Nie}{Instalację systemu Windows poprzez nośnik USB.}%
		{Tak}{Usługa pozwalająca na instalację systemu Windows przez sieć.}%
		{Nie}{Instalację i konfigurację aplikacji internetowej na serwerze IS.}
		\item \question{Windows Deployment Services (WDS) to technologia serwerowa, która pozwala na:}%
		{Nie}{Zdalne logowanie do systemu.}%
		{Tak}{Sieciową instalację systemu operacyjnego.}%
		{Tak}{Instalację systemu operacyjnego bez płyty instalacyjnej typu CD lub DVD.}%
		{Nie}{Lokalne monitorowanie systemu operacyjnego chroniąc przed złośliwym oprogramowaniem.}
		\item \question{Aby możliwa była zdalna istalacja, to maszyna kliencka może uruchamiać się z:}%
		{Nie}{dysku twardego}%
		{Tak}{karty sieciowej}%
		{Nie}{napędu CD / DVD}%
		{Nie}{nie ma to znaczenia}
		\item \question{Jakie elementy są wymagane do poprawnej pracy WDS?}%
		{Nie}{Windows Server w wersji 2008 lub wyższej.}%
		{Tak}{Usługa Windows Deployment Services zainstalowana na serwerze udostępniającym obrazy do instalacji.}%
		{Nie}{Sprzęt sieciowy obsługujący protokół WDS (ro\textsl{uter, switch, karta siecio}wa)}%
		{Tak}{Kontroler domeny, serwer DNS, serwer DHCP}
		\item \question{Które z poniższych zdań na temat wymagań instalacji zdalnej jest prawdziwe?}%
		{Tak}{Serwer WDS musi być członkiem domeny Active Directory.}%
		{Tak}{W sieci musi znajdować się serwer DNS.}%
		{Tak}{W sieci musi znajdować się serwer DHCP.}%
		{Nie}{Serwery DHCP i DNS muszą być niezależne od serwera WDS.}
		\item \question{Wykorzystując zdalną instalację systemu Windows:}%
		{Tak}{Jeden serwer umożliwia instalację wielu wersji systemu (użytkownik może sam wybrać).}%
		{Nie}{Jeden serwer pozwala na instalację tylko jednej wersji systemu (np. Ultimate)}%
		{Tak}{Pliki z obrazem systemu muszą być dostępne na serwerze.}%
		{Nie}{Do komputera na którym instalowany jest system trzeba włożyć płytę z obrazem systemu (ale konfiguracja instalowanego systemu jest (...))}
		\item \question{Jakie warunki muszą być spełnione by można było pomyślnie zainstalować usługę WDS?}%
		{Nie}{Sieć musi być połączona z Internetem.}%
		{Tak}{Komputer musi być członkiem domeny Active Directory.}%
		{Tak}{W sieci musi znajdować się serwer DNS.}%
		{Tak}{W sieci musi znajdować się serwer DHCP.}
		\item \question{Aby możliwe było wykorzystanie Windows Deployment Services konieczny jest:}%
		{Tak}{Serwer DHCP wskazujący lokalizację pliku uruchomieniowego.}%
		{Nie}{Serwer FTP z którego będą pobierane pliki instalacyjne.}%
		{Nie}{Obraz instalacyjny z systemem Windows 7 w edycji co najmniej Professional.}%
		{Tak}{Obraz środowiska Windows PE.}
		\item \question{Mechanizm WDS umożliwia:}%
		{Nie}{Zdalną instalację systemów z obrazów płyt .iso}%
		{Tak}{Zdalną instalację systemów Windows.}%
		{Nie}{Zdalne zarządzanie zainstalowanymi systemami Windows.}%
		{Tak}{Zdalną instalację systemów z obrazów płyt .wim}
		\item \question{Wskaż poprawne zdania dotyczące WDS:}%
		{Tak}{Proces instalacji systemu na komputerze klienckim rozpoczyna się od przesłania po sieci obrazu bardzo uproszczonego systemu operacyjnego (...) głównego instalatora.}%
		{Nie}{Serwer w momencie instalowania usługi WDS automatycznie instaluje obrazy płyt używane do instalacji systemu po sieci.}%
		{Tak}{Aby zainstalować na komputerze klienckim system Windows, używając mechanizmu WDS, należy ustawić w BIOSie bootowanie rozpoczynające (...) sieciowej.}%
		{Nie}{Używając WDS możemy instalować po sieci każdy system z rodziny Microsoft Windows i Linux.}
		\item \question{Wskaż poprawne zdania dotyczące WDS:}%
		{Nie}{Serwer w momencie instalowania usługi WDS automatycznie instaluje obrazy płyt używane to instalacji systemu po sieci}%
		{Tak}{Proces instalacji systemu na komputerze klienckim rozpoczyna się od przesłania po sieci obrazu bardzo uproszczonego systemu operacyjnego służącego do uruchomienia głównego instalatora}%
		{Nie}{Używając WDS możemy instalować po sieci każdy system z rodziny Microsoft Windows i Linux}%
		{Tak}{Aby zainstalować na komputerze klienckim system windows używając mechanizmu WDS należy ustawić w biosie boot'owanie rozpoczynające się od karty sieciowej}
		%\item \question{Serwer DHCP w systemie windows 2008 server:}%
		%{Nie}{jest zainstalowany w systemie po instalacji}%
		%{Tak}{jest dostępny w systemie jako rola}%
		%{Tak}{umożliwia tworzenie zakresu adresów, z których mają być przydzielane adresy klientom}%
		%{Nie}{jest w całości zarządzany tylko przy pomocy konsolowego narzędzia}
		%\item \question{Serwer DNS umożliwia:}%
		%{Nie}{dynamiczne przydzielanie adresów IP komputerom w sieci lokalnej}%
		%{Nie}{tłumaczenie adresów domenowych na adresy MAC}%
		%{Nie}{tłumaczenie adresów IP na adresy MAC}%
		%{Tak}{tłumaczenie adresów domenowych na adresy IP}
		%\item \question{Serwer DHCP umożliwia:}%
		%{Tak}{Automatyczną aktualizację adresu serwera DNS}%
		%{Nie}{Zamianę tekstowego adresu URL na adres IP}%
		%{Tak}{Dynamiczne przyznawanie adresu IP hostom}%
		%{Nie}{Dynamiczne nadawanie adresu MAC hostom}
		%\item \question{Za pomocą polecenia nslookup w systemie Windows możemy uzyskad informacje o:}%
		%{Tak}{Adresie IP serwera}%
		%{Tak}{Aliasach serwera}%
		%{Nie}{Lokalizacji geograficznej serwera}%
		%{Nie}{Czasu odpowiedzi serwera}

				
		%\item \question{}%
		%{Tak}{}%
		%{Nie}{}%
		%{}{}%
		%{}{}
		
	\end{enumerate}
	
	% --- RAID ----------------------------- %
	%\item \question{}%
%{Tak}{}%
%{Nie}{}%
%{}{}%
%{}{}

% !TeX spellcheck = pl_PL
% ***************************************************************************
% --- Źródło było dość dziwne, sprawdzić potem jeszcze raz wszystkie pytania
% *************************************************************************** 
\newpage
\section{Windows RAID}
	\begin{enumerate}
		\item \question{Na komputerze posiadającym 5 dysków ma zostać zainstalowany system operacyjny Windows 2008 Server, który powinien zapewnić pracę z minimalnym prawdopodobieństwem utraty danych oraz łatwą administracją dyskami. Jaką konfigurację powinien wybrać administrator zakładając, że nie może użyć macierzy sprzętowych?}%
		{Nie}{wszystkie dyski spięte w mirror}%
		{Tak}{2 dyski spięte w mirror, pozostałe 3 dyski spięte w RAID5}%
		{Nie}{wszystkie 5 dysków spiętych w RAID5}%
		{Nie}{dyski spięte w spanned volume, 2 dyski spięte w mirror}
		\item \question{Maksymalna ilość dysków, które mogą ulec awarii bez utraty danych wynosi:}%
		{Nie}{1, dla 2 dysków pracujących w RAID0}%
		{Tak}{1, dla 3 dysków pracujących w RAID5}%
		{Tak}{1, dla 2 dysków pracujących w RAID1}%
		{Nie}{2, dla 3 dysków pracujących w RAID5}
		\item \question{RAID:}%
		{Tak}{jest stosowane w celu zwiększenia niezawodności}%
		{Nie}{wymaga minimum 3 dysków fizycznych do pracy}%
		{Tak}{jest stosowane w celu zwiększenia wydajności transmisji danych}%
		{Tak}{jest stosowane w celu powiększenia przestrzeni dostępnej jako jedna całość}
		\item \question{Mirrored volume w systemie Windows 2008 ma następujące właściwości:}%
		{Tak}{może chronić wolumen bootowalnego systemu operacyjnego Windows 2008}%
		{Nie}{do założenia wymaga 2 identycznych partycji na dyskach typu „basic disk”}%
		{Tak}{można go utworzyć na 2 dyskach}%
		{Nie}{wymaga zakupienia specjalnego kontrolera dysków}
		\item \question{Które z poniższych zdań na temat macierzy RAID5 są prawdziwe?}%
		{Tak}{RAID5 działa poprawnie do awarii więcej niż jednego dysku}%
		{Nie}{Macierz RAID5 wymaga minimum 4 dysków}%
		{Nie}{W n-dyskowej macierzy bity parzystości są na n-1 dyskach}%
		{Tak}{Macierz złożona z n jednakowych dysków ma objętość n-1 dysków}
		\item \question{Aby wykorzystać programowy RAID5 w systemie Windows 2008 Serwer należy posiadać komputer z zainstalowanymi}%
		{Nie}{trzema dyskami}%
		{Nie}{trzema dyskami oraz kontrolerem umożliwiającym systemowi Windows 2008 Server utworzenie programowej macierzy RAID5}
		{Tak}{czterema dyskami}%
		{Tak}{pięcioma dyskami}%
		\newpage
		\item \question{Dla których wolumenów prawdopodobieństwo utraty danych jest większe niż dla wolumenu prostego (simple volume):}%
		{Tak}{spanned volume}%
		{Tak}{striped volume}%
		{Nie}{RAID5}%
		{Nie}{mirrored volume}
		\item \question{Na ilu dyskach można założyć wolumen paskowany używając systemu operacyjnego Windows 2008?}%
		{Nie}{na 1}%
		{Tak}{na 2}%
		{Tak}{na 3}%
		{Tak}{na 4}
		\item \question{Zaznacz poprawne stwierdzenia dotyczące dysków podstawowych i dynamicznych w systemach Windows:}%
		{Nie}{Dyski podstawowe posiadają te same możliwości i funkcje co dyski dynamiczne jednak ich konfiguracja jest nieco trudniejsza}%
		{Nie}{Dyski dynamiczne dostępne są tylko w systemach windows z rodziny serwer}%
		{Tak}{Dyski podstawowe pozwalają na tworzenie podstawowych partycji, rozszerzonych partycji oraz dysków logicznych}%
		{Tak}{W niektórych wersjach systemu windows istnieje możliwość scalenia kilku oddzielnych dynamicznych dysków w jeden wolumen dynamiczny}
		\item \question{Na komputerze posiadającym 6 dysków zostanie zainstalowany system operacyjny Windows 2008 Server. Która konfiguracja pozwoli na pracę z najlepszym wykorzystaniem przestrzeni na dyskach zakładając, że nie można użyć macierzy sprzętowych?}%
		{Nie}{2 dyski spięte w mirror, 3 dyski spięte w RAID5}%
		{Tak}{2 dyski spięte w mirror, pozostałe 4 dyski spięte w wolumen paskowany}%
		{Nie}{wszystkie 6 dysków spiętych w RAID5}%
		{Nie}{utworzone 3 mirrory po 2 dyski każdy}
		\item \question{Na ilu dyskach można założyć wolumen paskowany używając systemu operacyjnego Windows 7?}%
		{Nie}{na 1}%
		{Tak}{na 2}%
		{Tak}{na 3}%
		{Tak}{na 5}
		\item \question{Na komputerze posiadającym 3 dyski zostanie zainstalowany system operacyjny Windows 2008 Server. Która konfiguracja pozwoli na pracę z najlepszym wykorzystaniem przestrzeni na dyskach zakładając, że nie można użyć macierzy sprzętowych?}%
		{Tak}{2 dyski spięte w mirror, jeden dysk bez zabezpieczeń}%
		{Nie}{3 dyski spięte w spanned volume}%
		{Nie}{wszystkie 3 dyski spięte w RAID5}%
		{Nie}{wszystkie dyski spięte w mirror}
		\item \question{Które konfiguracje RAID zwiększają wydajność (gdzie wzrost wydajności należy zrozumieć jako wzrost prędkości odczytu i zapisu)?}%
		{Tak}{RAID0}%
		{Tak}{RAID0+1}%
		{Tak}{RAID1+0}%
		{Nie}{RAID1}
		\item \question{W systemie Windows 7 na 5 dyskach za pomocą systemu operacyjnego został założony RAID5. Po pewnym czasie podczas pracy systemu 1 dysk uległ uszkodzeniu.}%
		{Nie}{odzyskiwanie danych będzie możliwe tylko z ostatniej archiwizacji}%
		{Nie}{jeśli uszkodzony dysk zostanie wymieniony na nowy to po ponownym uruchomieniu systemu dane zostaną automatycznie odzyskane}%
		{Nie}{danych nie będzie można odzyskać}%
		{Tak}{w systemie Windows 7 nie można użyć RAID5}
		{\small \emph{Uzasadnienie:} W systemie Windows 7 nie można założyć RAID5, gdyż taki poziom RAID jest dostępny dopiero w systemach serwerowych.}
		\item \question{Konfiguracja RAID0:}%
		{Tak}{Pojemność wszystkich połączonych dysków jest równa N*pojemność\_najmniejszego\_dysku, gdzie N to liczba połączonych dysków.}%
		{Tak}{Nie dostarcza żadnego zabezpieczenia danych.}%
		{Tak}{Znajduje idealne zastosowanie gdzie wydajność jest ważniejsza od bezpieczeństaw danych.}%
		{Nie}{Pojemność wszystkich połączonych dysków jest równa pojemności najmniejszego z nich.}
		\item \question{Jakie są dostępne typy dysków dynamicznych w systemie Windows 2003?}%
		{Tak}{Mirror}%
		{Tak}{Spanned Volume}%
		{Tak}{Stripped Volume}%
		{Tak}{Simple Volume}
		\item \question{W konfiguracji RAID1:}%
		{Tak}{Dane zapisywane są na obu dyskach równocześnie.}%
		{Nie}{Dane są zapisywane na kolejnych dyskach bit po bicie, tak jak w przypadku RAID2.}%
		{Tak}{Efektywna pojemność wynosi 50\% pojemności dysków.}%
		{Nie}{Wykorzystuje paskowanie dysków.}
		\item \question{Które z poniższych zdań opisują macierz RAID1 (mirroring)?}%
		{Nie}{RAID1 oferuje możliwość strippingu danych.}%
		{Nie}{Całkowita pojemność danych macierzy jest równa pojemności największego dysku.}%
		{Nie}{Pojemność macierzy jest równa pojemności najmniejszego dysku pomnożonego przez liczbę dysków.}%
		{Tak}{Odporność na awarię $ N-1 $ dysków w $ N $-dyskowej macierzy.}
		\newpage
		\item \question{W przypadku którego typu konfiguracji dysków istnieje możliwość odzyskania danych jeśli jeden z dysków macierzy ulegnie awarii?}%
		{Nie}{konfiguracja typu stripped volume}%
		{Tak}{konfiguracja typu RAID5}%
		{Tak}{konfiguracja typu mirror}%
		{Nie}{konfiguracja typu spanned volume}
		\item \question{Mirrored volume w systemie Windows 2008 ma następujące właściwości:}%
		{Tak}{może chronić wolumen z bootowalnym systemem operacyjnym Windows 2008.}%
		{Nie}{może obejmować więcej niż 2 dyski.}%
		{Nie}{całkowicie likwiduje ryzyko utraty danych.}%
		{Tak}{nie można go założyć na dyskach typu "basic disk".}
		\item \question{Który z typów RAID zapewni bezpieczeństwo przy awarii jednego dysku?}%
		{Tak}{RAID0+1}%
		{Nie}{RAID0}%
		{Tak}{RAID1}%
		{Tak}{RAID5}
		\item \question{Wskaż poprawną odpowiedź:}%
		{Tak}{Przestrzeń macierzy w RAID0 jest zależna od rozmiaru najmniejszego z użytych dysków.}%
		{Nie}{RAID0+1 i RAID1+0 udostępniają 100\% sumy pojemności wszystkich użytych dysków.}%
		{Nie}{RAID4 to macierz, której dane na dyskach są paskowane.}%
		{Tak}{Awaria dwóch dysków w RAID6 nie powoduje utraty danych.}
		\item \question{Programowy RAID5 w systemie Windows 2008 Server:}%
		{Nie}{można utworzyć już na 2 dyskach.}%
		{Tak}{można utworzyć na 4 dyskach.}%
		{Tak}{Zwiększa odporność systemu na awarie dysków.}%
		{Nie}{można założyć na dyskach typu "dynamic" lub basic.}
		\item \question{Jakie właściwości ma programowy RAID5 w systemie operacyjnym Windows 2008?}%
		{Tak}{można go założyć na 5 dyskach.}%
		{Nie}{umożliwia lepsze wykorzystanie przestrzeni na dyskach niż wolumen paskowany.}%
		{Nie}{zapewnia bezawaryjną pracę systemu.}%
		{Nie}{pozwala uniknąć fragmentacji systemu plików.}
		\item \question{Zaznacz zdania prawdziwe:}%
		{Nie}{RAID występuje wyłącznie sprzętowy.}%
		{Nie}{RAID występuje wyłącznie programowy.}%
		{Tak}{RAID występuje zarówno programowy jak i sprzętowy.}%
		{Nie}{Nie ma żadnej możliwości uruchomienia RAID w domowym komputerze PC.}
		
		\newpage
		\item \question{Które z podanych zdań są prawdziwe?}%
		{Tak}{RAID programowy pozwala na bezpośredni start systemu z macierzy dyskowej.}%
		{Nie}{RAID sprzętowy posiada wyższą wydajność od RAID programowego, gdyż przeliczaniem sum kontrolnych zajmuje się dedykowany kontroler.}%
		{Nie}{RAID programowy posiada większą kompatybilność z mniej popularnymi systemami operacyjnymi, gdyż wszystkie systemy operacyjne obsługują technologię RAID.}%
		{Tak}{RAID sprzętowy pozwala na bezpośredni start systemu z macierzy dyskowej.}
		\item \question{W systemie windows 2008 na 5 dyskach za pomocą systemu operacyjnego został założony RAID5 Po pewnym czasie podczas pracy systemu 2 dyski uległy uszkodzeniu.}
		{Nie}{jeśli uszkodzone dyski zostaną wymienione na nowe to po ponownym uruchomieniu systemu dane zostaną automatycznie odzyskane}
		{Nie}{odzyskiwanie danych będzie przezroczyste dla użytkowników jeśli dyski są typu hot swap}
		{Nie}{w systemie Windows 2008 nie można użyć RAID5}
		{Tak}{dane będzie można odzyskać tylko z archiwizacji, a nie z RAID5}
		{\small \emph{Uzasadnienie:} po awarii 2 dysków RAID5 traci dane.}
		\item \question{Jakie właściwości ma programowy RAID5 na systemie operacyjnym Windows 2008?}%
		{Tak}{można go założyć na pięciu dyskach}%
		{Nie}{umożliwia lepsze wykorzystanie przestrzeni na dyskach niż wolumen paskowany}%
		{Nie}{zapewnia bezawaryjną pracę systemu}%
		{Nie}{pozwala uniknąć fragmentacji systemu plików}
		\item \question{Konfiguracja RAID2:}%
		{Tak}{jest rozszerzeniem architektury RAID0}%
		{Tak}{dane są zapisywane na kolejnych dyskach macierzy bit po bicie}%
		{Nie}{cechuje się dużą wydajnością przy operacjach odczytu}%
		{Nie}{jest często stosowana w macierzach dyskowych}
		\item \question{Dyski typu podstawowego (ang. basic disks) pozwalają na:}%
		{Tak}{oznaczenie partycji jako aktywnej}%
		{Nie}{rozszerzenie woluminów prostych (ang. simple volume)}%
		{Tak}{tworzenie partycji podstawowej}%
		{Nie}{tworzenie woluminów RAID5}
		\item \question{Dla których wolumenów prawdopodobieństwo utraty danych jest mniejsze niż dla wolumenu łączonego (spanned volume):}%
		{Tak}{mirrored volume}%
		{Nie}{striped volume}%
		{Tak}{simple volume}%
		{Tak}{RAID5}
		\item \question{Jakie właściwości ma programowy RAID5 na systemie operacyjnym Windows 2008?}%
		{Nie}{zapewnia bezawaryjną pracę systemu}%
		{Tak}{chroni system przed awarią tylko jednego dysku}%
		{Nie}{pozwala uniknąć fragmentacji systemu plików}%
		{Nie}{umożliwia lepsze wykorzystanie przestrzeni na dyskach niż wolumen paskowany}%		
		
		%\item \question{}%
		%{Tak}{}%
		%{Nie}{}%
		%{}{}%
		%{}{}
		
	\end{enumerate}
	
	% --- Interpreter poleceń PowerShell --- %
	%\item \question{}%
%{Tak}{}%
%{Nie}{}%
%{}{}%
%{}{}

% !TeX spellcheck = pl_PL
\newpage
\section{Interpreter poleceń PowerShell}
	\begin{enumerate}
		\item \question{Polecenie$>$ get-childitem C:\textbackslash test\textbackslash * -include *.txt -recurse | remove-item}%
		{Tak}{Znajduje i usuwa wszystkie pliki z rozszerzeniem .txt z folderu "C:\textbackslash test" i podfolderów.}%
		{Nie}{Znajduje i usuwa wszystkie pliki z rozszerzeniem .txt z folderu "C:\textbackslash test", bez podfolderów.}%
		{Nie}{Znajduje i wypisuje wszystkie pliki z rozszerzeniem .txt z folderu "C:\textbackslash test", bez podfolderów.}%
		{Nie}{Jest niepoprawne}
		\item \question{Które wersje systemu Windows NIE wpierają PowerShella?}%
		{Tak}{Windows 2000 SP4}%
		{Tak}{Windows 2000}%
		{Nie}{Windows Server 2008}%
		{Nie}{Windows 7}
		\item \question{Które polityki wykonywania skryptów w PowerShell zabraniają wykonywania skryptów pochodzących z lokalnego komputera, jeśli skrypty te nie są podpisane przez zaufanego wydawcę?}%
		{Tak}{Restricted}%
		{Tak}{AllSigned}%
		{Nie}{RemoteSigned}%
		{Nie}{Unrestricted}
		\item \question{Po wykonaniu poniższego skryptu w PowerShell\\
			\$przedmiot = "DSO" if (\$przedmiot -eq "DSO") {"Dedykowane Systemy Operacyjne"} elseif (\$przedmiot -eq "PK") {"Programowanie Komputerów"} else {"Nieznany przedmiot"}}%
		{Nie}{Na ekranie zostanie wyświetlony napis "Nieznany przedmiot".}%
		{Tak}{Wartość zmiennej \$przedmiot nie ulegnie zmianie.}%
		{Nie}{Na ekranie pojawi się komunikat o błędzie składniowym.}%
		{Nie}{Do zmiennej \$przedmiot zostanie przypisana wartość "Dedykowane Systemy Operacyjne".}
		\newpage
		\item \question{Aby zwrócić wszystkie obiekty w bieżącej lokalizacji nalezy użyć polecenia:}%
		{Tak}{Get-children}%
		{Nie}{Copy-item}%
		{Nie}{Get-content}%
		{Nie}{Get-process}
		\item \question{Polecenie "PS$>$ get-process d* | stop-process"}%
		{Tak}{poszczególne polecenia należą do tzw. poleceń Cmdlet.}%
		{Nie}{zatrzymuje wszystkie uruchomione procesy.}%
		{Nie}{zatrzymuje wszystkie procesy działające na partycji D.}%
		{Tak}{zatrzymuje wszystkie procesy których nazwa rozpoczyna się literą "d".}
		\item \question{Aby zwrócić wszystkie obiekty w bieżącej lokalizacji należy użyc polecenia:}%
		{Nie}{Get-process}%
		{Nie}{Copy-item}%
		{Nie}{Get-content}%
		{Tak}{Get-children}
		\item \question{Zaznacz poprawne przyporządkowania aliasów do Cmdletów}%
		{Nie}{taskkill -$>$ Kill-Process}%
		{Tak}{ls -$>$ Get-Children}%
		{Tak}{help -$>$ Get-Help}%
		{Tak}{man -$>$ Get-Help}
		\item \question{Polecenie Get-EventLog w Windows PowerShell pozwala:}%
		{Nie}{Zapisywać informacje do dziennika zdarzeń.}%
		{Nie}{Takie polecenie nie istnieje.}%
		{Tak}{Pobierać wpisy z dziennika zdarzeń.}%
		{Nie}{Pobierać wpisy z pliku C:\textbackslash Var\textbackslash Log\textbackslash Messages.}
		\item \question{Polecenia natywane dla Windows PowerShell, które pozwalają na wykonywanie podstawowych operacji na obiektach w środowisku WPS to:}%
		{Nie}{Potoki (pipelines)}%
		{Tak}{Aplety poleceń (cmdlets)}%
		{Nie}{Aplety skryptowe (scriptlets)}%
		{Nie}{Komendy linii poleceń (line commands)}
		\item \question{Wskaż wszystkie poprawne zdania dotyczące interpretera Windows PowerShell:}%
		{Tak}{PowerShell jest oparty o .NET}%
		{Nie}{PowerShell nie udostępnia mechanizmów potoku.}%
		{Tak}{PowerShell pozwala ustawić różne polityki kontrolujące jakie skrypty można uruchomić.}%
		{Nie}{PowerShell jest kompatybilny z bashem.}
		\newpage
		\item \question{Polityka Restricted wykonywania plików:}%
		{Tak}{Jest domyślną polityką w środowisku PowerShell.}%
		{Nie}{Pozwala na uruchamianie skryptów z rozszerzeniem .ps1.}%
		{Nie}{Nie pozwala na wykonywanie komend w oknie interpretera.}%
		{Nie}{Pozwala na uruchamianie skryptów z rozszerzeniem .ps1xml.}
		\item \question{Które polecenie wypisze zawartość bieżącego katalogu z pominięciem plików o rozszerzeniu .exe?}%
		{Nie}{Dir *.exe}%
		{Tak}{gci -exclude *.exe}%
		{Tak}{Get-Children -exclude *.exe}%
		{Nie}{ls -include *.exe}
		\item \question{Wskaż poprawne polecenia PowerShell usuwające z bieżącego katalogu pliki większe niż 2kB:}%
		{Nie}{Get-Childitem | Where-Object ( \$\_.length $>$ 2kB ) | Remove-Item}%
		{Nie}{Get-Childitem | Remove-Item | Where ( \$\_.length $>$ 2kB )}%
		{Tak}{Get-Childitem | Where-Object ( \$\_.length -gt 2kB ) | Remove-Item}%
		{Tak}{ls | where-object \{ \$\_.length -gt 2kB \} | rm}
		\item \question{Polecenie\\ "PS$ > $ get-process | where-object { \$\_.WS -gt 300MB } | stop-process"\\ wydane w interpreterze Windows PowerShell:}%
		{Nie}{Listuje procesy, które zużywają więcej niż 300 MB.}%
		{Nie}{Szuka procesu, który zużywa więcej niż 300 MB i wyświetla nazwę.}%
		{Tak}{Szuka procesu, który zużywa więcej niż 300 MB i zatrzymuje go.}%
		{Nie}{Szuka procesu, który zużywa mniej niż 300 MB i zatrzymuje go.}
		\item \question{Która z wersji systemu Windows obsługuje interpreter PowerShell?}%
		{Tak}{Windows Vista}%
		{Tak}{Windows 7}%
		{Tak}{Windows XP SP2/SP3}%
		{Nie}{Windows 95}
		\item \question{Polecenie Set-Location w Cmdlets (PowerShell) ma swój odpowiednik w interpreterze komend cmd.exe. Jest to:}%
		{Tak}{chdir}%
		{Nie}{set}%
		{Nie}{sloc}%
		{Tak}{cd}
		\newpage
		\item \question{Które z poleceń są poprawnymi podstawowymi aliasami w Windows PowerShell, służącymi do czyszczenia ekranu?}%
		{Nie}{Clear-Console}%
		{Nie}{Clear-Host}%
		{Tak}{clear}%
		{Tak}{cls}
		\item \question{W celu zatrzymania procesów zużywających więcej niż 100MB pamięci RAM należy użyć polecenia:}%
		{Nie}{PS$ > $ stop-process | where-object { \$\_.WS -gt 100MB }}%
		{Nie}{PS$ > $ stop-process \$Memory -gt 100MB}%
		{Nie}{PS$ > $ get-process | where-object { \$Memory -gt 100MB } | stop-process}%
		{Tak}{PS$ > $ get-process | where-object { \$\_.WS -gt 100MB } | stop-process}
		\item \question{Zaznacz poprawne zdania dotyczące powłoski PowerShell:}%
		{Tak}{Wszystkie zmienne są obiektami .NET.}%
		{Tak}{Do zmiennych odwołuje się używając znaku \$.}%
		{Nie}{Część zmiennych jest obiektami .NET.}%
		{Nie}{Do zmiennych odwołuje się używając znaku \#.}
		\item \question{Za pomocą polecenia:\\Get-Childitem C:\textbackslash Work\textbackslash  -Recurse -Force | Measure-Object -property length -sum\\(Komentarz: polecenie measure-object służy do generowania statystyk)}%
		{Tak}{Znajdziemy liczbę plików i ich całkowity rozmiar w folderze C:\textbackslash Work oraz w podfolderach.}%
		{Nie}{Wypiszemy zawartość folderu C:\textbackslash Work.}%
		{Nie}{Wypiszemy największy plik z folderu C:\textbackslash Work.}%
		{Nie}{Jest to niepoprawna składnia.}
		\item \question{Aby usunąć wszystkie pliki z katalogu c:\textbackslash temp\ o rozszerzeniu .xls w Windows PowerShell należy użyć polecenia:}%
		{Tak}{remove-item c:\textbackslash temp\textbackslash *.xls}%
		{Tak}{get-childitem c:\textbackslash temp\textbackslash *.xls | foreach-object { remove=item \$\_.fullname }}%
		{Nie}{remove-item c:\textbackslash temp\textbackslash * -exclude *.xls}%
		{Nie}{remove-file c:\textbackslash temp\textbackslash * -extension xls}
		\item \question{Polecenie:\\PS$ > $ get-childitem C:\textbackslash test\textbackslash * -include *.txt -recurse | remove-item }%
		{Tak}{Znajduje i usuwa wszystkie pliki z rozszerzeniem .txt z folderu "C:\textbackslash test" i podfolderów}%
		{Nie}{Znajduje i usuwa wszystkie pliki z rozszerzeniem .txt z folderu "C:\textbackslash test", bez podfolderów}%
		{Nie}{Znajduje i wypisuje wszystkie pliki z rozszerzeniem .txt z folderu "C:\textbackslash test", bez podfolderów}%
		{Nie}{Jest niepoprawne.}
		\newpage
		\item \question{Jakie rozszerzenia mogą mieć skrypty PowerShell?}%
		{Nie}{.wps}%
		{Nie}{.shl}%
		{Nie}{.cmd}%
		{Tak}{.ps1}
		\item \question{Której z niżej wymienionych polityk uruchamiania skryptów są dostępne w powerShell systemu Windows?}%
		{Nie}{NoneAllowed - nie pozwala na uruchamianie żadnych skryptów.}%
		{Tak}{AllSigned - możliwość uruchomienia tylko podpisanych skryptów.}%
		{Tak}{RemoteSigned - możliwość uruchamiania skryptów lokalnych oraz podpisanych pochodzących z Internetu.}%
		{Tak}{Unrestricted - pozwala na uruchamianie wszystkich skryptów.}
		\item \question{Czym charakteryzują się komendy (tzw. cmdlety) w PowerShell?}%
		{Tak}{Zazwyczaj zwracają obiekty.}%
		{Nie}{Nie mogą mieć zdefiniowanych kilku aliasów jednocześnie.}%
		{Nie}{Mają nazwy postaci "rzeczownik-czasownik"}%
		{Tak}{Mają nazwy postaci "czasownik-rzeczownik"}
		\item \question{Aby uzyskać pomoc na temat poleceń w Windows PowerShell należy użyć polecenia:}%
		{Nie}{please}%
		{Tak}{help}%
		{Nie}{Oh genie}%
		{Tak}{Get-Help}
		\item \question{Aby sprawdzić czy jakiś katalog już istnieje w Windows PowerShell można skorzystac z poleceń:}%
		{Nie}{remove-item}%
		{Tak}{test-path}%
		{Nie}{path}%
		{Nie}{mew-item}
		\item \question{Wskaż wszystkie prawdziwe zdania dotyczące interpretera Windows PowerShell:}%
		{Tak}{Polecenie ls jest aliasem polecenia Get-Children.}%
		{Nie}{PowerShell nie posiada modułów i przystawek pozwalających na rozszerzanie powłoki poprzez dodawanie nowych cmdletów.}%
		{Nie}{W systemie operacyjnym Windows XP SP2 domyślnie zainstalowaną wersją PowerShella jest wersja "PowerShell v2"}%
		{Tak}{PowerShell pozwala na przetwarzanie potokowe, które pozwala na przekazywanie obiektu z jednego cmdletu do drugiego, bez potrzeby korzystania z parsowania tekstu czy zmiany formatowania.}
		\newpage
		\item \question{Polecenie:
			"new-item c:\textbackslash temp\textbackslash test -type directory"\\
			spowoduje:}%
		{Nie}{Utworzenie katalogu directory w katalogu c:\textbackslash temp\textbackslash test}%
		{Nie}{Sprawdzi istnienie katalogu test w katalogu c:\textbackslash temp}%
		{Tak}{Utworzenie katalogu test w katalogu c:\textbackslash temp}%
		{Nie}{Sprawdzi czy "test" w katalogu c:\textbackslash temp jest katalogiem}
		\item \question{Które wersje systemu Windows NIE wpierają PowerShella?}%
		{Nie}{Windows Vista}%
		{Tak}{Windows 2000}%
		{Nie}{Windows XP SP2}%
		{Nie}{Windows 7}
		\item \question{Wskaż wszystkie prawdziwe zdania dotyczące interpretera Windows PowerShell:}%
		{Tak}{Wszystkie zmienne są obiektami .NET.}%
		{Tak}{Aby skopiować plik należy wpisać polecenie "Copy-item lokalizacja1 lokalizacja2"}%
		{Nie}{Aby skopiować plik należy wpisać polecenie "Set-Location lokalizacja1 lokalizacja2"}%
		{Tak}{PowerShell jest elementem pakietu Windows Management Framework.}
		\item \question{W Windows PowerShell poprawnie stworzona pętla to:}%
		{Tak}{ \$a = 1 do { \$a; \$a++ } while (\$a -lt 10) }%
		{Nie}{ \$a = 10 do { \$a; \$a-- } while (\$a -lt 3) }%
		{Tak}{ for (\$a = 1; \$a -le 10; \$a++) { \$a } }%
		{Nie}{ foreach ( \$i in get-child c:\textbackslash scripts ) {\$i.extended} }
		\item \question{Co należy wstawić w miejsce znaków zapytania, aby poniższy skrypt PowerShella wyświetlał nazwę procesu w danej chwili najbardziej obciążającego procesor?\\
			\$ps = get-process\\
			\$max = \$ps[0]\\
			foreach (\$p in \$ps )\\
			{\\
				if ( ??? )\\
				{ \$max = \$p }
			}\\
			\$max.processname
			}%
		{Nie}{ \$p $ > $ \$max }%
		{Tak}{ \$p.cpu -gt \$max.cpu }%
		{Nie}{Brak odpowiedzi w źródle.}%
		{Nie}{Brak odpowiedzi w źródle.}
		
	\end{enumerate}
	
	% --- Windows API ---------------------- %
	

% Gdzieniegdzie w pytaniach są oznaczenia W12-xx i Ox. Cholera wie co to, może jakieś identyfikatory w bazie danych pytań. Nie przepisywałem ich. %

\newpage
\section{Windows API}

\begin{enumerate}
	
	\item \question{Do funkcji Windows APi należą:}
	{Tak}{CreateWindowsEx}
	{Nie}{strcmp}
	{Tak}{ShowWindow}
	{Nie}{atoi}
	
	\item \question{Kiedy musi być zarejestrowana klasa okna w Windows API}
	{Nie}{klasa okna może być zarejestrowana zarówno przed jak i po utworzeniu okna}
	{Tak}{przed utworzeniem okna}
	{Nie}{po utworzeniu okna}
	{Nie}{klasa okna nie jest rejestrowana w Window API}
	
	\item \question{HWND:}
	{Nie}{Jest strukturą przechowującą wskaźniki do poszczególnych okien aplikacji}
	{Nie}{Jest wskaźnikiem na funkcję obsługującą komunikaty napływające do okna aplikacji}
	{Tak}{Jest uchwytem okna aplikacji}
	{Nie}{Jest funkcją pozwalającą na zdefiniowanie głównego okna aplikacji}
	
	\item \question{Aby wyświetlić krótki komunikat dla użytkownika przy użyciu okna modalnego można użyć funkcji}
	{Nie}{ShowDialog(...)}
	{Nie}{MsgBox(...)}
	{Tak}{MessageBox(...)}
	{Nie}{ShowModDialog(...)}
	
	\item \question{Kod programów pisanych z bezpośrednim wykorzystaniem Win32API musi zawierać:}
	{Nie}{Instrukcję $\sharp$include}
	{Nie}{Wywołanie funkcji CreateWindowEx(...)}
	{Tak}{Funkcję WinMain(...)}
	{Nie}{Funkcję WINAPI(...)}
	
	\item \question{Windows API pozwala na:}
	{Tak}{komunikację sieciową}
	{Tak}{ostęp do systemu plików}
	{Tak}{tworzenie interfejsu graficznego}
	{Tak}{dostęp do rejestrów systemu}

	\item \question{MDi w API jest skrótem od:}
	{Nie}{Media Download Interface}
	{Nie}{Mass Data Interface}
	{Tak}{Multiple Data Interface}
	{Nie}{Multicolor Data Interface}

	\newpage

	\item \question{UpdateWindow:}
	{Tak}{Jest funkcją wysyłającą komunikat do okna aplikacji informującym go o potrzebie przerysowania}
	{Nie}{Jest domyślną funkcją obsługującą przerysowanie okna lub jego fragmentu}
	{Nie}{Jest komunikatem wysyłanym do okna bezpośrednio po jego wyświetleniu}
	{Nie}{Jest komunukatem wysyłanym do okna aplikacji informującym go o potrzebe przerysowania}
	
	\item \question{Czy dany przycisk został naciśnięty możemy sprawdzić poprzez:}
	{Tak}{Porównanie uchwytu do przycisku wewnątrz procedury obsługi komunikatów przy zdarzeniu \texttt{WM\_COMMAND}}
	{Nie}{Porównanie adresu kontrolki przycisku}
	{Tak}{Porównanie ID przypisanego do przycisku wewnątrz procedury obsługi komunikatów przy zdarzeniu \texttt{WM\_COMMAND}}
	{Nie}{Wykonanie procesury obsługi przerwania danego przycisku}
	
	\item \question{Wyświetlenie okna Message Box:}
	{Nie}{Powoduje utworzenie dla niego nowego procesu w systemie}
	{Tak}{Jest wywołaniem blokującym (blokuje wykonanie dalszej części kodu aż do zamknięcia Message Box'a)}
	{Nie}{Polega na obsłudze odpowiedniego komunikatu w pętli obsługi komunikatów.}
	{Tak}{Możemy uzyskać poprzez wywołanie kodu: MessageBox(NULL, L"Welcome to Win32 Application Development$\backslash$n", NULL, NULL);}
	
	\item \question{DefWindowProc}
	{Tak}{Jest domyślną funkcją obsługującą komunikaty napływające do okna aplikacji}
	{Nie}{Jest wskaźnikiem na funkcję obsługującą komunikaty napływające do okna aplikacji}
	{Nie}{Jest funkcją pozwalającą na zdefiniowanie głównego okna aplikacji}
	{Nie}{Jest strukturą pozwalająca na m.in. zdefiniowanie głównego okna aplikacji}

	\item \question{Jakie rodzaje komunikatów mogą docierać do okna?}
	{Tak}{zmiana rozmiaru okna}
	{Tak}{pojedyncze bądź podwójne kliknięcie myszą w obszarze okna}
	{Tak}{zmiana położenia okna}
	{Tak}{naciśnięcie klawisza}
	
	\item \question{WNDCLASS$\slash$WNDCLASSEX}
	{Nie}{Obsługuje kolejkę komunikatów napływających do okna aplikacji}
	{Nie}{Jest strukturą przechowującą wskaźniki do poszczególnych okien aplikacji}
	{Tak}{Jest strukturą pozwalającą zdefiniować np. koloty okna aplikacji}
	{Nie}{Jest odpowiednikiem funkcji main() w programach pisanych w WinAPI}
	
\end{enumerate}
	
% ======== LINUX ============================================== %
\part{Linux}
	% --- Usługi graficzne Xwindow --------- %
	\section{Usługi graficzne Xwindow}

\begin{enumerate}
	
	\item \question{Wskaż wszystkie poprawne stwierdzenia odnoszące się do X Window System}
	{Tak}{Został on zaprojektowany w architekturze klient-serwer}
	{Tak}{Jest to zbiór funkcji i protokołów wyświetlających informacje graficzne na ekranie}
	{Nie}{Odpowiada za wygląd okien wyświetlanych w systemie}
	{Tak}{Pozwala na zdalną pracę na odległym komputerze, wykorzystując komputer lokalny jako serwer X}

	\item \question{Które z podanych komponentów NIE wchodzi w skład X Window System}
	{Tak}{Serwer Apache}
	{Nie}{Menadżer okien}
	{Tak}{Baza danych}
	{Nie}{X serwer}
	
	\item \question{Czym różnią się zdm/gdm/lightdm i startx?}
	{Tak}{Gdy X zostanie opuszczony za pomocą polecenia zakończenia menadżera okna \textbf{Xdm} ponownie pokazuje ekran logowania}
	{Tak}{Xdm/Gdm/lightdm uruchamia ekran logowania}
	{Nie}{Startx uruchamia ekran logowania}
	{Nie}{Gdy X zostanie opuszczony za pomocą polecenia zakończenia menadżera okna \textbf{startx} ponownie pokazuje ekran logowania}
	
	\item \question{Polecenie Xorg -configure}
	{Nie}{Jest narzędziem graficznym}
	{Tak}{Pracuje w trybie tekstowym}
	{Tak}{Służy do konfiguracji X-serwera}
	{Tak}{Modyfikuje/Generuje domyślny plik Xorg.conf}

	\item \question{Wpis do /etc/X11/xorg.conf: Section "Device"
		Identifier "Videocard0"
		Driver "nvidia"
		Endsection}
	{Tak}{wykorzysta sterownik nvidia do obsługi pierwszej karty graficznej}
	{Nie}{jest niepoprawnym wpisem}
	{Nie}{utworzy nową wirtualną kartę graficzną}
	{Nie}{nic nie zmieni, bo plik konfiguracyjny Xorg znajduje się w innej lokalizacji}
	
	\item \question{Manager okien w systemie Linux}
	{Nie}{Jest X-Serwerem}
	{Nie}{zarządza pamięcią X-serwera}
	{Tak}{Jest odpowiedzialny za wygląd i funkcjonalność pulpitu}
	{Tak}{Jest odpowiedzialny za wygląd okien}
	
	\item \question{Wartości domyślne używane przez standardowe aplikacje Systemu X mogą zostać zmienione. Służą do tego pliki w katalogu:}
	{Nie}{\textasciitilde/app-defaults/}
	{Tak}{/etc/X11/app-defaults/}
	{Nie}{\textasciitilde/defaults-app-values/}
	{Nie}{/etc/X11/default-app-values}
	
	\item \question{Dostępne są 2 komputery, serwer - saturn, oraz klient - jupiter. Po wykonaniu komend na komputerze saturn: \\
		\$ xhost +jupiter \\
		na komputerze jupiter: \\
		\$ export DISPLAY=saturn:0 \\
		\$ xeyes \\
		Efektem będzie:}
	{Nie}{Wynik programu "xeyes" widziany będzie na obu komputerach}
	{Tak}{Wynik programu "xeyes" widziany będzie tylko na komputerze saturn}
	{Tak}{Program "xeyes" wykonany zostanie na komputerze jupiter}
	{Nie}{Program "xeyes" wykonany zostanie na komputerze saturn}

	\item \question{Menadżerem okien jest:}
	{Nie}{gdm}
	{Nie}{lightdm}
	{Tak}{KDE}
	{Tak}{Gnome}
	
	\item \question{X11 (X Window System) to:}
	{Tak}{Graficzny system komputerowy}
	{Nie}{Manager okien}
	{Nie}{Aplikacja pozwalająca na zalogowanie się do systemu}
	{Nie}{żadna z powyższych}	

	\item \question{System X}
	{Tak}{jest zaprojektowany w architekturze klient-serwer}
	{Nie}{odpowiada za obsługę okien}
	{Tak}{odpowiada za obsługę urządzeń wejścia}
	{Nie}{odpowiada za zamykanie/otwieranie programów}

	\item \question{X Window Server}
	{Tak}{...zajmuje się obsługą urządzeń wejściowych (myszki, klawiatury, tabletu).}
	{Nie}{...dostarcza rozbudowany interfejs użytkownika.}
	{Nie}{...zajmuje się obsługą okien, dostarcza wbudowane mechanizmy do ich przesuwania, zmiany rozmiaru, zamykania i uruchamiania programów itd.}
	{Tak}{...udostępnia interfejs graficzny i pozwala rysować nieskomplikowane elementy na ekranie.}
	
	\item \question{Zaznacz implementacje X Window System}
	{Tak}{XFree86}
	{Nie}{Gnome}
	{Nie}{KDE}
	{Tak}{X.Org}
	
	\item \question{Dodatkowe skrypty startowe Systemu X Window mogą być zdefiniowane w}
	{Tak}{\textasciitilde/.xinitrc}
	{Tak}{/etc/X11/xinit/xinitrc}
	{Nie}{/etc/xorgrc}
	{Nie}{\textasciitilde/.xorgrc}
	
	\item \question{Podaj polececenie potrzebne o uruchomienia Xwindow}
	{Tak}{startx}
	{Nie}{/etc/init.d/gdm start}
	{Nie}{/etc/X11/xorg start}
	{Nie}{setx start}
	
	\item \question{Domyślne skróty klawiszowe dla serwera X, to:}
	{Tak}{[Alt]+[Ctrl]+[FX], gdzie X={1,2...7} - przełączanie się między konsolami tekstowymi. Zazwyczaj [Alt] + [F7] pozwala na przełączenie z trybu tekstowego  w tryb graficzny.}
	{Nie}{[Alt] + [Ctrl] + [F12] - otwiera tekstowy menadżer konfiguracji serwera X.}
	{Nie}{[Alt] + [Esc] - restart serwera X}
	{Tak}{[Ctrl] + [Alt] + [Backspace] - wyłączenie serwera X.}
	
	\item \question{W jaki sposób można uruchomić powłokę graficzną w systemie Linux?}
	{Tak}{Skorzystać z menadżera wyświetlania, np. xdm}
	{Tak}{Uruchomić aplikację startową dostarczaną wraz ze środowiskiem graficznym, np startxfce4}
	{Tak}{Może być skonfigurowany do uruchomienia na odpowiednim poziomie uruchomieniowym}
	{Tak}{Skorzystać ze skryptu startowego startx/xinit}

	\item \question{Plik /etc/X11/Xorg.conf pozwala na zmianę:}
	{Tak}{Ustawień myszy i klawiatury.}
	{Tak}{Modelu używanej karty graficznej i jej parametrów.}
	{Tak}{Rozdizelczości ekranu oraz częstotliwości odświeżania.}
	{Tak}{Zakres odświeżania pionowego dla używanego monitora.}
	
	\item \question{Uruchomienie w konsoli któregoś z menadżerów ekranu (ang. Display Manager, np gdm, xdm, lightdm) przez użytkownika root, przy założeniu, że X nie jest uruchomiony, spowoduje:}
	{Nie}{nie można uruchomić menadżera ekranu z konsoli}
	{Nie}{uruchomienie sesji X użytkownika, który uruchamiał polecenie}
	{Nie}{zakończenie sesji użytkownika root, w której wykonał polecenie}
	{Tak}{wyświetlenie ekranu logowania}
	
	\item \question{W skaład X-Window wchodzi:}
	{Tak}{Menadżer Okien}
	{Nie}{X-Writer}
	{Tak}{X-Serwer}
	{Tak}{X-klient}
	
	\item \question{Zazanacz zdania prawidzwe na temat podsystemu graficznego X Windows:}
	{Nie}{Jego implementacją jest np. Gnome lub KDE.}
	{Tak}{Jego implementacją jest X.org oraz XFree86}
	{Nie}{Po jego uruchomieniu oraz systemu Linux istnieje możliwość przejścia z trybu graficznego do konsoli tekstowej za pomocą skrótu ALT+CTRL+\textbf{1}}
	{Tak}{Po jego uruchomieniu oraz systemu Linux istnieje możliwość przejścia z trybu graficznego do konsoli tekstowej za pomocą skrótu ALT+CTRL+\textbf{F1}}
	
	\item \question{Plik /etx/X11/xorg.conf}
	{Nie}{(Nie wiadomo co jest tu napisane, zdaniem starszych roczników fałsz)}
	{Nie}{Zawiera ustawienia menadżera okien, takie jak np. ułożenie ikon na pulpicie, kolory, style obramowania okien itp.}
	{Tak}{Zawiera konfigurację urządzeń wejścia/wyjścia podłączonych do komputera}
	{Nie}{Jest plikiem wykonywalnym}
	
	\item \question{Wskaż poprawne zdania dotyczące pliku konfiguracyjnego Xorf.conf}
	{Nie}{W pliku Xorg.conf może znaleźć się tylko jedna sekcja Device}
	{Tak}{Rozdzielczość monitora definiuje się po słowie Modes}
	{Nie}{W jednej sekcji Display może zdefiniować maksymalnie jedną \textbf{rozdzielczość} monitora.}
	{Tak}{W jednej sekcji Display może zdefiniować maksymalnie jedną \textbf{głębię kolorów} monitora.}
	
	\item \question{Zaznacz prawidłowe stwierdzenia:}
	{Tak}{xinit wywołuje xterm}
	{Nie}{xterm wywołuje xinit}
	{Tak}{startx wywołuje xinit}
	{Nie}{xinit wywołuje xstart}
	
	\item \question{Plik konfiguracyjny X-Serwera (w systemie X.org)}
	{Tak}{nie jest wymagany (x-serwer wykona wtedy konfigurację dynamiczną)}
	{Nie}{musi zawierać sekcje Device, Monitor, Screen, Keyboard, Mouse}
	{Nie}{musi zawierać przynajmniej sekcję Device}
	{Nie}{musi zawierać skecje Device, Monitor, Screen oraz Display}
	
	\item \question{W pliku /etx/X1/xorg.conf mamy możliwość skonfigurowania:}
	{Tak}{rozdzielczości, z jaką startuje system graficzny}
	{Tak}{myszy}
	{Nie}{drukare, które są dostępne w systemie}
	{Tak}{sterownika grafiki, z którego skorzystać ma system}
	
	\item \question{X Window System:}
	{Tak}{zawiera mechanizmy obsługi klawiatury i myszy}
	{Nie}{dostarcza graficzny interfejs użytkownika (okna, przyciski itd.)}
	{Nie}{Jest rozbudowanym serwerem VNC}
	{Tak}{zawiera protokoły sieciowe umożliwiające wykonywanie programów X w jednym komputerze i wyświetlanie rezultatu ich pracy na drugim}
	
	\item \question{Które z podanych zdań prawidłowo opisują architekturę X Widnow System?}
	{Tak}{Serwer X jest lokalny i działa na komputerze użytkownika.}
	{Nie}{Klienci zawsze działają lokalnie, natomiast serwer X może działać na innej maszynie.}
	{Tak}{Klienci mogą działać na różnych maszynach.}
	{Nie}{Zarówno serwer X, jak i klienci muszą działać lokalnie, na komputerze użytkownika.}
	
	\item \question{Zaznacz zdania prawdziwe dotyczące systemu Linux}
	{Nie}{Środowisko graficzne X jest uruchamiane zawsze przy starcie systemu, niezależnie od konfiguracji.}
	{Tak}{W czasie pracy w sieci z wykorzystaniem Xwindow: X-Serwer jest uruchomiony na lokalnym komputerze, z którego odbywa się sterowanie, natomiast X-Klient na serwerze zdalnym, gdzie odbywa się przetwarzanie danych.}
	{Nie}{Xwindow pozwala pracować jedynie w trybie z jednym użytkownikiem.}
	{Tak}{Przejścia między konsolami tekstowymi odbywa się przy pomocy klawiszów [Alt]+[Ctrl]+[F1] do [F6]}
	
	\item \question{Co jest dodatkowym elementem systemu X Window}
	{Tak}{Serwer czcionek}
	{Tak}{Zarządca okien (window manager)}
	{Nie}{Serwer plików tekstowych}
	{Nie}{Zarządca sieci (network-manager)}
	
	\item \question{Jakie sekcje może zawierać plik Xorg.conf}
	{Nie}{WindowManager}
	{Tak}{Device}
	{Tak}{Screen}
	{Tak}{Monitor}
\end{enumerate}

	
	% --- Linux ACL ------------------------ %
	\section{Linux ACL}
	
	% --- Linux RAID ----------------------- %
	%\item \question{}%
%{Tak}{}%
%{Nie}{}%
%{}{}%
%{}{}

% !TeX spellcheck = pl_PL
% *****************************************************
% Możliwe, że coś się wymieszało z Windowsem,
% do sprawdzenia bliżej laborek
% *****************************************************
\newpage
\section{Linux RAID}
\begin{enumerate}
	\item \question{Macierz typu raid 5 złożona z 3 dysków o jednakowej pojemności i parametrach:}%
	{Nie}{ma pojemność 2 dysków i nie jest odporna na awarię ani jednego dysku}%
	{Tak}{oferuje spowolniony odczyt w przypadku awarii 1 dysku}%
	{Nie}{ma pojemność 1 dysku i jest odporna na awarię maksymalnie 2 dysków}%
	{Tak}{ma pojemność 2 dysków i jest odporna na awarię maksymalnie 1 dysku}
	\item \question{W systemie Ubuntu, zakładając, że pliki blokowe /dev/sdb1 i /dev/sdb2 reprezentują partycje o rozmiarze 50MB, bezpośrednio po utworzeniu woluminu /dev/md0 poleceniem:\\
		mdadm $ -- $create $ -- $verbose /dev/md0 $ -- $level=linear $ -- $raid-devices=2\\/dev/sdb1/dev/sdb2:}%
	{Tak}{wolumin /dev/md0 będzie miał wielkość 100MB}%
	{Nie}{wolumin /dev/md0 będzie miał wielkość 50MB}%
	{Nie}{wolumin /dev/md0 będzie można zamontować poleceniem mount /dev/md0 /mnt}%
	{Tak}{uszkodzenie dokładnie jednego spośród urządzeń /dev/sdb1 oraz /dev/sdb2 może spowodować utratę danych w woluminie /dev/md0}
	\item \question{Zaznacz prawdziwe stwierdzenia:}%
	{Tak}{Sprzętowy RAID oferuje większą wydajność poprzez zmniejszenie obciążenia CPU, gdyż przeliczaniem sum kontrolnych zajmuje się wówczas dedykowany kontroler.}%
	{Nie}{RAID sprzętowy jest niekompatybilny z dużą liczbą systemów operacyjnych, ze względu na zachowanie odróżniające taki RAID od pojedynczego dysku twardego.}%
	{Tak}{RAID software'owy oferuje możliwość łączenia różnych interfejsów takich jak ATA, SCSI, SATA, USB w obrębie jednej macierzy.}%
	{Nie}{Dla takich samych dysków RAID 6 oferuje większą szybkość zapisu niż RAID 0.}
	\item \question{RAID5 może składać się z następującej ilości dysków:}%
	{Nie}{2}%
	{Tak}{3}%
	{Tak}{4}%
	{Tak}{5}
	\item \question{RAID inaczej zwanym lustrzanym (mirroringiem) to:}%
	{Tak}{RAID1}%
	{Nie}{RAID2}%
	{Nie}{RAID3}%
	{Nie}{RAID5}
	\item \question{Jakie polecenie pozwoli na rozpoczęcie procedury tworzenia partycji:}%
	{Tak}{fdisk /dev/hda}%
	{Nie}{mkdir /dev/sda}%
	{Tak}{fdisk /dev/sdb}%
	{Nie}{mdadd /dev/sdb}
	\item \question{Jaka ilość dysków jest wystarczająca, aby zastosować RAID 5:}%
	{Nie}{1}%
	{Nie}{2}%
	{Tak}{3}%
	{Tak}{4}
	
	\newpage
	\item \question{Mając do dyspozycji 3 identyczne dyski twarde można stworzyć macierz RAID w konfiguracji:}%
	{Tak}{RAID 0}%
	{Tak}{RAID 5}%
	{Nie}{RAID 6}%
	{Nie}{RAID 10}
	\item \question{Trzy dyski zostały połączone w macierz RAID 0.}%
	{Nie}{Łączna przestrzeń dyskowa jest równa sumie przestrzeni, każdego z dysków}%
	{Tak}{Łączna przestrzeń dyskowa jest równa potrojonej przestrzeni dyskowej najmniejszego dysku}%
	{Tak}{Szybkość jest równa potrojonej szybkości najwolniejszego z dysków}%
	{Nie}{Szybkość jest równa szybkości najwolniejszego z dysków}
	\item \question{Zaznacz cele zastosowania macierzy RAID:}%
	{Tak}{Zwiększenie odporności na awarie}%
	{Tak}{Zwiększenie wydajności transmisji danych}%
	{Tak}{Powiększenie przestrzeni dyskowej, dostępnej jako jedna całość}%
	{Nie}{Dwukrotne zwiększenie całkowitej przestrzeni dyskowej}
	\item \question{Administrator podłączył do komputera dwa dyski twarde o pojemności 200GB każdy i połączył je w macierz RAID 1. Do komputera nie zostały podłączone żadne inne dyski. Które z poniższych twierdzeń są prawidłowe?}%
	{Tak}{Całkowita pojemność partycji dostępnych w systemie nie przekracza 200GB.}%
	{Nie}{Rozwiązanie takie zapewnia o wiele większą prędkość odczytu i zapisu danych niż macierz RAID 0.}%
	{Tak}{Rozwiązanie takie zapewnia o wiele większe bezpieczeństwo danych niż macierz RAID 0.}%
	{Nie}{W przypadku awarii jednego dysku użytkownik straci wszystkie swoje dane}
	\item \question{Zaznacz zdania prawdziwe dotyczące sprzętowej macierzy RAID:}%
	{Tak}{Macierz jest zupełnie przezroczysta, przez co z punktu widzenia Systemu Operacyjnego zachowuje się ona jak każdy inny dysk twardy}%
	{Nie}{mniejsza wydajność poprzez zwiększenie obciążenia CPU}%
	{Tak}{Minimalna liczba dysków potrzebna do stworzenia macierzy to 2}%
	{Nie}{Sprzętowa macierz RAID zawsze umożliwia przywrócenie danych w razie awarii jednego z dysków}
	\item \question{Zaznacz zdania prawdziwe dotyczące programowej macierzy RAID:}%
	{Nie}{Macierz jest zupełnie przezroczysta, przez co z punktu widzenia Systemu Operacyjnego zachowuje się ona jak każdy inny dysk twardy}%
	{Tak}{mniejsza wydajność poprzez zwiększenie obciążenia CPU}%
	{Tak}{Minimalna liczba dysków potrzebna do stworzenia macierzy to 2}%
	{Nie}{Programowa macierz RAID zawsze umożliwia przywrócenie danych w razie awarii jednego z dysków}
	
	\newpage
	\item \question{System Linux pozwala na:}%
	{Tak}{Tworzenie programowych macierzy RAID.}%
	{Tak}{Tworzenie wolumenów liniowych.}%
	{Nie}{Tworzenie partycji za pomocą polecenia "create"}%
	{Tak}{Tworzenie macierzy RAID 5.}
	\item \question{Woluminy liniowe w katalogu dev oznaczone są jako:}%
	{Tak}{md0,md1,...}%
	{Nie}{ma0,ma1,...,mb0,mb1,...}%
	{Nie}{raid0,raid1,...}%
	{Nie}{rda0,rda1,...,rdb0,rdb1,...}
	\item \question{Za pomocą polecenia mdadm można:}%
	{Tak}{utworzyć wolumin liniowy}%
	{Nie}{Sformatować partycję}%
	{Tak}{Sprawdzić konfigurację macierzy}%
	{Tak}{Zasymulować awarię woluminu}
	\item \question{Która z aplikacji umożliwia stworzenie partycji na twardym dysku?}%
	{Nie}{/etc/fstab}%
	{Tak}{/sbin/fdisk}%
	{Tak}{/sbin/cfdisk}%
	{Nie}{/etc/mtab}
	\item \question{Wskaż poprawne zdania dotyczące RAID.}%
	{Nie}{Polecenie „mdadm -C -v /dev/md0 -{}-level=0 -n 2 /dev/sda1 /dev/sdb1” służy do stworzenia wolumenu liniowego na partycjach sda1 i sdb1.}%
	{Tak}{Polecenie „mdadm -C -v /dev/md0 -{}-level=1 -n 2 /dev/sda1 /dev/sdb1” służy do stworzenia mirroru.}%
	{Tak}{Polecenie „mkfs -t ext3 /dev/md0” służy do sformatowania urządzenia.}%
	{Nie}{Wolumenu liniowego /dev/md0 nie można dodać do pliku /etc/fstab, aby była montowana przy starcie systemu operacyjnego.}
	\item \question{Które z wymienionych rodzajów macierzy RAID zapewniają mirroring:}%
	{Nie}{RAID 0}%
	{Tak}{RAID 1}%
	{Tak}{RAID 5}%
	{Tak}{RAID 10}
	\item \question{Które z wymienionych poleceń umożliwia zarządzanie macierzami RAID w systemie GNU/Linux:}%
	{Nie}{hdparm}%
	{Tak}{mdadm}%
	{Nie}{fdisk}%
	{Nie}{parted}
	
	\newpage
	\item \question{Celem wyłączenia automatycznego montowania urządzenia cdrom w systemie Linux należy:}%
	{Tak}{Odpowiednio zmodyfikować plik '/etc/fstab'.}%
	{Nie}{Wykonać polecenie 'nmount -n cdrom'.}%
	{Nie}{Wykonać polecenie 'nmount cdrom'.}%
	{Nie}{Odpowiednio zmodyfikować plik '/etc/amount'.}
	\item \question{Polecenie 'fdisk' w systemie Linux można wykorzystać do:}%
	{Tak}{tworzenia partycji.}%
	{Tak}{wypisania informacji o dysku.}%
	{Nie}{montowania dysku.}%
	{Nie}{tworzenia kopii zapasowej danych.}
	\item \question{Wskaż poprawne odpowiedzi dotyczące RAID5:}%
	{Tak}{Umożliwia odzyskanie danych w razie awarii jednego z dysków}%
	{Nie}{Składa się z minimum 2 dysków}%
	{Nie}{Odzyskiwanie danych w razie awarii odbywa się przy wykorzystaniu danych i kodów korekcyjnych zapisanych na jednym, specjalnie do tego przeznaczonym dysku}%
	{Tak}{W przypadku awarii dysku dostęp do danych jest spowolniony}
	\item \question{Wskaż poprawne odpowiedzi dotyczące mirroring-u:}%
	{Tak}{Polega na zapisywaniu tych samych danych na dwóch lub więcej dyskach jednocześnie}%
	{Nie}{W przypadku awarii co najmniej połowy z dysków nie ma możliwości odzyskania wszystkich danych}%
	{Tak}{Dostępna przestrzeń ma rozmiar najmniejszego nośnika}%
	{Tak}{Czas równoległego zapisu jest równy czasowi zapisu na najwolniejszym dysku}
	\item \question{Wskaż poprawne zdania dotyczące RAID5 w systemie Linux:}%
	{Nie}{Do utworzenia RAID5 potrzebne są co najmniej dwie partycje.}%
	{Nie}{Do utworzenia RAID5 można użyć maksymalnie trzech partycji.}%
	{Nie}{Do odtworzenia danych z uszkodzonej partycji zawsze wykorzystywana jest jedna, specjalnie do tego przygotowanej partycja.}%
	{Tak}{RAID5 jest całkowicie odporny na uszkodzenie jednej partycji (dane można w pełni odtworzyć).}
	\item \question{Wskaż poprawne zdania dotyczące RAID1 (mirror) w systemie Linux.}%
	{Tak}{Całkowita pojemność partycji połączonych w RAID1 jest taka jak pojemność najmniejszej z tych partycji.}%
	{Tak}{Do utworzenia RAID1 można wykorzystać trzy partycje.}%
	{Nie}{Zastosowanie RAID1 pozwala na zwiększenie szybkości zapisu i odczytu danych.}%
	{Tak}{RAID1 jest całkowicie odporny na uszkodzenie jednej partycji (dane można w pełni odtworzyć).}
	\item \question{Które z poniższych funkcji macierzy RAID zwiększają bezpieczeństwo danych?}%
	{Tak}{mirroring (lustrzane odbicie)}%
	{Nie}{stripping (paskowanie)}%
	{Nie}{macierze liniowe}%
	{Tak}{kontrola parzystości}
	
	\newpage
	\item \question{Trzy dyski, każdy o pojemności 1TB, połączyliśmy w macierz RAID5. Jaką pojemnośd ma uzyskany wolumien?}%
	{Nie}{0.5 TB}%
	{Nie}{1 TB}%
	{Tak}{2 TB}%
	{Nie}{3 TB}
	\item \question{Zaznacz poprawną odpowiedz dotyczącą RAID:}%
	{Tak}{RAID pozwala łączyć ze sobą dyski celem stworzenia pamięci masowej o dużej pojemności I niezawodności}%
	{Tak}{macierz RAID można stworzyć za pomocą sprzętowych kontrolerów oraz systemowych narzędzi}%
	{Nie}{do utworzenia RAID5 wystarczą dwa dyski}%
	{Nie}{nie da stworzyć się macierzy dyskowej z dwóch dysków}
	\item \question{Skrót RAID oznacza:}%
	{Tak}{Redundant Array of Independent Disks}%
	{Nie}{Redundant Array of Independent Drives}%
	{Nie}{Remote Array of Independent Disks}%
	{Nie}{Reserved Array of Independent Disks}
	\item \question{Macierz RAID 5 charakteryzuje się}%
	{Nie}{Zastosowaniem minimum 2 dysków}%
	{Tak}{Zastosowaniem minimum 3 dysków}%
	{Nie}{Odpornością na awarię dwóch dysków}%
	{Tak}{Zmniejszoną szybkością zapisu}
	\item \question{Macierz RAID 0 używana jest do:}%
	{Tak}{Poprawy wydajności zapisu}%
	{Nie}{Zabezpieczeniem danych przed awarią dysku kosztem dostępnego miejsca}%
	{Nie}{Zabezpieczeniem danych przed awarią dysku kosztem czasu dostępu}%
	{Nie}{Skrócenia czasu odbudowy macierzy}
	\item \question{Co jest zawartością pliku /proc/mdstat ?}%
	{Tak}{Konfiguracje RAID}%
	{Tak}{Aktualny stan macierzy}%
	{Nie}{Standardowe procery obsługi RAID}%
	{Nie}{Listę uruchomionych procesów}
	\item \question{Aby połączyć dwa wolumeny w wolumen liniowy użyjemy instrukcji:}%
	{Tak}{mdadm –create –verbose /dev/md0/ $ -- $level=linear –raid-dervices=2 /dev/sdb1 /dev/sdb2}%
	{Nie}{Mdfs –create –verbose /dev/md0/ $ -- $level=linear –raid-dervices=2 /dev/sdb1 /dev/sdb2}%
	{Nie}{mdadm –create –verbose /dev/md0/ $ -- $level=raid1 –raid-dervices=2 /dev/sdb1 /dev/sdb2}%
	{Nie}{mdadm –new –verbose /dev/md0/ $ -- $level=linear –raid-dervices=2 /dev/sdb1 /dev/sdb2}
	\item \question{Zaznacz poprawne twierdzenia na temat RAID 0 :}%
	{Nie}{Zapewnia ochronę przed utratą danych}%
	{Tak}{Zapewnia zwiększoną wydajność zapisu}%
	{Tak}{Zapewnia zwiększoną wydajność odczytu}%
	{Nie}{Do jej stworzenia potrzebne są minimalnie 3 dyski}
	
	\newpage
	\item \question{Na komputerze została stworzona macierz RAID 1 złożona z 3 partycji sda1, sdb1 i sdc1, wszystkie dyski pracuja poprawnie i nie są uszkodzone, co się stanie w momencie wywołania komendy:\\'mdadm /dev/md0 -- remove /dev/sda1'}%
	{Nie}{Partycja sda1 zostanie usunięta z macierzy md0}%
	{Tak}{Nic}%
	{Tak}{Partycja sda1 zostanie usunięta z macierzy jeśli przedtem wywołano komendę 'mdadm $ -- $fail /dev/md0 /dev/sda1'}%
	{Nie}{Macierz md0 zostanie usunięta}
	\item \question{Wskaż typy macierzy dyskowych, które do ochrony danych wykorzystują sumy kontrolne}%
	{Nie}{RAID 0}%
	{Nie}{RAID 1}%
	{Tak}{RAID 3}%
	{Tak}{RAID 5}
	\item \question{Cztery dyski twarde o rozmiarach 200GB 200GB 150GB 150GB połączono w macierz typu striped volume:}%
	{Nie}{Macierz taka jest bardziej odporna na awarie niż pojedynczy dysk}%
	{Tak}{Sumaryczna szybkość takiej macierzy jest 4-krotnością szybkości najwolniejszego z dysków}%
	{Nie}{Macierz jest widziana w systemie jako pojedynczy dysk logiczny o rozmiarze 700GB}%
	{Tak}{Prawdopodobieństwo utraty danych jest większe niż dla analogicznej macierzy RAID 1}
	\item \question{Zaznacz prawdziwe zdania dotyczące RAID5.}%
	{Nie}{RAID5 polega na tworzeniu kopi danych na rożnych dyskach (mirroring)}%
	{Nie}{Macierz składa się z 5 lub więcej dysków}%
	{Nie}{Macierz składająca się z n dysków jest odporna na awarię n – 2 dysków}%
	{Tak}{Wszystkie powyższe odpowiedzi są nie poprawne}
	\item \question{W maszynie zainstalowana jest macierz RAID. Jeden z dysków podlega awarii. Zaznacz zdania prawdziwe.}%
	{Tak}{Dla macierzy RAID 5 po wymianie uszkodzonego dysku dane zostaną odbudowane.}%
	{Nie}{Macierz RAID 1 przestanie funkcjonować.}%
	{Tak}{Jeśli zainstalowane były 3 dyski, macierz RAID 1 pozwoli na dalsza pracę bez utraty danych.}%
	{Tak}{Macierz RAID 5 nie wymaga wymiany dysku na nowy przed wznowieniem pracy.}
	\item \question{Wpisanie polecenia fdisk /dev/hda oraz p spowoduje:}%
	{Nie}{sformatowanie dysku hda}%
	{Tak}{wypisanie listy partycji istniejących na dysku hda}%
	{Nie}{utworzenie na dysku hda partycji zajmującej całą dostępną przestrzeń}%
	{Nie}{uruchomienie systemu operacyjnego z dysku hda}
	
	\newpage
	\item \question{Wskaż cechy RAID 5:}%
	{Nie}{bity parzystości są zapisywane na specjalnie do tego przeznaczonym dysku}%
	{Nie}{szybkość dostępu do danych nie ulega zmianie w wypadku awarii jednego z dysków}%
	{Tak}{gwarantuje stuprocentowe bezpieczeństwo danych przy awarii jednego dysku}%
	{Tak}{jego zaletą jest szybki odczyt, jego wada to powolny zapis}
	\item \question{Co odróżnia macierze RAID programowe od sprzętowych?}%
	{Tak}{Obsługą macierzy programowych zajmuje się odpowiednie oprogramowanie, np. mdadm.}%
	{Nie}{Macierze programowe mają większą wydajność w porównaniu do sprzętowych.}%
	{Nie}{Problem awarii fizycznego nośnika w żaden sposób nie dotyczy macierzy programowych.}%
	{Nie}{W macierzach programowych problem awarii fizycznego dotyczy jedynie poziomu RAID 0.}
	\item \question{Wykonywanie jakich czynności związanych z macierzami RAID umożliwia polecenie „mdadm” w systemach z rodziny Linux?}%
	{Tak}{Podłączanie nowych urządzeń do macierzy.}%
	{Tak}{Generowanie zawartości plików konfiguracyjnych macierzy.}%
	{Tak}{Sprawdzanie statusu macierzy.}%
	{Tak}{Programowe symulowanie awarii w macierzy.}
	
	
\end{enumerate}
	
	% --- Linux LAMP ----------------------- %
	%\item \questionVIII{%
%	question={} %
%}{%
%	isTrue1=, %
%	answer1={}, %
%	isTrue2=, %
%	answer2={}, %
%	isTrue3=, %
%	answer3={}, %
%	isTrue4=, %
%	answer4={}, %
%	isTrue5=, %
%	answer5={}, %
%	isTrue6=, %
%	answer6={}, %
%	isTrue7=, %
%	answer7={}, %
%	isTrue8=, %
%	answer8={}, %
%	isTrue9=, %
%	answer9= {}, %
%	isTrue10=, %
%	answer10={},%
%	isTrue11=, %
%	answer11={}, %
%	isTrue12=, %
%	answer12={}.
%}

% !TeX spellcheck = pl_PL
\newpage
\section{Linux LAMP}
	\begin{enumerate}
		\item \questionVIII{%
				question=Zaznacz wszystkie poprawne stwierdzenia dotyczące rozwiązania LAMP: %
			}{%
				isTrue1=Nie, %
				answer1=Konfiguracja baz danych może odbywać się wyłącznie poprzez narzędzie phpMyAdmin., %
				isTrue2=Nie, %
				answer2=MySQL pozwala na wykonywanie kodu zapisanego w języku PHP na stronie wwww., %
				isTrue3=Tak, %
				answer3=Funkcją MySQL jest zarządzanie bazą danych., %
				isTrue4=Tak, %
				answer4=Podstawową funkcją serwera Apache jest przesyłanie do klienta treści plików znajdujących się na dysku przy wykorzystaniu protokołu HTTP., %
				isTrue5=Nie, %
				answer5=Kod PHP wewnątrz pliku z rozszerzeniem .html może znajdować się pomiędzy znacznikiem $ < $php$ > $ oraz znacznikiem $ < $/php$ > $., %
				isTrue6=Tak, %
				answer6=Kod PHP wewnątrz pliku z rozszerzeniem .php może znajdować się pomiędzy znacznikiem $ < $? oraz znacznikiem ?$ > $., %
				isTrue7=Tak, %
				answer7=Pliki konfiguracyjne serwera Apache znajdują się w katalogu /etc/apache2/, %
				isTrue8=Nie, %
				answer8=phpMyAdmin jest narzędziem do konfiguracji w trybie tekstowym., %
				isTrue9=Nie, %
				answer9= {Elementy LAMP to Apache, MySQL i Prolog}, %
				isTrue10=Tak, %
				answer10={Można powiedzieć, że dynamiczna strona internetowa stworzona w PHP na Linuksie, korzystająca z serwera Apache, z bazą danych MySQL jest opartą o LAMP.},%
				isTrue11=Nie, %
				answer11=Jako język programowania stron w LAMP można wykorzystać wyłącznie PHP., %
				isTrue12=Tak, %
				answer12={Elementy LAMP zostały stworzone jako osobne rozwiązania, ale razem stanowią popularną platformę systemową.},%
				isTrue13=Nie, %
				answer13={Kod w HTML wymaga kompilacji zanim zostanie umieszczony na serwerze.}, %
				isTrue14=Nie, %
				answer14={Narzędzie phpMyAdmin służy do konfiguracji serwera Apache.}, %
				isTrue15=Tak, %
				answer15={MySql może być użyty jako serwer bazy danych.}, %
				isTrue16=Tak, %
				answer16={PHP może być użyty do tworzenia stron dynamicznych.},%
				isTrue17=Nie, %
				answer17={Tylko administrator może korzystać z narzędzia phpMyAdmin.}, %
				isTrue18=Nie, %
				answer18={PostgreSQL może być użyty jako język skryptowy do tworzenia stron dynamicznych.}, %
				isTrue19=Nie, %
				answer19={Kod w PHP wymaga kompilacji zanim zostanie umieszczony na serwerze.}, %
			}
			
		\item \questionVIII{%
			question=Wskaż zdania prawdziwe dotyczące języka PHP%
		}{%
			isTrue1=Nie, %
			answer1={PHP wymaga by zmiennym nadawać typy.}, %
			isTrue2=Tak, %
			answer2={Nazwy zmiennych zaczynają się znakiem dolara.}, %
			isTrue3=Nie, %
			answer3={Jeśli kod PHP jest połączony ze znacznikami HTML, to musi się znajdować w pliku o rozszerzeniu phtml.}, %
			isTrue4=Nie, %
			answer4={Skrypt MUSI znajdować się w znacznikach $ < $?php ?$ > $ (żadnych innych)}, %
			isTrue5=Nie, %
			answer5={Funkcja mysql\_query() zwraca wynik w formie tablicy stringów.}, %
			isTrue6=Tak, %
			answer6={Skrypty PHP w typowych rozwiązaniach wykonywane są po stronie serwera.}, %
			isTrue7=Nie, %
			answer7={W pliku .php może wystąpić tylko jeden blok ograniczony znacznikami $ < $? i ?$ > $.}, %
			isTrue8=Tak, %
			answer8={PHP jest językiem interpretowanym.}, %
		}
		
		\newpage
		\item \questionVIII{%
			question=Język PHP: %
		}{%
			isTrue1=Nie, %
			answer1={Jest językiem kompilowanym}, %
			isTrue2=Tak, %
			answer2={Posiada biblioteki umożliwiające dostęp do bazy danych np. MySQL.}, %
			isTrue3=Tak, %
			answer3={Może być przeplatany z kodem HTML.}, %
			isTrue4=Nie, %
			answer4={Jest statycznie typowany.}, %
			isTrue5=Tak, %
			answer5={może być przeplatany z językiem HTML.}, %
			isTrue6=Nie, %
			answer6={wymaga deklarowania zmiennych.}, %
			isTrue7=Tak, %
			answer7={nie wymaga deklarowania zmiennych.}, %
			isTrue8=Tak, %
			answer8={wymaga, aby każda zmienna była poprzedzona znakiem \$.}, %
		}
		
		\item \question{Każda zmienna w PHP poprzedzona jest znakiem:}%
			{Nie}{\%}%
			{Nie}{\#}%
			{Nie}{$ < $?}%
			{Tak}{\$}
			
		\item \questionVIII{%
			question={W jaki sposób w języku PHP można odczytać dane (lub ich część) przesłane przez formularz na stronie internetowej (pobrać dane z formularza)?}%
		}{%
			isTrue1=Tak, %
			answer1={Używając tablicy \$\_POST}, %
			isTrue2=Nie, %
			answer2={Używając tablicy \$\_SEND\_DATA}, %
			isTrue3=Tak, %
			answer3={Używając tablicy \$\_GET}, %
			isTrue4=Tak, %
			answer4={Używając tablicy \$\_REQUEST}, %
			isTrue5=Nie, %
			answer5={Używając tablicy \$\_DATA}, %
			isTrue6=Nie, %
			answer6={Używając tablicy \$\_RESPONSE}, %
		}
			
		\item \question{Skrypty PHP:}%
			{Nie}{Są wykonywane po stronie przeglądarki internetowej klienta.}%
			{Tak}{Mogą zostać osadzone w plikach HTML.}%
			{Tak}{ZAWSZE rozpoczynają się od: $ < $?php .}%
			{Tak}{Mogą być zdefiniowane w osobnych plikach, bez osadzania w kodzie HTML.}
			
		\item \questionVIII{%
			question=Od jakich elementów systemu pochodzi określenie LAMP?
		}{%
			isTrue1=Nie, %
			answer1={Linux, Apache, McEdit, Perl}, %
			isTrue2=Tak, %
			answer2={Linux, Apache, MySQL, Perl}, %
			isTrue3=Nie, %
			answer3={Linux, Access, McEdit, PHP}, %
			isTrue4=Tak, %
			answer4={Linux, Apache, MySQL, Python}, %
			isTrue5=Nie, %
			answer5={Linux, Access, MySQL, PHP}, %
		}
		
		\item \question{W skład LAMP wchodzi:}%
			{Nie}{PostgreSQL}%
			{Tak}{Linux}%
			{Tak}{Perl}%
			{Nie}{Windows}
		
		\newpage
		\item \question{Co może oznaczać „P” w skrócie LAMP?}%
			{Nie}{PostgreSQL}%
			{Tak}{Perl}%
			{Tak}{Python}%
			{Tak}{PHP}
				
		\item \question{Do poprawnego działania LAMP pod Linuxem potrzebny jest:}%
			{Tak}{PHP}%
			{Tak}{Apache}%
			{Tak}{MySQL}%
			{Tak}{Pakiety wiążące ze sobą pozostałe składniki.}%
			
		\item \question{Jaki serwer www wchodzi w skład LAMP?}%
			{Nie}{MySQL.}%
			{Nie}{IIS}%
			{Tak}{Apache}%
			{Nie}{Zależy od konfiguracji}
			
		\item \question{Serwer Apache:}%
			{Tak}{Jest serwerem www.}%
			{Tak}{Można zainstalować osobno.}%
			{Nie}{Można zainstalować tylko razem z serwerem bazy danych MySQL oraz bibliotekami języka PHP.}%
			{Tak}{Współpracuje z interpreterem języka PHP po doinstalowaniu odpowiednich pakietów.}
			
		\item \question{Moduł userdir serwera Apache umożliwia:}%
			{Nie}{Edycję ustawień dotyczących folderów znajdujących się w pliku konfiguracyjnym serwera Apache.}%
			{Tak}{Zakładanie stron poprzez dodawanie folderu public\_html w katalogu domowym użytkownika.}%
			{Tak}{Proste dodawanie stron www użytkownikom systemu.}%
			{Nie}{Dostęp do założonych stron użytkownika poprzez adres http://localhost/?NAZWA\_UZYTKOWNIKA}
		
		\item \question{Przy prawidłowo działającym w domyślnej konfiguracji module userdir zawartość strony http://localhost/$ \sim $joe to:}%
			{Tak}{zawartość folderu public\_html w katalogu domowym użytkownika joe.}%
			{Nie}{zawartość folderu localhost na pulpicie użytkownika joe.}%
			{Nie}{zawartość folderu Joe na dysku C.}%
			{Nie}{zawartość folderu www w katalogu domowym użytkownika joe.}
			
		\newpage
		\item \question{Po włączeniu w Apache modułu userdir, pliki umieszczone przez użytkownika "jan" w folderze "/home/jan/public\_html" będą (...) :}%
			{Tak}{http://localhost/$\sim$jan}%
			{Nie}{http://localhost/home/jan/public\_html}%
			{Nie}{http://localhost/jan}%
			{Tak}{http://127.0.0.1/$\sim$jan}
			
		\item \questionVIII{%
			question={Które z podanych niżej operacji są prawidłowe, aby włączyć dowolny moduł w serwerze Apache?}%
		}{%
			isTrue1=Nie, %
			answer1={Skorzystanie z polecenia /etc/init.d/apache2 restart}, %
			isTrue2=Tak, %
			answer2={Utworzenie łącza symbolicznego w katalogu mods-enabled do pliku z katalogu mods-available.}, %
			isTrue3=Tak, %
			answer3={Skorzystanie z polecenia 'a2enmod'}, %
			isTrue4=Nie, %
			answer4={Skorzystanie z polecenia 'anenmod $ < $ nazwa\_modulu $ > $'}, %
			isTrue5=Nie, %
			answer5={Skorzystanie z polecenia /etc/init.d/apache2 force-reload} %
		}
			
		\item \question{Jakim poleceniem można zrestartować serwer Apache w celu odświeżenia konfiguracji?}
			{Tak}{/etc/init.d/apache2 restart}%
			{Tak}{/etc/init.d/apache2 stop \&\& /etc/init.d/apache2/start}%
			{Nie}{/etc/init.d/apache2 refresh}%
			{Nie}{apache2-restart}%
			
		\item \question{Konfiguracja serwera Apache w systemie Ubuntu Server:}
			{Nie}{wpływa na działający serwer zarz po zapisaniu pliku.}%
			{Tak}{w przypadku modułów serwera opiera się o dowiązania plików.}%
			{Tak}{znajduje się w folderze /etc/apache2.}%
			{Tak}{jest wstępnie przygotowana po zainstalowaniu serwera.}%
			
		\item \question{Który z modułów odpowiada za włączenie obsługi języka PHP w serwerze Apache?}%
			{Nie}{status}%
			{Nie}{proxy}%
			{Tak}{php5}%
			{Nie}{userdir}
			
		\item \question{Który z modułów pozwala dodawać strony www w Apache poprzez utworzenie katalogu public\_html w katalogu domowym?}%
			{Nie}{status}%
			{Nie}{proxy}%
			{Nie}{php5}%
			{Tak}{userdir}
			
		\item \question{Które moduły należy uruchomić, aby była możliwość dodawania stron www przez zwykłego użytkownika?}%
			{Nie}{usertrack}%
			{Nie}{proxy}%
			{Nie}{cache}%
			{Tak}{userdir}
		
		\newpage
		\item \question{Do wybrania bazy danych w MySQL w języku PHP służy funkcja:}%
			{Nie}{mysql\_db\_name}%
			{Nie}{mysql\_connect\_db}%
			{Nie}{mysql\_select\_db}%
			{Tak}{mysqli\_connect}
		
		\item \question{Jaki będzie wynik polecenia w języku skryptowym PHP:\\
			mysql\_connect("server:db", password, user) ?}%
			{Nie}{Połączenie się z bazą "db:server" na lokalnym komputerze.}%
			{Nie}{Połączenie się z bazą "server" na serwerze "db".}%
			{Nie}{Połączenie się z bazą "db" na serwerze "server".}%
			{Tak}{Zwrócenie błędu.}
		
	\end{enumerate}






	
	% --- Wielosystemowość ----------------- %
	\newpage
\section{Wielosystemowość}
\begin{enumerate}

	\item \question{Po zmianie w plikach konfiguracyjnych programu GRUB:}
	{Tak}{zmiany NIE SĄ automatycznie wprowadzone po zmianie zawartości plików}
	{Tak}{nalezy wydać polecenie update-grub jako root, aby konfiguracja nowa konfiguracja została wprowadzona}
	{Nie}{zmiany od razu nie są wprowadzone, zaraz po zmianie pliku}
	{Nie}{plików konfiguracyjnych GRUBa nie wolno edytować (jest to robione automatycznie przez system)}
	
	\item \question{Wksaż poprawne zdanie na temat dysku /dev/sdd3}
	{Nie}{Jest to czwarta partycja czwartego dysku SATA}
	{Nie}{Jest to czwarta partycja trzeciego dysku SATA}
	{Nie}{Oznaczenie nie jest poprawne}
	{Tak}{Jest to trzecia partycja czwartego dysku SATA}
	
	\item \question{Czym charakteryzuje się plik konfiguracyjny "grub.cfg" menedżera GRUB 2, znajdujący się standardowo w katalogu "/boot/grub"?}
	{Nie}{Jest to jedyny plik konfiguracji GRUB 2, którego własnoręczna edycja nie jest odradzana}
	{Tak}{Nie powinien być bezpośrednio edytowany przez użytkownika.}
	{Tak}{Może zostać nadpisany w wyniku polecenia "update-grub".}
	{Tak}{Zawiera wpisy dotyczące uruchamianych systemów operacyjnych.}
	
	\item \question{Polecenie mount -a}
	{Nie}{montuje wszystkie systemy plików wylistowane w pliku /etc/fstab}
	{Tak}{montuje systemy plików wylistowane w pliku /etc/fstab, które nie korzystają z opcji noauto}
	{Nie}{może być wykonane przez dowolnego użytkownika}
	{Tak}{zarezerwowane jest tylko dla roota}
	
	\item \question{Wskaż, które z poniższych twierdzeń odnoszących się do pliku konfiguracyjnego "/etc/fstab" są poprawne.}
	{Tak}{Definiując poszczególne systemy plików możemy posłużyć się zarówno unikalnym identyfikatorem dysku, jak i nazwą urządzenia.}
	{Nie}{Edytując plik użytkownik może wskazać jako miejsce montowania nieistniejący katalog, w trakcie uruchomienia systemu, katalog taki zostanie utworzony.}
	{Tak}{Plik ten zawiera informację na temat wszystkich systemów plików, które powinny być montowane w trakcie uruchamiania systemu.}
	{Tak}{Do edycji pliku wymagane są uprawnienia administratora.}
	
	\item \question{Używając bootloader'a GRUB2:}
	{Tak}{hd1 oznacza drugi dysk w systemie (/dev/sdb)}
	{Nie}{hd1 oznacza pierwszy dysk w systemie (/dev/sda)}
	{Tak}{setroot(hd0, 1) odwoła się do pierwszej partycji pierwszego dysku (dev/sda1)}
	{Nie}{setroot(hd0, 1) odwoła się do drugiej partycji pierwszego dysku (dev/sda2)}
	
	\item \question{Co spowoduje dodanie następującego wpisu do pliku /etc/grub.d/4-\_custom \\
		menuentry "Windows" \{ \\
		ser root='(hd0,1)' \\
		chainloader + 1 \\
		\} }
	{Nie}{Podczas startu bootloadera będziemy mogli wybrać system o nazwie "Windows" i będzie one pierwszy na liście dostępnych systemów.}
	{Nie}{Jest to niepoprawny wpis.}
	{Tak}{Podczas startu bootloadera będziemy mogli wybrać system o nazwie "Windows", znajdujący się na dysku "hd0".}
	{Tak}{W celu załadowania systemu Windows sterowanie zostanie przekazane do pierwszego sektora z podanej partycji (zostanie uruchomiony kod, który się tam znajduje).}

	\item \question{Program Grub pozwala na:}
	{Nie}{Rekompilację jądra Linux}
	{Tak}{Automatyczne uruchomienie wybranego systemu z pominięciem wyświetlania ekranu wyboru.}
	{Nie}{Zarządzanie dyskami i ich partycjonowanie}
	{Tak}{Wybór systemu operacyjnego, który będzie uruchomiony.}

	\item \question{Parametr w opcjach montowania pliku /etc/fstab oznacza, że:}
	{Nie}{możliwy jest zapis i odczyt na danym systemie plików}
	{Tak}{system plików jest zamontowany w trybie tylko do odczytu}
	{Nie}{urządzenie może być montowane przez użytkownika}
	{Nie}{system plików może być montowany przez każdego użytkownika}
	
	\item \question{W jaki sposób dodajemy informacje o innych systemach opracyjnych do GRUB2}
	{Tak}{Do pliku /etc/grub.d/40\_custom dodajemy wpis o systemie, następnie uruchamiamy polecenie sidu update-grub2}
	{Nie}{Należy wykonanać polecenie grub2-add-new-os z prawami użytkownika}
	{Tak}{Można nadać prawa wykonywania skryptowi: /etc/grub.d/30\_od-prober. Grub2 podczas aktualizacji wyszuka dostępne systemy operacyjne na dyskach twardych}
	{Nie}{GRUB2 sam wykryje wszystkie systemy operacyjne bez konfiguracji}
	
	\item \question{Plik /boot/grub.cfg dla Grand United Bootloader w wersji 2:}
	{Tak}{posiada definicje wszystkich systemów uruchamianych przez niego}
	{Tak}{w przypadku edycji za każdym razem musi być zaktualizowany za pomocą polecenia update-grub}
	{Tak}{Tworzony jest automatycznie na podstawie skryptów znajdujących się w ktalogu /etc/grub.d/}
	{Nie}{Tworzony jest automatycznie na podstawie konfiguracji zdefiniowanej w pliku /etc/grub/default}
	
	\newpage

	\item \question{Wskaż, które z poniższych twierdzeń odnoszących się do bootmanagera GRUB2 są poprawne.}
	{Nie}{Aby zablokować możliwość wykonywania się danego skryptu podczas aktualizacji GRUB'a wystarczy odebrać mu uprawnienia do odczytu.}
	{Tak}{Lista zdefiniowanych, uruchamianych przez GRUB2 systemów operacyjnych zdefiniowana jest w pliku "/boot/grub/grub.cfg".}
	{Tak}{Wywołanie polecenia "update-grub" powoduje uruchomienie skryptów umieszczonych w katalogu "/etc/grub.d"}
	{Nie}{Po wywołaniu polecenia "update-grub" skrypt "30\_os-prober" zostanie uruchomiony przed skryptem "10\_linux".}
	
	\item \question{Jakim poleceniem tworzony (bądź aktualizowany) jest plik konfiguracyjny /boot/grub.grub.cfg?}
	{Nie}{grub-config}
	{Nie}{grub-install}
	{Nie}{grub-refresh}
	{Tak}{update-grub}
	
	\item \question{Na jednym fizycznym komputerze, na osobnych partycjach są zainstalowane systemu buntu Linux i Windows 7. Przy obecnej konfiguracji użytkownik mam możliwość (przy użyciu bootmanagera GRUB 2) uruchomienia TYLKO systemu Ubuntu. W jaki sposób można zapeewić użytkownikowi możliwość wyboru systemu operacyjnego przy uruchamianu komputera?}
	{Tak}{Należy utworzyć własny plik z odpowiednim wpisem systemu oraz prawami uruchamiania w /etc/grub.d/, a następnie zaktualizować pliki konfiguracyjne GRUB'a}
	{Tak}{Dodać odpowiedni wpis w pliku /boot/grub/grub.cfg}
	{Nie}{Należy włożyć dysk instalacyjny Windowsa i z linii poleceń, za pomocą komendy bootrec /fixmbr zainstalować w MBR bootloader dla systemu Windows}
	{Tak}{Ustawić prawa uruchamiania dla skrypty /etc/grub,d/30\_os-prober oraz uruchomić update-grub}

	\item \question{Zaznacz, które z podanych plików w systemach z rodziny Linux zawierają informacje o systemach, które mają zostać automatycznie zamontowane przy uruchomieniu systemu operacyjnego.}
	{Nie}{/boot/grub/grub.cfg}
	{Nie}{/etc/default/grub}
	{Nie}{/etc/mtab}
	{Tak}{/etc/fstab}
	
	\item \question{Wskaż wszystkie poprawne odpowiedzi dotyczące bootmanagera GRUB2}
	{Tak}{Skrypty konfiguracyjne znajdujące się w katalogu /etc/grub.d/ uruchamiane są w momencie wywołania grub-update}
	{Nie}{Nie wymaga aktualizowania pliku /etc/boot/grub.cfg po wprowadzeniu zmian do pliku konfiguracyjnego /etc/default/grub - zawartość tego pliku odczytywana jest na bieżąco w momencie uruchamiania systemu.}
	{Tak}{Jest domyślnym managerem bootowania systemu Linux Ubuntu od dystybucji 9.10}
	{Tak}{Plik /boot/grub/grub.cfg jest jednym z najistotniejszych plików konfiguracyjnych managera GRUB2}
	
	\newpage
	
	\item \question{Program fdisk}
	{Tak}{Pozwala na sformatowanie wybranej partycji}
	{Tak}{Wywołany z parametrem -i wyświetla tablice partycji dla podanych urządzeń}
	{Nie}{Pozwala na obsługę tablicy partycji systemu linux}
	{Nie}{Zmiany wprowadzone za pomocą tego programu automatycznie modyfikują zawartość plików /etx/fstab i /etc/mtab}

	\item \question{Jeżeli nie chcemy, aby konfiguracja zdefiniowana w pewnym skrypcie konfiguracyjnym GRUBA znajdującym się w katalogu /etc/grub.d/ była uwzględniona po wykonaiu polecenia update-grub, należy:}
	{Nie}{Zabrać temu skryptowi uprawnienia zapisu}
	{Nie}{Wprowadzić odpowiednie zmiany w pliku /etc/default/grub}
	{Nie}{Zabrać temu skryptowi uprawnienia odczytu}
	{Tak}{Zabrać temu skryptowi uprawnienia wykonywalności}
	
	\item \question{Plik /etc/fstab zawiera informacje o:}
	{Tak}{systemach plików montowanych podczas uruchomienia systemu}
	{Nie}{aktualnie zamontowanych systemach plików}
	{Nie}{tablicach partycji na aktualnie podłączonych dyskach}
	{Nie}{mapowaniu identyfikatorów UUID na oznaczenia linuksowe (sda, sdb, itd.)}
	
	\item \question{Wskaż prawdziwe zdania:}
	{Tak}{Plik /boot.grub/grub.cfg jest generowany automatycznie na podstawie skryptów z katalogu /etc/grub.d/}
	{Tak}{Pod Windowsem możliwe jest odczytywanie partycji ext2/ext3 za pomocą dodatkowego oprogramowania}
	{Nie}{Pod Linuksem jest możliwość obsługi partycji NTFS, ale jedynie w trybie do odczytu}
	{Nie}{GRUB jest w stanie uruchamiać jedynie Linuksa i Windowsa}
	
	\item \question{Domyślnie skrypt /etc/grub.d/30\_os-prober}
	{Nie}{ustawia tło, kolory tekstu, motyw graficzny}
	{Nie}{lokalizuje jądra hurd}
	{Nie}{lokalizuje jądro Linuksa}
	{Tak}{wyszukuje w każdej partycji systemów operacyjnych i integruje je w startowym menu}

	\item \question{Plik /etc/mtab przechowuje informacje o:}
	{Nie}{Systemach plików montowanych przy starcie systemu}
	{Tak}{Aktualnie zamontowanych systemach plików}
	{Nie}{Systemach plików oczekujących na zamontowanie w systemie}
	{Nie}{Systemach plików, które z jakiś powodów nie mogły zostać zamontowane i pojawić się tym samym pliku /etc/fstab}
	
	\newpage
	
	\item \question{Dodanie systemu operacyjnego do menu GRUB'a może nastąpić w wyniku}
	{Tak}{wykonania standardowego skryptu 30\_os-prober, a następnie wykonania polecenia update-grub}
	{Tak}{stworzenia własnego skryptu w katalogu /etc/grub.d/, a następnie wykonania polecenia update-grub}
	{Nie}{dodania odpowiedniego wpisu do pliku device.map, a następnie wykonania polecenia update-grub}
	{Tak}{dodania odpowiedniego wpisu do pliku 40\_custom, a następnie wykonania polecenia update-grub}
	
	\item \question{Wskaż wszystkie poprawne zdania odnoścnie pliku device.map}
	{Tak}{Ręczna zmiana pliku device.map wymaga aktualizacji konfiguracji GRUBa}
	{Tak}{Zawiera zmapowane nazwy urządzeń GRUBa na nazwy Linuxowe}
	{Nie}{Po każdym restracie systemu zapisywana jest do niego aktualna struktura dysków.}
	{Nie}{W wersji bootloadera GRUB2 plik ten nie istnieje}
	
	\item \question{Parametr ro w opcjoach montowania pliku etc/fstab oznacza, że:}
	{Nie}{możliwy jest zapis i odczyt na danym systemie plików}
	{Tak}{system plików jest zamontowany w trybie tylko do odczytu}
	{Nie}{urządzenie może być montowane przez użytkownika}
	{Nie}{system plików może być montowany przez każdego użytkownika}
	
	\item \question{Dodajemy własny wpis do menu GRUB2. Które z poniższych wartości parametru "setroot" bloku menuentry są poprawne?}
	{Nie}{setroot = (hda,1)}
	{Tak}{setroot = (hd0, msdos1)}
	{Nie}{setroot = (sda,1)}
	{Tak}{setroot = (hd0,1)}
	
	\item \question{Informacje na temat wszystkich systemów plików, które mają być montowane podczas uruchamiania systemu znajdują się w pliku:}
	{Nie}{/mnt}
	{Tak}{/etc/fstab}
	{Nie}{/etc/default/fstab}
	{Nie}{/etc/mtab}
	
	\item \question{Plik /boot/grub/grub.cfg zawiera:}
	{Nie}{tryb, w jakim ma się ładować system.}
	{Tak}{liste systemów operacyjnych, które można uruchomić za pomocą GRUBa}
	{Tak}{informację o tym, który sytem jest systemem domyślnym.}
	{Tak}{czas oczekiwania na wybór systemu przez użytkownika, po upływie którego uruchomi się domyślny system.}
	
	\newpage
	
	\item \question{Jakie informacje na temat zamontowanych systemów plików znajdują się w /etc/fstab?}
	{Nie}{Data zamontowania urządzenia.}
	{Tak}{Miejsce zamontowania systemu plików}
	{Tak}{Typ systemu plików.}
	{Nie}{Wielkość partycji.}

	\item \question{Zaznacz zdania poprawne dotyczące odwoływania się do systemów plików w systemie Linux.}
	{Tak}{/dev/fd0 - oznacza dyskietkę/}
	{Tak}{/dev/hdd2 - oznacza drugą partycję znajdującą się na dysku "slave" podpiętego do drugiego kontrolera IDE.}
	{Nie}{/dev/sda1 - oznacza pierwszą partycję pierwszego dysku SCSII lub drugą partycję na kontrolerze SATA1.}
	{Nie}{/dev/ssd1 - oznacza pierwszą partycję dysku stworzonego w oparciu o technologię SSD}
	
	\item \question{Plik /etc/fstab:}
	{Tak}{może być modyfikowany przez administartora systemu}
	{Nie}{zawiera informacje o aktualnie zalogowanych użytkownikach}
	{Tak}{Jest odczytywany w trakcie uruchamiania systemu operacyjnego}
	{Tak}{zawiera informacje o systemach plików, jakie mają być montowane w trakcie uruchamiania systemu.}
		
\end{enumerate}
	
	% --- Linux Kernel --------------------- %
	
\newpage
\section{Linux Kernel}

\begin{itemize}
	
	
	\item \questionVIII{%
		question={Zaznacz wszystkie poprawne odpowiedzi:} %
	}{%
		isTrue1=Nie, %
		answer1={Jądro Linuxa jest mikrojądrem (microkernel)}, %
		isTrue2=Nie, %
		answer2={Jądro Linuxa jest jądrem typu hybrydowego (hybrid)}, %
		isTrue3=Tak, %
		answer3={Jądro Linuxa jest jądrem typu monolitycznego (monolythic)}, %
		isTrue4=Nie, %
		answer4={Jądro Linuxa jest napisane w C++}, %
		isTrue5=Nie, %
		answer5={Jądro Linuxa wykorzystuje bibliotekę libc (dzięki temu można wykorzystywać np. funkcję printf()}, %
		isTrue6=Tak, %
		answer6={Jądro Linuxa jest napisane w C},
		isTrue7=Tak,
		answer7={Jądtro Linuxa zarządza pamięcią operacyjną (przydziały/zwolnienia)}.
	}
	
	\item \questionVIII{%
		question={Zaznacza wszystkie poprawne odpowiedzi:} %
	}{%
		isTrue1=Nie, %
		answer1={Do sterowania paramterami pracy jądra można wykorzystać pliki znajdujące się w katalogu \textbf{/var}}, %
		isTrue2=Tak, %
		answer2={Do sterowania pracą jądra Linuxa można wykorzystać polecenie \textbf{sysctl}}, %
		isTrue3=Tak, %
		answer3={Do jądra systemu operacyjnego Linux można, w czasie jego pracy, dołączać różnorodną funkjconalność (np. sterowniki urządzenia)}, %
		isTrue4=Nie, %
		answer4={Do załadowania modułu w jądrze można wykorzystać polecenia rmmod oraz modprobe -r}, %
		isTrue5=Tak, %
		answer5={Do sterowania paramterami pracy jądra można wykorzystać pliki znajdujące się w katalogu \textbf{/proc}}, %
		isTrue6=Nie, %
		answer6={Do sterowania pracą jądra Linuxa można wykorzystać polecenie \textbf{sysinfo}}, %
		isTrue7=Nie, %
		answer7={Do usunięcia modułu z jądra można wykorzystać polecenie insmod}, %
		isTrue8=Tak, %
		answer8={Do sprawdzenia jakie moduły załadowane są do jądra można wykorzystać polecenie lsmod}, %
		isTrue9=Tak, %
		answer9={Do załadowania modułu w jądrze można wykorzystać polecenie modprobe oraz insmod}, %
		isTrue10=Tak, %
		answer10={Katalog \textbf{/proc} zawiera plki, pozwalające na zmianę sposobu przydzielania pamięci programom przez jądro Linux},%
		isTrue11=Nie, %
		answer11={Katalog \textbf{/var} zawiera pliki, pozwalające na zmianę sposobu przydzielania pamięci programom przez jądro systemu Linux}, %
		isTrue12=Tak, %
		answer12={Do usunięcia modułu z jądra można wykorzystać polecenia modprobe oraz mmod},
		isTrue13=Tak, %
		answer13={Katalogi /proc, /sys oraz polecenie sysctl pozwalają na kontrolę pracy systemu},
		isTrue14=Tak,
		answer14={Z jądra systemu operacyjnego Linux, w trakcie jego pracy, można usuwać różnorodną funkcjonalność (na przykład sterowniki urządzenia)}.
	}
	
	
	\item \question{Zaznacz wszystkie funkcje realizowane przez jądro monolityczne (na przykład jądro Linuxa)}
	{Tak}{Szeregowanie procesów}
	{Tak}{Zarządzanie pamięcią (zwalnianie/przydzielanie)}
	{Tak}{Szeregowanie I/O}
	{Tak}{Obsługa systemu plików}
	
	\item \question{Jakie operację można wykonać za pomocą polecenia sysctl?}
	{Tak}{Ustawić wartości dla paramterów jądra}
	{Nie}{Ustawić wartości dla parametrów systemu plików}
	{Tak}{Wypisać wszystkie parametry jądra w trakcie działania systemu}
	{Nie}{Wypisać wszystkie parametry systemu plików}
	
	\newpage
	
		\item \questionVIII{%
			question={Polecenie sysctl:} %
		}{%
		isTrue1=Nie, %
		answer1={Służy do zmiany hasła użytkownika systemu}, %
		isTrue2=Nie, %
		answer2={Umożliwia zmianę nazwy użytkownika}, %
		isTrue3=Nie, %
		answer3={Wyświetla listę użytkowników w systemie}, %
		isTrue4=Tak, %
		answer4={Pozwala na zmianę parametrów jądra systemu w trakcie działania systemu operacyjnego}, %
		isTrue5=Tak, %
		answer5={To komenda pozwalająca na konfiguracje parametró jądra systemu Linux.}, %
		isTrue6=Tak, %
		answer6={Wykonuje konfigurację jaką można także wykonać w weirtualnym systemie plików /proc/sys.}, %
		isTrue7=Nie, %
		answer7={Pozwala na rekompilację jądra z uwzględnieniem nowych plików konfiguracyjnych.}, %
		isTrue8=Nie, %
		answer8={Wyświetla wszystkie procesy w systemie.}.
	}
	
	\item \question{Wskaż prawdziwe zdania:}
	{Nie}{przy overcommit\_memory ustawionym na 2 system zawsze pzydzieli aplikacjom dokładnie 100\% pamięci RAM}
	{Tak}{przy overcommit\_memory ustawionym na 1 możliwe jest uzyskanie za pomocą malloc() ilości pamięci wirtualnej większej niż objętość pamięci fizycznej + swap}
	{Tak}{przy overcommit\_memory ustawionym na 2 ilość pamięci przydzielonej aplikacjom zależy od ovecommit\_ratio}
	{Nie}{kernel nigdy nie przydziela więcej pamięci niż jest dostępne fizycznie}
	
	\item \question{Sterowanie jądrem systemu Linux. Zaznacz poprawne odpowiedzi:}
	{Nie}{Nawet najdrobniejsza zmiana w pracy jądra systemu wymaga jego ponownej kompilacji.}
	{Tak}{Możliwa jest zmiana niektórych parametrów jądra w "locie" korzystając z komendy sysctl.}
	{Nie}{Po każdej zmianie parametru w jądrze systemu Linux należy ponownie uruchomić komputer.}
	{Tak}{Wartości sysctl wczytywane są podczas startu systemu z pliku /etc/sysct.conf.}
	
	\item \question{Sterowniki w systemach Linuxowych: Wskaż poprawne odpowiedzi.}
	{Tak}{Można wkompilować w jądro, ale można ładować dynamicznie bez potrzeby wkompilowywania.}
	{Tak}{Mogą być ładowane dynamicznie w trakcie działania systemu.}
	{Nie}{Są tylko wkompilowane w jądro i uruchamiane automatycznie. Nie ma innej możliwości instalacji i uruchomienia.}
	{Nie}{Po instalacji nowego sterownika zawsze wymagane jest ponowne uruchomienie komputera.}
	
	\item \question{W jaki sposóbm ożna wyłączyć partycję SWAP?}
	{Nie}{Nie można wyłączyć partycji SWAP}
	{Nie}{sudo setswap off}
	{Tak}{sudo swapoff -a}
	{Nie}{sudo swap stop}
	
	\item \question{Jakie jest zadanie jądra w systemie Linux?}
	{Tak}{Ładuje i odładowuje sterowniki urządzeń.}
	{Nie}{Tylko i wyłącznie zarządza pamięcią.}
	{Tak}{Pośredniczy pomiędzy aplikacją użytkownika a sprzętem.}
	{Tak}{Zarządza pamięcią.}
	
	\item \question{Jądro w systemie Linux odpowiedzialne jest za:}
	{Tak}{Sterowniki urządzeń}
	{Nie}{Wygląd interfejsu graficznego}
	{Tak}{Zarządzanie procesami}
	{Tak}{Obsługę pamięci}
	
	\item \question{Moduły jądra systemu Linux: wskaż wszystkie poprawne odpowiedzi.}
	{Tak}{Można pisać w języku C}
	{Nie}{Mogą być załadowane przez każdego użytkownika}
	{Tak}{Nie posiadają możliwości wyprowadzania danych na standardowe wyjście stdout za pomocą printf}
	{Tak}{Można je kompilować na tym samym systemie na którym zamierzamy je uruchomić.}
	
	\item \question{Co znajduje się w katalogu /proc/?}
	{Tak}{Informacje o procesach w systemie}
	{Nie}{Informacje o użytkownikach}
	{Tak}{Informacje o sieci}
	{Tak}{Ogólne onformacje o systemie}
	
	\item \question{Program modprobe:}
	{Nie}{wymaga restartu aby zmiany zostały wprowadzone}
	{Tak}{umożliwia usuwanie modułów z kernela}
	{Tak}{umożliwia ładowanie modułów kernela}
	{Tak}{automatycznie dodaje moduły zależne}
	
	\item \question{Parametry jądra systemu Linux można odczytać za pomocą:}
	{Nie}{pliku /proc/stat}
	{Tak}{katalogu /proc/sys}
	{Nie}{komendy ps}
	{Tak}{komendy sysctl}
	
	\item \question{Które z poniższych komend sprawdza logi jądra systemu Linux}
	{Tak}{dmesg}
	{Nie}{klog}
	{Nie}{kmllg}
	{Nie}{kernelog}
	
\end{itemize}


\end{document}