% !TeX spellcheck = pl_PL
\documentclass[a4paper,twoside]{article}
\usepackage{polski}
\usepackage[utf8]{inputenc}
\usepackage{graphicx}
\usepackage{amsmath}

\usepackage[unicode, bookmarks=true]{hyperref} %do zakładek
\usepackage{tabto} % do tabulacji
\NumTabs{6} % globalne ustawienie wielkosci tabulacji
\usepackage{array}
\usepackage{multirow}
\usepackage{array}
\usepackage{dcolumn}
\usepackage{bigstrut}
\usepackage{color}
\usepackage[usenames,dvipsnames]{xcolor}
\usepackage{svg}
\usepackage{xfrac}
\usepackage{floatrow}
% Table float box with bottom caption, box width adjusted to content
\newfloatcommand{capbtabbox}{table}[][\FBwidth]
\usepackage{blindtext}
\usepackage{enumerate}
\usepackage{wrapfig}


\setlength{\textheight}{24cm}
\setlength{\textwidth}{15.92cm}
\setlength{\footskip}{10mm}
\setlength{\oddsidemargin}{0mm}
\setlength{\evensidemargin}{0mm}
\setlength{\topmargin}{0mm}
\setlength{\headsep}{5mm}

\newcolumntype{M}[1]{>{\centering\arraybackslash}m{#1}}
\newcolumntype{N}{@{}m{0pt}@{}}

% === Reset inkrementacji sekcji przy nowym parcie === %
\usepackage{titlesec}

\makeatletter
\@addtoreset{section}{part}
\makeatother
\titleformat{\part}[display]
{\normalfont\LARGE\bfseries\centering}{}{0pt}{}

% === DEFINICJA ZIELONEGO ==================== %
\definecolor{Gurin}{rgb}{0, 0.35, 0}

% === MAKRODEFINICJA POPRAWNEJ I ZŁEJ ODPOWIEDZI ==================== %
\newcommand{\Tak}[1] {
	\color{Gurin}{#1}
}
\newcommand{\Nie}[1] {
	\color{Red}{#1}
}

% === MAKRODEFINICJA PYTANIA I ODPOWIEDZI =========================== %
% ** Pierwszy argument to treść pytania
% ** Kolejne argumenty to:
% * parzyste - Tak lub Nie
% * nieparzyste - Treści odpowiedzi
\newcommand{\question}[9] {
	\textbf{#1}
	\begin{enumerate}[a.]
		\ifnum\pdfstrcmp{#2}{Tak}=0
			\Tak{\item #3}
		\else
			\Nie{\item #3}
		\fi
		\color{black}
		\ifnum\pdfstrcmp{#4}{Tak}=0
			\Tak{\item #5}
		\else
			\Nie{\item #5}
		\fi
		\color{black}
		\ifnum\pdfstrcmp{#6}{Tak}=0
			\Tak{\item #7}
		\else
			\Nie{\item #7}
		\fi
		\color{black}
		\ifnum\pdfstrcmp{#8}{Tak}=0
			\Tak{\item #9}
		\else
			\Nie{\item #9}
		\fi
	\end{enumerate}
}
\begin{document}
\bibliographystyle{plain}

% ************************************************************
% *** WZÓR PYTANIA DO SKOPIOWANIA

%\item \question{}%
%{Tak}{}%
%{Nie}{}%
%{}{}%
%{}{}

% ************************************************************

\begin{titlepage}
\title{\huge Dedykowane Systemy Operacyjne - zbiór pytań}
\author{\large zebrał SonMati}
\maketitle
\end{titlepage}

% === PYTANIA ===
\part{Windows}
	\section{Usługi katalogowe (Active Directory)}
	\begin{enumerate}
		\item \question{Wykonanie kwerendy do wyszukiwania wyłączonych kont użytkowników jest możliwe za pomocą:}%
		{Nie}{polecenia ds-query}%
		{Tak}{polecenia dsquery}%
		{Tak}{konsoli Active Directory Users and Computers}%
		{Nie}{polecenia ds-get}
		\item \question{Role FSMO można:}%
		{Tak}{Przejmować}%
		{Nie}{Filtrować}%
		{Nie}{Nadpisywać}%
		{Tak}{Transferować}
		\item \question{W jednostkach organizacyjnych (Organization Unit) można utworzyć:}%
		{Tak}{Użytkownika}%
		{Tak}{Grupę}%
		{Tak}{Komputer}%
		{Tak}{Drukarkę}
		\item \question{Za pomocą jakiego polecenia można dodać obiekt określonego typu (korzystając z wiersza poleceń) w Active Directory}%
		{Nie}{dscreate user}%
		{Tak}{dsadd user}%
		{Tak}{dsadd computer}%
		{Nie}{dscreate computer}
		\item \question{Wybierz prawidłowe odpowiedzi dotyczące struktury Active Directory:}%
		{Tak}{Jeśli domeny wchodzące w skład lasu mają nieciągłe nazwy DNS, tworzą kilka odrębnych drzew w obrębie lasu}%
		{Tak}{Drzewo posiada zawsze przynajmniej jedną domenę - domenę najwyższego poziomu (ang. root) - korzeń drzewa}%
		{Nie}{Drzewo domen - domeny potomne mogą, ale nie muszą zawierać nazwy bezpośredniej domeny nadrzędnej}%
		{Nie}{Las jest zestawem przynajmniej dwóch lub więcej drzew, które formują zwartą, ciągłą przestrzeń nazw.}
		\newpage
		\item \question{Do czego służy polecenie dsget?}%
		{Nie}{Wyświetla różne właściwości grupy, włącznie z członkami grupy w katalogu}%
		{Nie}{Wyświetla właściwości komputera w katalogu.}%
		{Tak}{Umożliwia dodawanie użytkowników, grup, komputerów, kontaktów i jednostek organizacyjnych do usługi Active Directory.}%
		{Nie}{Umożliwia tworzenie, modyfikowanie i usuwanie obiektów katalogu.}
		\item \question{Jakie obiekty można dodawać za pomocą polecenia dsadd?}%
		{Tak}{Grupy}%
		{Tak}{Użytkowników}%
		{Tak}{Jednostki organizacyjne}%
		{Nie}{Pliki}
		\item \question{Wskaż zdania prawdziwe dotyczące usługi Active Directory:}%
		{Tak}{Struktura Active Directory ma strukturę drzewiastą.}%
		{Tak}{Liśćmi drzewa mogą być użytkownicy, grupy i }%
		{Nie}{Nie ma możliwości nadania określonemu użytkownikowi praw do zarządzania użytkownikami w jednostce organizacyjnej bez nadawania (...)}%
		{Tak}{Możliwe jest kopiowanie użytkowników.}
		\item \question{Program dsget.exe:}%
		{Tak}{wymaga praw administratora do działania}%
		{Nie}{może być uruchomione w graficznym interfejsie użytkownika za pomocą odpowiedniej opcji linii poleceń.}%
		{Tak}{ma tekstowy interfejs użytkownika.}%
		{Nie}{umożliwia tworzenie jednostek organizacyjnych.}
		\item \question{Jakie możliwości daje użycie polecenia dsadd (jako polecenie dla Active Directory)? }%
		{Nie}{Nie może dodawać obiektu typu Computer do katalogu}%
		{Tak}{Nie może dodać obiektu typu Doman Service do katalogu.}%
		{Tak}{Może dodawać obiekt typu Computer do katalogu.}%
		{Nie}{Może dodać obiekt typu Domain Service do katalogu.}
		\item \question{Różnica pomiędzy zaufaniem do domen forest i external polega na:}%
		{Nie}{Zaufanie typu forest pozwala na korzystanie z zasobów tylko w obrębie danego drzewa, zaś external we wszystkich drzewach.}%
		{Tak}{Zaufanie typu external pozwala większej liczby domen na korzystanie ze swoich zasobów niż typ forest.}%
		{Tak}{Zaufanie typu forest pozwala na korzystanie z zasobów tylko w obrębie drzew połączonych tego typu zaufaniem, zaś external nie musi (...)}%
		{Nie}{Zaufanie typu external jest dwukierunkowe, typu forest tylko jednokierunkowe.}
		\newpage
		\item \question{Polecenie dsquery:}%
		{Nie}{Zgodnie z określonymi kryteriami wykonuje kwerendę dotyczącą drzewa usługi DNS.}%
		{Nie}{Jest równoważne poleceniu dsget.}%
		{Nie}{Wykonuje kwerendę na dowolnej bazie danych.}%
		{Tak}{Zgodnie z określonymi kryteriami wykonuje kwerendę dotyczącą usługi Active Directory.}
		\item \question{Dodać grupę można za pomocą:}%
		{Tak}{Konsoli "Active directory users and computers".}%
		{Nie}{Polecenia Addgroup.}%
		{Tak}{Polecenia dsadd group.}%
		{Nie}{Polecenia adadd group.}
		\item \question{Jakie występują typy zaufania w Active Directory:}%
		{Tak}{Lasu (forest)}%
		{Nie}{Wewnętrzne (internal)}%
		{Tak}{Zewnętrzne (external)}%
		{Nie}{Płatka śniegu (snowflake)}
		\item \question{Struktura Active Directory:}%
		{Tak}{Podstawową jednostką jest tzw. liść, który położony jest w kontenerze w Active Directory nazywanym jednostką centralną.}%
		{Tak}{Liście i kontenery zorganizowane są w domeny.}%
		{Nie}{Domeny zorganizowane w drzewo reprezentowane są w rożnych przestrzeniach adresowych DNS.}%
		{Nie}{Domena może istnieć samodzielnie, nie musi istnieć w jakimś drzewie i jakimś lesie}
		\item \question{Polecenie dsadd może posłużyć do:}%
		{Nie}{Modyfikowania obiektów wewnątrz AD.}%
		{Tak}{Dodawania grup do AD.}%
		{Tak}{Dodawania użytkowników do AD.}%
		{Nie}{Wyszukiwania informacji o obiektach AD.}
		\item \question{Za pomoca konsoli Active Directory Users and Computers wykonano polecenie "dsadd Ala -pwd makota". Wskaż poprawne odpowiedzi.}%
		{Nie}{Jeżeli użytkownik "Ala" nie istnieje w systemie wykonanie polecenia nie powiedzie się.}%
		{Tak}{Powodzenie operacji jest zależne od poziomu uprawnień wykonującego go użytkownika.}%
		{Tak}{W przypadku powodzenia operacji zostanie utworzony nowy użytkownik o nazwie "Ala" i haśle "makota".}%
		{Nie}{W przypadku powodzenia operacji hasło istniejącego użytkownika "Ala" zostanie zmienione na "makota".}
		\item \question{Za pomoca polecenia dsadd można:}%
		{Tak}{Utworzyć grupę zabezpieczeń.}%
		{Tak}{Utworzyć jednostkę organizacyjną.}%
		{Tak}{Utworzyć konto użytkownika.}%
		{Tak}{Utworzyć grupę dystrybucyjną.}
		\item \question{Czym jest Active Directory?}%
		{Nie}{Narzędzie certyfikujące w Windows Server 2003.}%
		{Tak}{Usługą w systemie Windows, która udostępnia środki pozwalające zarządzać tożsamościami i relacjami.}%
		{Tak}{Jest to zbiór funkcji, która upraszczają zarządzanie użytkownikami i komputerem.}%
		{Nie}{Nakładka na system Linux pozwalająca zarządzać komputerem.}
		\item \question{Poznane w czasach laboratorium narzędzia pozwalające na zarządzanie Active Directory to:}%
		{Nie}{netsh}%
		{Tak}{dsget}%
		{Tak}{dsadd}%
		{Tak}{Przystawka Active Directory Users and Computers}
		\item \question{Ile kontrolerów może działać dla domeny w usłudze Active Directory?}%
		{Tak}{więcej niż 2}%
		{Tak}{2}%
		{Tak}{1}%
		{Nie}{domena może nie posiadać komputera}
		\item \question{Które z typów relacji zaufania są dwukierunkowe?}%
		{Tak}{Lasu (forest)}%
		{Tak}{Zewnętrzne (external)}%
		{Tak}{Obszaru (realm)}%
		{Tak}{Skrótu (shortcut)}
		\item \question{Aby uruchomić na serwerze usługę Active Directory należy:}%
		{Nie}{Utworzyć użytkownika do zarządzania usługą.}%
		{Tak}{Zainstalować serwer DNS.}%
		{Tak}{Dodać rolę Active Directory Domain Services.}%
		{Nie}{Zainstalować serwer DHCP.}
		\item \question{Co to jest jednostka organizacyjna (Organization Unit, OU)?}%
		{Nie}{Pojedynczy komputer wchodzący w skład domeny Active Directory.}%
		{Nie}{Serwer pracujący pod kontrolą systemu Windows 2008 Server.}%
		{Tak}{Kontener wykorzystywany do grupowania obiektów wewnątrz domeny w logiczne grupy, na których wykonywane są zadania administracyjne (...)}%
		{Nie}{Serwer służący do zarządzania protokołem ldap}
		\item \question{Jakie obiekty mogą być modyfikowane za pomocą usługi Active Directory Sites and Services?}%
		{Tak}{Łącza lokacji i lokacje.}%
		{Nie}{Tylko serwery i lokacje.}%
		{Nie}{Serwery, lokacje, replikacje.}%
		{Tak}{Serwery, podsieci, ustawienia usługi katalogowej (NTDS)}
		\item \question{Wskaż zdania prawdziwe:}%
		{Nie}{Usługa Active Directory dostępna jest w systemach Windows od wersji Vista.}%
		{Tak}{Za pomocą polecenia dsadd user możliwe jest dodanie użytkownika do Active Directory.}%
		{Tak}{Za pomocą polecenia dsadd group możliwe jest dodanie grupy do Active Directory.}%
		{Tak}{Szablon konta użytkownika to konto zawierające wzorcowy zestaw najczęściej wykorzystywanych właściwości, odpowiednich dla danej grupy użytkowników.}
		\item \question{Które polecenie umożliwia usuwanie obiektów określonego typu?}%
		{Nie}{dsmod}%
		{Nie}{csvde}%
		{Nie}{dsadd}%
		{Tak}{dsrm}
		\item \question{W Windows Server 2008 - Active Directory administrator edytując właściwości obiektu klasy Użytkownik, w zakładce Konto ma możliwość:}%
		{Nie}{Wyznaczenia daty i godziny, w której konto Użytkownika zostanie usunięte z systemu.}%
		{Tak}{Zablokowania Użytkownikowi możliwości samodzielnej zmiany hasła.}%
		{Tak}{Ustalenia godzin w których Użytkownik będzie mógł lub nie będzie mógł zalogować się do systemu.}%
		{Tak}{Określenia do jakich komputerów znajdujących się w domenie Użytkownik ma prawo do zalogowania się.}
		\item \question{Za pomocą polecenia dsadd w systemie Windows można:}%
		{Nie}{Zmodyfikować konto użytkownika}%
		{Tak}{Utworzyć jednostkę organizacyjną}%
		{Tak}{Utworzyć konto użytkownika}%
		{Nie}{Przenieść jednostkę organizacyjną z jednej domeny do drugiej}
		\item \question{Wskaż najważniejsze funkcje Active directory:}%
		{Tak}{Obsługa LDAP}%
		{Tak}{Obsługa DNS}%
		{Tak}{Obsługa zabezpieczeń}%
		{Tak}{Zgodność z TCP/IP}
		\item \question{W jednostce organizacyjnej można umieszczać:}%
		{Tak}{Komputery}%
		{Tak}{Użytkowników}%
		{Tak}{Inne jednostki organizacyjne}%
		{Nie}{Obiekty z innych domen niż domena jednostki organizacyjnej}
		\item \question{Wskaż poprawne stwierdzenia na temat uprawnień w Active Directory:}%
		{Nie}{Uprawnienia nie mogą być dziedziczone}%
		{Tak}{Istnieje możliwość nadania uprawnień administratora na danym komputerze bez posiadania uprawnień administratora w domenie}%
		{Nie}{Aby skopiować użytkownika, jego konto musi być uprzednio wyłączone}%
		{Tak}{Użytkownik może należeć do wielu grup}
		\item \question{Jakie protokoły są wykorzystywane przez Active Directory?}%
		{Nie}{SSH}%
		{Tak}{LDAP}%
		{Tak}{Kerberos}%
		{Tak}{DNS}
		\item \question{Do czego służy polecenie dsadd?}%
		{Nie}{Dodaje możliwość zdalnego dostępu do elementu serwisu Active Directory}%
		{Nie}{Dodaje atrybut do elementu serwisu Active Directory}%
		{Nie}{Dodaje połączenie do zarządzania elementem serwisu Active Directory}%
		{Tak}{Dodanie określony typ elementu do serwisu Active Directory}
		\item \question{Wskaż prawdziwe zdania dotyczące jednostek organizacyjnych (Organizational Units):}%
		{Tak}{Mogą tworzyć strukturę hierarchiczną}%
		{Nie}{Grupują wyłącznie użytkowników}%
		{Tak}{Mogą być wykorzystane do delegowania uprawnień administracyjnych}%
		{Nie}{Nie mogą być w sobie zagnieżdżane}
		\item \question{Za pomocą konsoli Active Directory Users and Computers wykonano polecenie: „dsmod group Alfa -addmbr Beta”. Wskaż wszystkie poprawne odpowiedzi.:}%
		{Nie}{W przypadku powodzenia operacji do grupy Beta zostanie dodany obiekt Alfa.}%
		{Tak}{W przypadku powodzenia operacji do grupy Alfa zostanie dodany obiekt Beta.}%
		{Tak}{Powodzenie operacji jest zależne od poziomu uprawnień wykonującego go użytkownika.}%
		{Tak}{Obiekt o nazwie Beta może reprezentować użytkownika.}
		
		
	\end{enumerate}
	
	\section{Obiekty Zasad Grup (GPO)}
	
	\section{Windows Instalacja zdalna}
	
	\section{Windows RAID}
	
	\section{Interpreter poleceń PowerShell}
	
	\section{Windows API}
	
\part{Linux}
	\section{Usługi graficzne Xwindow}
	
	\section{Linux ACL}
	
	\section{Linux RAID}
	
	\section{Linux LAMP}
	
	\section{Wielosystemowość}
	
	\section{Linux Kernel}


\end{document}