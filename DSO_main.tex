% !TeX spellcheck = pl_PL
\documentclass[a4paper,twoside]{article}
\usepackage{polski}
\usepackage[utf8]{inputenc}
\usepackage{graphicx}
\usepackage{amsmath}

\usepackage[unicode, bookmarks=true]{hyperref} %do zakładek
\usepackage{tabto} % do tabulacji
\NumTabs{6} % globalne ustawienie wielkosci tabulacji
\usepackage{array}
\usepackage{multirow}
\usepackage{array}
\usepackage{dcolumn}
\usepackage{bigstrut}
\usepackage{color}
\usepackage[usenames,dvipsnames]{xcolor}
\usepackage{svg}
\usepackage{xfrac}
\usepackage{floatrow}
% Table float box with bottom caption, box width adjusted to content
\newfloatcommand{capbtabbox}{table}[][\FBwidth]
\usepackage{blindtext}
\usepackage{enumerate}
\usepackage{wrapfig}


\setlength{\textheight}{24cm}
\setlength{\textwidth}{15.92cm}
\setlength{\footskip}{10mm}
\setlength{\oddsidemargin}{0mm}
\setlength{\evensidemargin}{0mm}
\setlength{\topmargin}{0mm}
\setlength{\headsep}{5mm}

\newcolumntype{M}[1]{>{\centering\arraybackslash}m{#1}}
\newcolumntype{N}{@{}m{0pt}@{}}

% === Reset inkrementacji sekcji przy nowym parcie === %
\usepackage{titlesec}

\makeatletter
\@addtoreset{section}{part}
\makeatother
\titleformat{\part}[display]
{\normalfont\LARGE\bfseries\centering}{}{0pt}{}

% === DEFINICJA ZIELONEGO ==================== %
\definecolor{Gurin}{rgb}{0, 0.35, 0}

% === MAKRODEFINICJA POPRAWNEJ I ZŁEJ ODPOWIEDZI ==================== %
\newcommand{\Tak}[1] {
	\color{Gurin}{#1}
}
\newcommand{\Nie}[1] {
	\color{Red}{#1}
}

% === MAKRODEFINICJA PYTANIA I ODPOWIEDZI =========================== %
% ** Pierwszy argument to treść pytania
% ** Kolejne argumenty to treści odpowiedzi
\newcommand{\question}[9] {
	\textbf{#1}
	\begin{enumerate}[a.]
		\ifnum\pdfstrcmp{#2}{Tak}=0
			\Tak{\item #3}
		\else
			\Nie{\item #3}
		\fi
		\color{black}
		\ifnum\pdfstrcmp{#4}{Tak}=0
			\Tak{\item #5}
		\else
			\Nie{\item #5}
		\fi
		\color{black}
		\ifnum\pdfstrcmp{#6}{Tak}=0
			\Tak{\item #7}
		\else
			\Nie{\item #7}
		\fi
		\color{black}
		\ifnum\pdfstrcmp{#8}{Tak}=0
			\Tak{\item #9}
		\else
			\Nie{\item #9}
		\fi
	\end{enumerate}
}
\begin{document}
\bibliographystyle{plain}

% ************************************************************
% *** WZÓR PYTANIA DO SKOPIOWANIA

%\item \question{}%
%{Tak}{}%
%{Nie}{}%
%{}{}%
%{}{}

% ************************************************************

\begin{titlepage}
\title{\huge Dedykowane Systemy Operacyjne - zbiór pytań}
\author{\large zebrał SonMati}
\maketitle
\end{titlepage}

% === PYTANIA ===
\part{Windows}
	\section{Usługi katalogowe (Active Directory)}
	\begin{enumerate}
		\item \question{Wykonanie kwerendy do wyszukiwania wyłączonych kont użytkowników jest możliwe za pomocą:}%
		{Nie}{polecenia ds-query}%
		{Tak}{polecenia dsquery}%
		{Tak}{konsoli Active Directory Users and Computers}%
		{Nie}{polecenia ds-get}
		\item \question{Role FSMO można:}%
		{Tak}{Przejmować}%
		{Nie}{Filtrować}%
		{Nie}{Nadpisywać}%
		{Tak}{Transferować}
		\item \question{W jednostkach organizacyjnych (Organization Unit) można utworzyć:}%
		{Tak}{Użytkownika}%
		{Tak}{Grupę}%
		{Tak}{Komputer}%
		{Tak}{Drukarkę}
		\item \question{Za pomocą jakiego polecenia można dodać obiekt określonego typu (korzystając z wiersza poleceń) w Active Directory}%
		{Nie}{dscreate user}%
		{Tak}{dsadd user}%
		{Tak}{dsadd computer}%
		{Nie}{dscreate computer}
	\end{enumerate}
	
	\section{Obiekty Zasad Grup (GPO)}
	
	\section{Windows Instalacja zdalna}
	
	\section{Windows RAID}
	
	\section{Interpreter poleceń PowerShell}
	
	\section{Windows API}
	
\part{Linux}
	\section{Usługi graficzne Xwindow}
	
	\section{Linux ACL}
	
	\section{Linux RAID}
	
	\section{Linux LAMP}
	
	\section{Wielosystemowość}
	
	\section{Linux Kernel}


\end{document}