

% === DEFINICJA ZIELONEGO ==================== %
\definecolor{Gurin}{rgb}{0, 0.35, 0}

% === MAKRODEFINICJA POPRAWNEJ I ZŁEJ ODPOWIEDZI ==================== %
\newcommand{\Tak}[1] {
	\color{Gurin}{#1}
}
\newcommand{\Nie}[1] {
	\color{Red}{#1}
}

% === Porównywarka odpowiedzi
\newcommand{\answer}[2] {
	\ifnum\pdfstrcmp{#1}{Tak}=0
	\Tak{\item #2}
	\else
	\Nie{\item #2}
	\fi
}

% === MAKRODEFINICJA PYTANIA I ODPOWIEDZI =========================== %
% ** Pierwszy argument to treść pytania
% ** Kolejne argumenty to:
% * parzyste - Tak lub Nie
% * nieparzyste - Treści odpowiedzi
\newcommand{\question}[9] {
	\textbf{#1}
	\begin{enumerate}[a.]
		\answer{#2}{#3}
		\answer{#4}{#5}
		\answer{#6}{#7}
		\answer{#8}{#9}
	\end{enumerate}
}


% === MAKRODEFINICJE PYTANIA I WIĘCEJ NIŻ 4 ODPOWIEDZI =========================== %
% ** Pierwszy argument to treść pytania
% ** Kolejne argumenty to:
% * parzyste - Tak lub Nie
% * nieparzyste - Treści odpowiedzi

\newcommand{\questionVIII}[2]{ %
	\setkeys{Question}{#1} %
	\begin{enumerate}[a.]
		\setkeys{Answers}{#2}
	\end{enumerate}
}

\makeatletter
\define@key{Question}{question}{\textbf{#1}}
\define@key{Answers}{isTrue1}{\def\QA@isTrueI{#1}}
\define@key{Answers}{answer1}{\answer{\QA@isTrueI}{#1}}
\define@key{Answers}{isTrue2}{\def\QA@isTrueII{#1}}
\define@key{Answers}{answer2}{\answer{\QA@isTrueII}{#1}}
\define@key{Answers}{isTrue3}{\def\QA@isTrueIII{#1}}
\define@key{Answers}{answer3}{\answer{\QA@isTrueIII}{#1}}
\define@key{Answers}{isTrue4}{\def\QA@isTrueIV{#1}}
\define@key{Answers}{answer4}{\answer{\QA@isTrueIV}{#1}}
\define@key{Answers}{isTrue5}{\def\QA@isTrueV{#1}}
\define@key{Answers}{answer5}{\answer{\QA@isTrueV}{#1}}
\define@key{Answers}{isTrue6}{\def\QA@isTrueVI{#1}}
\define@key{Answers}{answer6}{\answer{\QA@isTrueVI}{#1}}
\define@key{Answers}{isTrue7}{\def\QA@isTrueVII{#1}}
\define@key{Answers}{answer7}{\answer{\QA@isTrueVII}{#1}}
\define@key{Answers}{isTrue8}{\def\QA@isTrueVIII{#1}}
\define@key{Answers}{answer8}{\answer{\QA@isTrueVIII}{#1}}
\define@key{Answers}{isTrue9}{\def\QA@isTrueIX{#1}}
\define@key{Answers}{answer9}{\answer{\QA@isTrueIX}{#1}}
\define@key{Answers}{isTrue10}{\def\QA@isTrueX{#1}}
\define@key{Answers}{answer10}{\answer{\QA@isTrueX}{#1}}
\define@key{Answers}{isTrue11}{\def\QA@isTrueXI{#1}}
\define@key{Answers}{answer11}{\answer{\QA@isTrueXI}{#1}}
\define@key{Answers}{isTrue12}{\def\QA@isTrueXII{#1}}
\define@key{Answers}{answer12}{\answer{\QA@isTrueXII}{#1}}

\makeatother














